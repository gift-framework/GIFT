% GIFT Framework - Supplement S7: Dimensional Observables
% Version 2.2.0 - November 2025
% Absolute Masses, Scale Bridge, and Cosmological Parameters

\documentclass[11pt,a4paper]{article}

% Essential packages
\usepackage[utf8]{inputenc}
\usepackage[T1]{fontenc}
\usepackage{amsmath,amssymb,amsthm}
\usepackage{mathtools}
\usepackage{booktabs}
\usepackage{array}
\usepackage{longtable}
\usepackage{hyperref}
\usepackage{cleveref}
\usepackage{fancyhdr}
\usepackage{geometry}
\usepackage{enumitem}
\usepackage{xcolor}
\usepackage{graphicx}

% Page geometry
\geometry{margin=2.5cm}

% Hyperref setup
\hypersetup{
    colorlinks=true,
    linkcolor=blue!70!black,
    citecolor=green!50!black,
    urlcolor=blue!70!black,
    pdftitle={GIFT Supplement S7: Dimensional Observables},
    pdfauthor={GIFT Framework}
}

% Header/Footer
\pagestyle{fancy}
\fancyhf{}
\fancyhead[L]{\textsc{GIFT Framework}}
\fancyhead[R]{\textsc{Supplement S7}}
\fancyfoot[C]{\thepage}

% Custom commands
\newcommand{\E}{E_8}
\newcommand{\EE}{E_8 \times E_8}
\newcommand{\Gtwo}{G_2}
\newcommand{\Kseven}{K_7}
\newcommand{\Hstar}{H^*}
\newcommand{\btwo}{b_2}
\newcommand{\bthree}{b_3}
\newcommand{\kappaT}{\kappa_T}
\newcommand{\sinW}{\sin^2\theta_W}
\newcommand{\deltaCP}{\delta_{CP}}
\newcommand{\Ngen}{N_{\text{gen}}}
\newcommand{\Mpl}{M_{\text{Pl}}}
\newcommand{\Ms}{M_s}
\newcommand{\Mgut}{M_{\text{GUT}}}
\newcommand{\vev}{v}
\newcommand{\LambdaGIFT}{\Lambda_{\text{GIFT}}}
\newcommand{\proven}{\textbf{PROVEN}}
\newcommand{\topological}{\textbf{TOPOLOGICAL}}
\newcommand{\derived}{\textbf{DERIVED}}
\newcommand{\theoretical}{\textbf{THEORETICAL}}
\newcommand{\exploratory}{\textbf{EXPLORATORY}}
\newcommand{\GeV}{\,\text{GeV}}
\newcommand{\MeV}{\,\text{MeV}}
\newcommand{\eV}{\,\text{eV}}

% Units
\newcommand{\kms}{\,\text{km/s/Mpc}}

% Theorem environments
\theoremstyle{definition}
\newtheorem{formula}{Formula}
\newtheorem{prediction}{Prediction}

\title{\textbf{Supplement S7: Dimensional Observables}\\[0.5em]
\large Absolute Masses, Scale Bridge, and Cosmological Parameters}
\author{GIFT Framework v2.2.0}
\date{November 2025}

\begin{document}

\maketitle

\begin{abstract}
This supplement extends the dimensionless predictions of the main document to absolute mass scales and cosmological observables, addressing the dimensional transmutation problem. We present the scale bridge mechanism connecting topological predictions to physical masses in GeV, and provide detailed comparisons with PDG 2024 and Planck 2020 data.
\end{abstract}

\tableofcontents
\newpage

%=============================================================================
\part{The Scale Bridge}
%=============================================================================

\section{Dimensional Transmutation Problem}

\subsection{The Challenge}

\textbf{Problem}: How do dimensionless topological numbers acquire dimensions (GeV)?

The GIFT framework predicts many dimensionless ratios exactly (e.g., $m_s/m_d = 20$), but connecting these to absolute masses requires a dimensional scale.

\subsection{Natural Scales}

The framework contains several natural scales:
\begin{itemize}
    \item Planck mass: $\Mpl \sim 10^{19}\GeV$
    \item String scale: $\Ms \sim \Mpl / e^8 \sim 10^{16}\GeV$
    \item GUT scale: $\Mgut \sim 10^{16}\GeV$
    \item Electroweak scale: $\vev \sim 246\GeV$
\end{itemize}

\section{The $\LambdaGIFT$ Structure}

\subsection{Formula}

\begin{equation}
\LambdaGIFT = \frac{21 \cdot e^8 \cdot 248}{7 \cdot \pi^4}
\end{equation}

\subsection{Components}

\begin{itemize}
    \item \textbf{21} = $\btwo(\Kseven)$: Gauge cohomology
    \item $\mathbf{e^8}$ = $\exp(\text{rank}(\E))$: Exponential hierarchy factor
    \item \textbf{248} = $\dim(\E)$: Gauge dimension
    \item \textbf{7} = $\dim(\Kseven)$: Manifold dimension
    \item $\pi^4$: Geometric normalization
\end{itemize}

\subsection{Numerical Value}

\begin{equation}
\LambdaGIFT = \frac{21 \times 2980.96 \times 248}{7 \times 97.409} = \frac{15{,}536{,}076}{681.86} \approx 1.632 \times 10^6
\end{equation}

\subsection{Derivation}

The $21 \times e^8$ structure emerges from:
\begin{enumerate}
    \item $\btwo = 21$ harmonic 2-forms (gauge sector)
    \item Exponential suppression from $\E$ rank
    \item Normalization by $\Kseven$ volume
\end{enumerate}

\section{From Dimensionless to Dimensional}

\subsection{VEV Derivation}

\begin{formula}[Higgs VEV]
\begin{equation}
\vev = \Mpl \cdot \left(\frac{\Mpl}{\Ms}\right)^{\tau/7} \cdot f(21 \cdot e^8)
\end{equation}
\end{formula}

\textbf{Parameters}:
\begin{itemize}
    \item $\Ms = \Mpl / e^8$ (string scale)
    \item $\tau/7 = \frac{3472}{891 \times 7} = \frac{3472}{6237} = 0.5567...$ (exact)
    \item $f(21 \times e^8)$: Normalization function
\end{itemize}

\textbf{Result}: $\vev \approx 246.87\GeV$

\textbf{Experimental}: $\vev = 246.22\GeV$

\textbf{Deviation}: 0.264\%

\subsection{Reference Scale Selection}

The electron mass $m_e$ serves as reference:
\begin{itemize}
    \item Most precisely measured fermion mass
    \item Stable particle
    \item All other masses expressed as ratios $\times\, m_e$
\end{itemize}

\section{Hierarchy Generation}

The exponential hierarchy $e^8 \approx 2981$ generates:
\begin{itemize}
    \item Planck/Electroweak ratio $\sim 10^{17}$
    \item Mass ratios between generations
    \item Yukawa coupling hierarchies
\end{itemize}

%=============================================================================
\part{Absolute Fermion Masses}
%=============================================================================

\section{Lepton Masses}

\subsection{Electron Mass (Reference)}

\begin{equation}
m_e = 0.51099895\MeV
\end{equation}

This is the reference scale. GIFT does not predict $m_e$ from first principles; it predicts all mass ratios relative to $m_e$.

\subsection{Muon Mass}

\textbf{From ratio}: $m_\mu/m_e = 27^\varphi = 207.012$

\begin{equation}
m_\mu = 207.012 \times m_e = 105.78\MeV
\end{equation}

\textbf{Experimental}: $105.658\MeV$

\textbf{Deviation}: 0.118\%

\subsection{Tau Mass (\proven)}

\textbf{From ratio}: $m_\tau/m_e = 3477$ (exact)

\begin{equation}
m_\tau = 3477 \times m_e = 1776.87\MeV
\end{equation}

\textbf{Experimental}: $1776.86\MeV$

\textbf{Deviation}: 0.004\%

\textbf{Status}: \proven{} (exact integer ratio)

\section{Quark Masses}

\subsection{Light Quarks}

\begin{table}[h]
\centering
\begin{tabular}{@{}llccc@{}}
\toprule
\textbf{Quark} & \textbf{Formula} & \textbf{GIFT (MeV)} & \textbf{PDG (MeV)} & \textbf{Dev.} \\
\midrule
$u$ & $\sqrt{14/3} \times$ MeV & 2.16 & $2.16 \pm 0.07$ & 0.0\% \\
$d$ & $\log(107) \times$ MeV & 4.67 & $4.67 \pm 0.09$ & 0.0\% \\
$s$ & $24\tau \times$ MeV & 93.5 & $93.4 \pm 0.8$ & 0.1\% \\
\bottomrule
\end{tabular}
\caption{Light quark masses. Note: $s$-quark formula uses $\tau = 3472/891 = 3.8967...$}
\end{table}

\subsection{Heavy Quarks}

\begin{table}[h]
\centering
\begin{tabular}{@{}llccc@{}}
\toprule
\textbf{Quark} & \textbf{Formula} & \textbf{GIFT (GeV)} & \textbf{PDG (GeV)} & \textbf{Dev.} \\
\midrule
$c$ & $(14-\pi)^3 \times 0.1$ & 1.280 & $1.27 \pm 0.02$ & 0.8\% \\
$b$ & $42 \times 99 \times$ MeV & 4.158 & $4.18 \pm 0.03$ & 0.5\% \\
$t$ & $(496/3)^\xi$ & 173.1 & $173.1 \pm 0.6$ & 0.0\% \\
\bottomrule
\end{tabular}
\caption{Heavy quark masses}
\end{table}

\subsection{Strange-Down Ratio (\proven)}

\begin{equation}
\frac{m_s}{m_d} = p_2^2 \times W_f = 4 \times 5 = 20
\end{equation}

\textbf{Status}: \proven{} (exact from topology)

\section{Neutrino Masses}

\subsection{Hierarchy Prediction}

\textbf{Prediction}: Normal hierarchy

\subsection{Mass Sum}

\begin{equation}
\Sigma m_\nu = 0.0587\eV
\end{equation}

\textbf{Current bound}: $\Sigma m_\nu < 0.12\eV$ (cosmological)

\textbf{Status}: Consistent

\subsection{Individual Masses}

\begin{table}[h]
\centering
\begin{tabular}{@{}lcc@{}}
\toprule
\textbf{Neutrino} & \textbf{Mass (eV)} & \textbf{Notes} \\
\midrule
$m_1$ & $\sim 0.001$ & Lightest \\
$m_2$ & $\sim 0.009$ & Solar splitting \\
$m_3$ & $\sim 0.05$ & Atmospheric splitting \\
\bottomrule
\end{tabular}
\caption{Neutrino mass estimates}
\end{table}

\subsection{Mechanism}

See-saw from $\Kseven$ volume:
\begin{equation}
m_\nu \sim \frac{\vev^2}{M_{K_7}}
\end{equation}

\textbf{Status}: \exploratory

%=============================================================================
\part{Boson Masses}
%=============================================================================

\section{W and Z Masses}

\subsection{W Boson Mass}

\begin{equation}
M_W = \frac{\vev}{2} \cdot g_2 = 80.38\GeV
\end{equation}

\textbf{Experimental}: $80.377 \pm 0.012\GeV$

\textbf{Deviation}: 0.004\%

\subsection{Z Boson Mass}

\begin{equation}
M_Z = \frac{M_W}{\cos\theta_W}
\end{equation}

Using $\sinW = 3/13$:
\begin{equation}
\cos^2\theta_W = 1 - \frac{3}{13} = \frac{10}{13}
\end{equation}
\begin{equation}
M_Z = M_W \cdot \sqrt{\frac{13}{10}} = 91.19\GeV
\end{equation}

\textbf{Experimental}: $91.188\GeV$

\textbf{Deviation}: 0.002\%

\section{Higgs Mass}

\subsection{Higgs Quartic Coupling (\proven)}

\begin{equation}
\lambda_H = \frac{\sqrt{17}}{32} = 0.12891
\end{equation}

\subsection{Higgs Mass}

\begin{equation}
m_H = \sqrt{2\lambda_H} \cdot \vev = \sqrt{2 \times 0.12891} \times 246.22 = 125.09\GeV
\end{equation}

\textbf{Experimental}: $125.25 \pm 0.17\GeV$

\textbf{Deviation}: 0.13\%

\subsection{Connection to $\lambda_H$}

The number $17 = \dim(\Gtwo) + \Ngen$ connects Higgs mass to $\Kseven$ geometry.

\section{Hypothetical BSM Masses}

\subsection{Second $\E$ Sector}

The hidden $\E$ sector may contain:
\begin{itemize}
    \item Dark matter candidates
    \item Heavy gauge bosons
    \item Moduli fields
\end{itemize}

\textbf{Characteristic scale}: $M \sim \Mpl / e^8 \sim 10^{16}\GeV$

\subsection{KK Modes}

Kaluza-Klein excitations from $\Kseven$:
\begin{equation}
m_{KK}^{(n)} \sim \frac{n}{R_{K_7}}
\end{equation}

\textbf{Typical scale}: $> 10^{16}\GeV$ (beyond collider reach)

%=============================================================================
\part{Cosmological Observables}
%=============================================================================

\section{Hubble Constant}

\subsection{The Hubble Tension}

\textbf{Early universe (CMB)}: $H_0 = 67.4 \pm 0.5\kms$

\textbf{Late universe (SNe)}: $H_0 = 73.0 \pm 1.0\kms$

\subsection{GIFT Ratio}

\begin{equation}
\frac{H_0^{\text{early}}}{H_0^{\text{late}}} = \frac{\bthree}{\Hstar} = \frac{77}{99} = 0.778
\end{equation}

\textbf{Observed ratio}: $67.4/73.0 = 0.923$

This ratio may contribute to understanding the tension but does not resolve it completely.

\subsection{Intermediate Value}

GIFT suggests:
\begin{equation}
H_0^{\text{GIFT}} = 69.8\kms
\end{equation}

This lies between early and late measurements.

\textbf{Status}: \exploratory

\section{Dark Energy Density (\proven)}

\subsection{Formula}

\begin{equation}
\Omega_{DE} = \ln(2) \times \frac{98}{99} = 0.686146
\end{equation}

\subsection{Triple Origin of $\ln(2)$}

\begin{align}
\ln(p_2) &= \ln(2) \\
\ln\left(\frac{\dim(\EE)}{\dim(\E)}\right) &= \ln\left(\frac{496}{248}\right) = \ln(2) \\
\ln\left(\frac{\dim(\Gtwo)}{\dim(\Kseven)}\right) &= \ln\left(\frac{14}{7}\right) = \ln(2)
\end{align}

\subsection{Comparison}

\textbf{Experimental (Planck 2020)}: $\Omega_{DE} = 0.6847 \pm 0.0073$

\textbf{GIFT}: 0.6861

\textbf{Deviation}: 0.21\%

\textbf{Status}: \proven

\section{Dark Matter Density}

\subsection{Formula}

\begin{equation}
\Omega_{DM} = \frac{\btwo(\Kseven)}{\bthree(\Kseven)} = \frac{21}{77} = 0.2727
\end{equation}

\subsection{Second $\E$ Interpretation}

Dark matter may reside in the hidden $\E$ sector:
\begin{itemize}
    \item Gauge-neutral under visible $\E$
    \item Gravitationally coupled
    \item Topologically protected
\end{itemize}

\subsection{Comparison}

\textbf{Experimental}: $\Omega_{DM} = 0.265 \pm 0.007$

\textbf{GIFT}: 0.2727

\textbf{Deviation}: 2.9\%

\textbf{Status}: \theoretical

\section{Cosmological Constant}

\subsection{From $\Kseven$ Volume}

\begin{equation}
\Lambda_{\text{cosmo}} \sim \frac{1}{V(\Kseven)^2}
\end{equation}

\subsection{The Cosmological Constant Problem}

GIFT suggests vacuum energy is related to topological structure, but does not fully resolve the $10^{120}$ discrepancy.

\textbf{Status}: \exploratory

%=============================================================================
\part{Scaling Relations}
%=============================================================================

\section{The $\tau$ Parameter in Mass Hierarchies}

\subsection{Definition}

\begin{equation}
\tau = \frac{3472}{891} = 3.8967452...
\end{equation}

\textbf{Status}: \proven{} (exact rational)

\subsection{Application to Quark Masses}

Strange quark mass:
\begin{equation}
m_s = 24 \times \tau\MeV = 24 \times 3.8967 = 93.5\MeV
\end{equation}

\subsection{Prime Factorization}

\begin{equation}
\tau = \frac{2^4 \times 7 \times 31}{3^4 \times 11}
\end{equation}

All factors are framework constants.

\section{Hausdorff Dimension Relation}

\subsection{Discovery}

\begin{equation}
\frac{D_H}{\tau} = \frac{\ln(2)}{\pi} = 0.2206
\end{equation}

where $D_H \approx 0.856$ is the Hausdorff dimension of observable space.

\subsection{With Exact $\tau$}

\begin{equation}
D_H = \frac{3472}{891} \times \frac{\ln(2)}{\pi} = \frac{3472 \ln(2)}{891\pi}
\end{equation}

\textbf{Deviation}: 0.41\%

\subsection{Interpretation}

\begin{itemize}
    \item $D_H$: Scaling dimension of observable space
    \item $\tau$: Hierarchical parameter
    \item $\ln(2)$: Dark energy connection
    \item $\pi$: Geometric constant
\end{itemize}

\section{RG Flow and Mass Running}

\subsection{Running Masses}

Quark masses run with energy scale:
\begin{equation}
m_q(\mu) = m_q(m_q) \left(\frac{\alpha_s(\mu)}{\alpha_s(m_q)}\right)^{\gamma_m/\beta_0}
\end{equation}

\subsection{GIFT Consistency}

All mass predictions must be compared at consistent renormalization scale. PDG values are typically given at the quark mass itself ($\overline{\text{MS}}$ scheme).

%=============================================================================
\part{Experimental Comparison}
%=============================================================================

\section{Mass Predictions vs PDG 2024}

\subsection{Leptons}

\begin{table}[h]
\centering
\begin{tabular}{@{}lccc@{}}
\toprule
\textbf{Particle} & \textbf{GIFT (MeV)} & \textbf{PDG 2024} & \textbf{Dev.} \\
\midrule
$e$ & reference & 0.510999 & --- \\
$\mu$ & 105.78 & 105.658 & 0.12\% \\
$\tau$ & 1776.87 & 1776.86 & 0.004\% \\
\bottomrule
\end{tabular}
\caption{Lepton mass predictions}
\end{table}

\subsection{Quarks}

\begin{table}[h]
\centering
\begin{tabular}{@{}lccc@{}}
\toprule
\textbf{Particle} & \textbf{GIFT (MeV)} & \textbf{PDG 2024} & \textbf{Dev.} \\
\midrule
$u$ & 2.16 & $2.16 \pm 0.07$ & 0.0\% \\
$d$ & 4.67 & $4.67 \pm 0.09$ & 0.0\% \\
$s$ & 93.5 & $93.4 \pm 0.8$ & 0.1\% \\
$c$ & 1280 & $1270 \pm 20$ & 0.8\% \\
$b$ & 4158 & $4180 \pm 30$ & 0.5\% \\
$t$ & 173100 & $173100 \pm 600$ & 0.0\% \\
\bottomrule
\end{tabular}
\caption{Quark mass predictions}
\end{table}

\subsection{Bosons}

\begin{table}[h]
\centering
\begin{tabular}{@{}lccc@{}}
\toprule
\textbf{Particle} & \textbf{GIFT (GeV)} & \textbf{PDG 2024} & \textbf{Dev.} \\
\midrule
$W$ & 80.38 & 80.377 & 0.004\% \\
$Z$ & 91.19 & 91.188 & 0.002\% \\
$H$ & 125.09 & 125.25 & 0.13\% \\
\bottomrule
\end{tabular}
\caption{Boson mass predictions}
\end{table}

\section{Cosmological Predictions vs Planck 2020}

\begin{table}[h]
\centering
\begin{tabular}{@{}lccc@{}}
\toprule
\textbf{Parameter} & \textbf{GIFT} & \textbf{Planck 2020} & \textbf{Dev.} \\
\midrule
$\Omega_{DE}$ & 0.6861 & $0.6847 \pm 0.0073$ & 0.21\% \\
$\Omega_{DM}$ & 0.2727 & $0.265 \pm 0.007$ & 2.9\% \\
$H_0$ & 69.8 & $67.4 \pm 0.5$ & 3.6\% \\
$n_s$ & 0.9649 & $0.9649 \pm 0.0042$ & 0.00\% \\
\bottomrule
\end{tabular}
\caption{Cosmological predictions vs Planck 2020}
\end{table}

\section{DESI DR2 Compatibility}

\subsection{Torsion Constraint}

\textbf{DESI bound}: $|T|^2 < 10^{-3}$

\textbf{GIFT value}: $\kappaT^2 = (1/61)^2 = 2.69 \times 10^{-4}$

\textbf{Result}: Well within bounds

\subsection{$w_0$-$w_a$ Constraints}

DESI DR2 suggests $w_0 \neq -1$ at $\sim 2\sigma$. GIFT predicts deviations from $\Lambda$CDM through torsion corrections.

\section{Precision Summary Table}

\begin{table}[h]
\centering
\begin{tabular}{@{}lccc@{}}
\toprule
\textbf{Category} & $N$ & \textbf{Mean Dev.} & \textbf{Best} \\
\midrule
Lepton masses & 3 & 0.04\% & $m_\tau$ \\
Quark masses & 6 & 0.23\% & $u$, $d$, $t$ \\
Boson masses & 3 & 0.05\% & $Z$ \\
Cosmology & 4 & 1.7\% & $n_s$ \\
\bottomrule
\end{tabular}
\caption{Precision summary by category}
\end{table}

%=============================================================================
\part{Limitations}
%=============================================================================

\section{Scale Bridge Assumptions}

\subsection{Current Limitations}

\begin{enumerate}
    \item Electron mass $m_e$ is input (not predicted)
    \item Planck mass $\Mpl$ is input
    \item Dimensional transmutation mechanism incomplete
    \item Some mass formulas are heuristic
\end{enumerate}

\subsection{What GIFT Predicts vs.\ Assumes}

\textbf{Predicted}:
\begin{itemize}
    \item All mass ratios (dimensionless)
    \item Gauge couplings at $M_Z$
    \item Mixing angles and phases
    \item Cosmological ratios
\end{itemize}

\textbf{Assumed}:
\begin{itemize}
    \item Reference scale ($m_e$ or $\vev$)
    \item Fundamental constants ($c$, $\hbar$, $G$)
\end{itemize}

\section{Theoretical Uncertainties}

\subsection{Higher-Order Corrections}

\begin{itemize}
    \item QCD corrections to quark masses
    \item Electroweak radiative corrections
    \item Threshold effects at mass scales
\end{itemize}

\subsection{Non-Perturbative Effects}

\begin{itemize}
    \item Confinement corrections to light quarks
    \item Instanton contributions
    \item Strong CP effects
\end{itemize}

\section{Future Improvements}

\subsection{Needed Developments}

\begin{enumerate}
    \item First-principles derivation of electron mass
    \item Complete dimensional transmutation mechanism
    \item Moduli stabilization explanation
    \item Connection to string/M-theory scales
\end{enumerate}

\subsection{Experimental Tests}

\begin{itemize}
    \item Precision lepton mass measurements
    \item Lattice QCD quark mass determinations
    \item Higgs self-coupling at future colliders
    \item Cosmological parameter refinement
\end{itemize}

%=============================================================================
\section*{References}
%=============================================================================

\begin{enumerate}
    \item Particle Data Group (2024). \textit{Review of Particle Physics}.
    \item Planck Collaboration (2020). Cosmological parameters.
    \item DESI Collaboration (2025). DR2 cosmological constraints.
    \item Lattice QCD FLAG review (2024). Quark masses.
    \item Weinberg, S. (1972). \textit{Gravitation and Cosmology}.
\end{enumerate}

\vspace{2em}
\hrule
\vspace{1em}
\noindent\textbf{Document Version}: 2.2.0\\
\textbf{Last Updated}: November 2025\\
\textbf{GIFT Framework}: \url{https://github.com/gift-framework/GIFT}

\vspace{0.5em}
\noindent\textit{This supplement contains dimensional observables and scale bridge content extracted from former S9 (Extensions).}

\end{document}
