\documentclass[11pt,a4paper]{article}

% ============================================
% ENCODING & FONTS
% ============================================
\usepackage[utf8]{inputenc}
\usepackage[T1]{fontenc}
\usepackage{lmodern}

% ============================================
% PAGE LAYOUT (Golden Ratio)
% ============================================
\usepackage[margin=1.618cm, top=2.618cm, bottom=2.618cm]{geometry}

% ============================================
% ESSENTIAL PACKAGES
% ============================================
\usepackage{float}
\usepackage{caption}
\usepackage{subcaption}
\usepackage{setspace}
\usepackage{fancyhdr}
\usepackage{xcolor}
\usepackage{hyperref}
\usepackage{csquotes}
\usepackage{amsmath}
\usepackage{amssymb}
\usepackage{booktabs}
\usepackage{longtable}
\usepackage{array}
\usepackage{tikz}
\usepackage{graphicx}

% ============================================
% HEADER/FOOTER CONFIGURATION
% ============================================
\setlength{\headheight}{14pt}
\pagestyle{fancy}
\fancyhf{}
\fancyhead[L]{Your Document Title}  % Modify this
\fancyhead[R]{\thepage}
\renewcommand{\headrulewidth}{0.2pt}

% ============================================
% HYPERREF CONFIGURATION
% ============================================
\hypersetup{
    colorlinks=true,
    linkcolor=blue,
    citecolor=blue,
    urlcolor=blue,
    pdftitle={Your Document Title},  % Modify this
    pdfauthor={Your Name}  % Modify this
}

% ============================================
% SPACING AND FORMATTING
% ============================================
\setstretch{1.2}
\setlength{\parskip}{0.4em}
\setlength{\parindent}{0pt}

% ============================================
% TITLE FORMATTING
% ============================================
\usepackage{titling}
\pretitle{\LARGE\bfseries}
\posttitle{\vspace{-0.4em}}
\preauthor{}
\postauthor{}
\predate{}
\postdate{}
\setlength{\droptitle}{-2.0em}

% ============================================
% CUSTOM COMMANDS (Optional - Add your own)
% ============================================
% Example mathematical shortcuts:
% \newcommand{\E}{\mathrm{E}}
% \newcommand{\Gtwo}{\mathrm{G}_2}

% ============================================
% TITLE PAGE SETUP
% ============================================
\title{%
\LARGE\textbf{Your Main Title Here:\\
Optional Subtitle}
}
\author{}
\date{}

% ============================================
% DOCUMENT START
% ============================================
\begin{document}

% ============================================
% TITLE PAGE WITH CUSTOM LAYOUT
% ============================================
\maketitle
\noindent\rule{\textwidth}{0.2pt}

\vspace{0.5em}

{Your Name\\
Your Affiliation\\
your.email@example.com}

\vfill

\begin{abstract}
Your abstract text goes here. This template provides a clean, professional layout with golden ratio-based margins, proper spacing, and header/footer configuration. It is designed to be simple and compatible with Markdown or Quarto markdown workflows.

\vspace{0.5em}

\textbf{Keywords}: keyword1, keyword2, keyword3
\end{abstract}

\vfill

\noindent\rule{\textwidth}{0.2pt}

\newpage

% ============================================
% TABLE OF CONTENTS (Optional)
% ============================================
\tableofcontents
\newpage

% ============================================
% MAIN CONTENT
% ============================================

\section{Introduction}

Your content begins here. This template maintains consistent spacing and professional formatting throughout.

\subsection{Subsection Example}

Example text with mathematical notation: $E = mc^2$

Block equation example:
\begin{equation}
\int_{-\infty}^{\infty} e^{-x^2} dx = \sqrt{\pi}
\end{equation}

\subsection{Lists and Formatting}

Itemized list:
\begin{itemize}
    \item First item
    \item Second item
    \item Third item
\end{itemize}

Enumerated list:
\begin{enumerate}
    \item First point
    \item Second point
    \item Third point
\end{enumerate}

\section{Tables and Figures}

\subsection{Table Example}

\begin{table}[H]
\centering
\begin{tabular}{lll}
\toprule
\textbf{Column 1} & \textbf{Column 2} & \textbf{Column 3} \\
\midrule
Data 1 & Data 2 & Data 3 \\
Data 4 & Data 5 & Data 6 \\
\bottomrule
\end{tabular}
\caption{Example table with booktabs formatting}
\end{table}

\subsection{Figure Example}

\begin{figure}[H]
\centering
% \includegraphics[width=0.8\textwidth]{your-figure.png}
\caption{Example figure caption}
\end{figure}

\section{Citations and References}

Example citation reference \cite{example2024}.

\section{Conclusion}

Your concluding remarks go here.

% ============================================
% BIBLIOGRAPHY (Optional)
% ============================================
\begin{thebibliography}{99}

\bibitem{example2024}
Author, A. (2024). Title of the work. \emph{Journal Name}, Volume(Issue), pages.

\end{thebibliography}

% ============================================
% APPENDICES (Optional)
% ============================================
% \appendix
% \section{Appendix Title}
% Appendix content here.

\end{document}

