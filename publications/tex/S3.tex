\documentclass[11pt,a4paper]{article}

% ============================================
% ENCODING & FONTS
% ============================================
\usepackage[utf8]{inputenc}
\usepackage[T1]{fontenc}
\usepackage{lmodern}

% ============================================
% PAGE LAYOUT (Golden Ratio)
% ============================================
\usepackage[margin=1.618cm, top=2.618cm, bottom=2.618cm]{geometry}

% ============================================
% ESSENTIAL PACKAGES
% ============================================
\usepackage{float}
\usepackage{caption}
\usepackage{subcaption}
\usepackage{setspace}
\usepackage{fancyhdr}
\usepackage{xcolor}
\usepackage{hyperref}
\usepackage{csquotes}
\usepackage{amsmath}
\usepackage{amssymb}
\usepackage{booktabs}
\usepackage{longtable}
\usepackage{array}
\usepackage{tikz}
\usepackage{graphicx}

% ============================================
% HEADER/FOOTER CONFIGURATION
% ============================================
\setlength{\headheight}{14pt}
\pagestyle{fancy}
\fancyhf{}
\fancyhead[L]{GIFT Framework - Supplement S3}
\fancyhead[R]{\thepage}
\renewcommand{\headrulewidth}{0.2pt}

% ============================================
% HYPERREF CONFIGURATION
% ============================================
\hypersetup{
    colorlinks=true,
    linkcolor=blue,
    citecolor=blue,
    urlcolor=blue,
    pdftitle={GIFT Supplement S3: Torsional Dynamics},
    pdfauthor={Brieuc de La Fournière}
}

% ============================================
% SPACING AND FORMATTING
% ============================================
\setstretch{1.2}
\setlength{\parskip}{0.4em}
\setlength{\parindent}{0pt}

% ============================================
% TITLE FORMATTING
% ============================================
\usepackage{titling}
\pretitle{\LARGE\bfseries}
\posttitle{\vspace{-0.4em}}
\preauthor{}
\postauthor{}
\predate{}
\postdate{}
\setlength{\droptitle}{-2.0em}

% ============================================
% CUSTOM COMMANDS
% ============================================
\newcommand{\E}{\mathrm{E}}
\newcommand{\Gtwo}{\mathrm{G}_2}
\newcommand{\Kseven}{K_7}
\newcommand{\AdS}{\mathrm{AdS}}
\newcommand{\dimE}{\mathrm{dim}}
\newcommand{\Weyl}{\mathrm{Weyl}}
\newcommand{\rk}{\mathrm{rank}}
\newcommand{\SM}{\mathrm{SM}}
\newcommand{\SU}{\mathrm{SU}}
\newcommand{\SO}{\mathrm{SO}}
\newcommand{\U}{\mathrm{U}}
\newcommand{\Spin}{\mathrm{Spin}}
\newcommand{\Sp}{\mathrm{Sp}}
\newcommand{\Aut}{\mathrm{Aut}}
\newcommand{\Der}{\mathrm{Der}}
\newcommand{\Vol}{\mathrm{Vol}}
\newcommand{\Ric}{\mathrm{Ric}}
\newcommand{\Riem}{\mathrm{Riem}}
\newcommand{\Tr}{\mathrm{Tr}}
\newcommand{\Det}{\mathrm{det}}
\newcommand{\Index}{\mathrm{Index}}

% ============================================
% TITLE PAGE SETUP
% ============================================
\title{%
\LARGE\textbf{Supplement S3: Torsional Dynamics\\[0.5em]
\large Complete Formulation of Torsional Geodesic Dynamics\\and Connection to RG Flow}
}
\author{}
\date{}

% ============================================
% DOCUMENT START
% ============================================
\begin{document}

% ============================================
% TITLE PAGE WITH CUSTOM LAYOUT
% ============================================
\maketitle
\noindent\rule{\textwidth}{0.2pt}

\vspace{0.5em}

{GIFT Framework v2.1\\
Geometric Information Field Theory}

\vfill

\begin{abstract}
We present the complete dynamical framework connecting static topological structure to physical evolution. Section 1 develops the torsion tensor from the non-closure of the G\textsubscript{2} 3-form, establishing its physical origin and component structure. Section 2 derives the geodesic flow equation from variational principles and establishes conservation laws. Section 3 identifies geodesic flow with renormalization group evolution, providing geometric foundations for quantum field theory β-functions. Key results include the torsion magnitude \(|\mathbf{T}| \approx 0.0164\), the torsional geodesic equation, and the ultra-slow flow velocity \(|v| \approx 0.015\) ensuring constant variation bounds.

\vspace{0.5em}

\textbf{Keywords}: Torsion tensor, geodesic flow, renormalization group, β-functions, constant variation

\vspace{0.5em}

\textit{This supplement provides the mathematical formulation of torsional geodesic dynamics underlying the GIFT framework. For \(\Kseven\) metric construction, see Supplement S2. For physical applications to observables, see Supplement S5.}
\end{abstract}

\vfill

\noindent\rule{\textwidth}{0.2pt}

\newpage

% ============================================
% TABLE OF CONTENTS
% ============================================
\tableofcontents

\vfill
% ============================================
% STATUS CLASSIFICATIONS
% ============================================
\section*{Status Classifications}
\addcontentsline{toc}{section}{Status Classifications}

\begin{itemize}
    \item \textbf{PROVEN}: Exact mathematical result with rigorous derivation
    \item \textbf{TOPOLOGICAL}: Direct consequence of manifold structure
    \item \textbf{THEORETICAL}: Has theoretical justification, numerical verification pending
    \item \textbf{PHENOMENOLOGICAL}: Constrained by experimental data
\end{itemize}

\newpage

% ============================================
% MAIN CONTENT
% ============================================

\section{Torsion Tensor}

\subsection{Definition and Properties}

\subsubsection{Torsion in Differential Geometry}

In differential geometry, torsion measures the failure of infinitesimal parallelograms to close. For a connection \(\nabla\) on a manifold \(M\), the torsion tensor \(T\) is defined by:

\[
T(X, Y) = \nabla_X Y - \nabla_Y X - [X, Y]
\]

for vector fields \(X\), \(Y\). In components:

\[
T^k_{ij} = \Gamma^k_{ij} - \Gamma^k_{ji}
\]

where \(\Gamma^k_{ij}\) are the connection coefficients.

\subsubsection{Torsion-Free vs Torsionful Connections}

\textbf{Levi-Civita connection}: The unique torsion-free, metric-compatible connection:
\begin{itemize}
    \item \(T^k_{ij} = 0\) (torsion-free)
    \item \(\nabla_k g_{ij} = 0\) (metric-compatible)
\end{itemize}

\textbf{Torsionful connection}: Preserves metric compatibility but allows non-zero torsion:
\begin{itemize}
    \item \(T^k_{ij} \neq 0\)
    \item \(\nabla_k g_{ij} = 0\) (metric-compatible)
\end{itemize}

The GIFT framework employs a torsionful connection arising from the non-closure of the \(\Gtwo\) 3-form.

\subsubsection{Contorsion Tensor}

The difference between a torsionful connection and Levi-Civita is the contorsion tensor \(K\):

\[
\Gamma^k_{ij} = \mathring{\Gamma}^k_{ij} + K^k_{ij}
\]

where \(\mathring{\Gamma}\) denotes Levi-Civita. The contorsion relates to torsion by:

\[
K^k_{ij} = \frac{1}{2}(T^k_{ij} + T_i{}^k{}_j + T_j{}^k{}_i)
\]

For totally antisymmetric torsion \(T_{ijk} = T_{[ijk]}\):

\[
K^k_{ij} = \frac{1}{2} T^k_{ij}
\]

\subsubsection{Torsion Classes for G\textsubscript{2} Manifolds}

On a 7-manifold with \(\Gtwo\) structure, torsion decomposes into four irreducible \(\Gtwo\) representations:

\[
T \in W_1 \oplus W_7 \oplus W_{14} \oplus W_{27}
\]

\begin{table}[H]
\centering
\begin{tabular}{lll}
\toprule
\textbf{Class} & \textbf{Dimension} & \textbf{Characterization} \\
\midrule
\(W_1\) & 1 & \(d\varphi \wedge \varphi \neq 0\) \\
\(W_7\) & 7 & \(*d\varphi - \theta \wedge \varphi\) for 1-form \(\theta\) \\
\(W_{14}\) & 14 & Traceless part of \(d*\varphi\) \\
\(W_{27}\) & 27 & Symmetric traceless \\
\bottomrule
\end{tabular}
\caption{Torsion classes for \(\Gtwo\) manifolds}
\end{table}

\textbf{Torsion-free \(\Gtwo\)}: All classes vanish (\(d\varphi = 0\), \(d*\varphi = 0\))

\textbf{GIFT framework}: Controlled non-zero torsion in specific classes generates physical interactions.

\subsection{Physical Origin}

\subsubsection{G\textsubscript{2} Holonomy and the 3-Form}

A 7-manifold \(M\) has \(\Gtwo\) holonomy if it admits a parallel 3-form \(\varphi\):

\[
\nabla \varphi = 0
\]

This is equivalent to the closure conditions:

\[
d\varphi = 0, \quad d*\varphi = 0
\]

Such manifolds are Ricci-flat and have trivial canonical bundle.

\subsubsection{Non-Closure as Source of Interactions}

Physical interactions require departure from the torsion-free condition. The framework introduces controlled non-closure:

\[
|d\varphi|^2 + |d*\varphi|^2 = \epsilon^2
\]

where \(\epsilon\) is small but non-zero.

\textbf{Physical motivation}: A perfectly torsion-free manifold has no geometric coupling between sectors. Torsion provides the mechanism for particle interactions.

\textbf{Numerical value}: From metric reconstruction (Supplement S2):

\[
\epsilon = 0.0164 \pm 0.002
\]

\subsubsection{Torsion from Non-Closure}

The torsion tensor components arise from the 4-form \(d\varphi\) and 5-form \(d*\varphi\):

\[
T_{ijk} \sim (d\varphi)_{lijk} g^{lm} + \text{(dual terms)}
\]

The precise relation involves the \(\Gtwo\) structure equations and metric factors.

\subsubsection{Global Torsion Magnitude}

The global torsion norm:

\[
|\mathbf{T}| = \sqrt{|d\varphi|^2 + |d*\varphi|^2} \approx 0.0164
\]

\textbf{Physical interpretation}: This small value ensures:
\begin{enumerate}
    \item Approximate \(\Gtwo\) structure preservation
    \item Ultra-slow evolution of constants
    \item Consistency with experimental bounds on constant variation
\end{enumerate}

\subsection{Component Analysis}

\subsubsection{Coordinate System}

The \(\Kseven\) metric is expressed in coordinates \((e, \pi, \varphi)\) with physical interpretation:

\begin{table}[H]
\centering
\begin{tabular}{lll}
\toprule
\textbf{Coordinate} & \textbf{Physical Sector} & \textbf{Range} \\
\midrule
\(e\) & Electromagnetic & [0.1, 2.0] \\
\(\pi\) & Hadronic/strong & [0.1, 3.0] \\
\(\varphi\) & Electroweak/Higgs & [0.1, 1.5] \\
\bottomrule
\end{tabular}
\caption{Physical coordinates on \(\Kseven\)}
\end{table}

These span a 3-dimensional subspace encoding essential parameter information.

\subsubsection{Torsion Tensor Components}

From numerical metric reconstruction, the key torsion components are:

\begin{align}
T_{e\varphi,\pi} &= -4.89 \pm 0.02 \\
T_{\pi\varphi,e} &= -0.45 \pm 0.01 \\
T_{e\pi,\varphi} &= (3.1 \pm 0.3) \times 10^{-5}
\end{align}

\subsubsection{Hierarchical Structure}

The torsion components span four orders of magnitude:

\begin{table}[H]
\centering
\begin{tabular}{lll}
\toprule
\textbf{Component} & \textbf{Magnitude} & \textbf{Physical Role} \\
\midrule
\(T_{e\varphi,\pi}\) & \(\sim 5\) & Mass hierarchies (large ratios) \\
\(T_{\pi\varphi,e}\) & \(\sim 0.5\) & CP violation phase \\
\(T_{e\pi,\varphi}\) & \(\sim 10^{-5}\) & Jarlskog invariant \\
\bottomrule
\end{tabular}
\caption{Torsion hierarchy and physical interpretation}
\end{table}

\textbf{Key insight}: The torsion hierarchy directly encodes the observed hierarchy of physical observables.

\subsubsection{Physical Interpretation}

\textbf{\(T_{e\varphi,\pi} \approx -4.89\) (large)}:
\begin{itemize}
    \item Drives geodesics in \((e,\varphi)\) plane
    \item Source of mass hierarchies like \(m_\tau/m_e = 3477\)
    \item Large torsion amplifies path lengths
\end{itemize}

\textbf{\(T_{\pi\varphi,e} \approx -0.45\) (moderate)}:
\begin{itemize}
    \item Torsional twist in \((\pi,\varphi)\) sector
    \item Source of CP violation \(\delta_{\mathrm{CP}} = 197^\circ\)
    \item Accumulated geometric phase
\end{itemize}

\textbf{\(T_{e\pi,\varphi} \approx 3\times 10^{-5}\) (tiny)}:
\begin{itemize}
    \item Weak electromagnetic-hadronic coupling
    \item Related to Jarlskog invariant \(J \approx 3\times 10^{-5}\)
    \item Suppressed CP violation in quark sector
\end{itemize}

\subsection{Symmetry Properties}

\subsubsection{Antisymmetry}

The torsion tensor is antisymmetric in its lower indices:

\[
T_{ijk} = -T_{jik}
\]

This follows from the definition \(T^k_{ij} = \Gamma^k_{ij} - \Gamma^k_{ji}\).

\subsubsection{Bianchi-Type Identities}

Torsion satisfies algebraic Bianchi identities:

\[
T_{[ijk]} = T_{ijk} + T_{jki} + T_{kij} = 0
\]

(cyclic sum vanishes for metric-compatible connection)

\subsubsection{G\textsubscript{2} Transformation Properties}

Under \(\Gtwo\) structure group transformations:

\[
T_{ijk} \to g_i{}^{i'} g_j{}^{j'} g_k{}^{k'} T_{i'j'k'}
\]

where \(g \in \Gtwo \subset \SO(7)\).

\subsubsection{Conservation Laws}

The torsion tensor satisfies differential Bianchi identities relating its covariant derivatives to curvature:

\[
\nabla_{[i} T_{jk]l} = R_{[ijk]l} - \text{(torsion squared terms)}
\]

These constrain the evolution of torsion components.

\section{Geodesic Flow Equation}

\subsection{Derivation from Action}

\subsubsection{Geodesic Action}

Consider a curve \(x^k(\lambda)\) on \(\Kseven\) parametrized by affine parameter \(\lambda\). The geodesic action is:

\[
S = \int d\lambda \, \mathcal{L} = \int d\lambda \, \frac{1}{2} g_{ij} \frac{dx^i}{d\lambda} \frac{dx^j}{d\lambda}
\]

Using dot notation \(\dot{x}^i = dx^i/d\lambda\):

\[
S = \int d\lambda \, \frac{1}{2} g_{ij} \dot{x}^i \dot{x}^j
\]

\subsubsection{Euler-Lagrange Equations}

The Euler-Lagrange equations:

\[
\frac{d}{d\lambda} \left( \frac{\partial \mathcal{L}}{\partial \dot{x}^k} \right) - \frac{\partial \mathcal{L}}{\partial x^k} = 0
\]

\textbf{Calculation}:

\[
\frac{\partial \mathcal{L}}{\partial \dot{x}^k} = g_{kj} \dot{x}^j
\]

\[
\frac{d}{d\lambda}(g_{kj} \dot{x}^j) = \partial_i g_{kj} \dot{x}^i \dot{x}^j + g_{kj} \ddot{x}^j
\]

\[
\frac{\partial \mathcal{L}}{\partial x^k} = \frac{1}{2} \partial_k g_{ij} \dot{x}^i \dot{x}^j
\]

\textbf{Euler-Lagrange result}:

\[
g_{kj} \ddot{x}^j + \left(\partial_i g_{kj} - \frac{1}{2} \partial_k g_{ij}\right) \dot{x}^i \dot{x}^j = 0
\]

\subsubsection{Standard Geodesic Equation}

Multiplying by \(g^{mk}\):

\[
\ddot{x}^m + \Gamma^m_{ij} \dot{x}^i \dot{x}^j = 0
\]

where \(\Gamma^m_{ij}\) is the Christoffel symbol:

\[
\Gamma^m_{ij} = \frac{1}{2} g^{mk}(\partial_i g_{kj} + \partial_j g_{ik} - \partial_k g_{ij})
\]

\subsubsection{Torsional Modification}

For locally constant metric (\(\partial_k g_{ij} \approx 0\) over coordinate patches):

\[
\Gamma^m_{ij}|_{\text{Levi-Civita}} \approx 0
\]

The effective connection becomes purely torsional:

\[
\boxed{\Gamma^k_{ij} = -\frac{1}{2} g^{kl} T_{ijl}}
\]

\textbf{Physical meaning}: Acceleration arises from torsion, not metric gradients.

\subsection{Torsional Geodesic Equation}

\subsubsection{Main Result}

Substituting the torsional connection into the geodesic equation:

\[
\boxed{\frac{d^2 x^k}{d\lambda^2} = \frac{1}{2} g^{kl} T_{ijl} \frac{dx^i}{d\lambda} \frac{dx^j}{d\lambda}}
\]

This is the \textbf{torsional geodesic equation} governing parameter evolution.

\subsubsection{Component Form}

In explicit component notation for \((e, \pi, \varphi)\) coordinates:

\[
\ddot{e} = \frac{1}{2} g^{em} T_{ijm} \dot{x}^i \dot{x}^j
\]

\[
\ddot{\pi} = \frac{1}{2} g^{\pi m} T_{ijm} \dot{x}^i \dot{x}^j
\]

\[
\ddot{\varphi} = \frac{1}{2} g^{\varphi m} T_{ijm} \dot{x}^i \dot{x}^j
\]

\subsubsection{Quadratic Velocity Dependence}

The right-hand side is quadratic in velocities:

\[
\ddot{x}^k \propto \dot{x}^i \dot{x}^j
\]

This produces nonlinear dynamics analogous to:
\begin{itemize}
    \item Geodesic deviation in general relativity
    \item Nonlinear β-function evolution in QFT
    \item Chaotic dynamics in mechanical systems
\end{itemize}

\subsubsection{Physical Interpretation}

\begin{table}[H]
\centering
\begin{tabular}{lll}
\toprule
\textbf{Quantity} & \textbf{Geometric} & \textbf{Physical} \\
\midrule
\(x^k(\lambda)\) & Position on \(\Kseven\) & Coupling constant value \\
\(\lambda\) & Curve parameter & RG scale \(\ln(\mu)\) \\
\(\dot{x}^k\) & Velocity & β-function \\
\(\ddot{x}^k\) & Acceleration & β-function derivative \\
\(T_{ijl}\) & Torsion & Interaction strength \\
\(g^{kl}\) & Inverse metric & Coupling response \\
\bottomrule
\end{tabular}
\caption{Geometric-physical dictionary}
\end{table}

\subsection{Conservation Laws}

\subsubsection{Energy Conservation}

For affine parameter \(\lambda\), the kinetic energy:

\[
E = g_{ij} \frac{dx^i}{d\lambda} \frac{dx^j}{d\lambda}
\]

is conserved along geodesics:

\[
\frac{dE}{d\lambda} = 0
\]

\textbf{Proof}:

\[
\frac{dE}{d\lambda} = 2 g_{ij} \dot{x}^i \ddot{x}^j + \partial_k g_{ij} \dot{x}^k \dot{x}^i \dot{x}^j
\]

Using the geodesic equation and metric compatibility:

\[
= 2 g_{ij} \dot{x}^i \left(-\Gamma^j_{kl} \dot{x}^k \dot{x}^l\right) + \partial_k g_{ij} \dot{x}^k \dot{x}^i \dot{x}^j = 0
\]

\textbf{Status}: PROVEN

\subsubsection{Killing Vector Conservation}

If the metric admits a Killing vector \(\xi^i\) (satisfying \(\nabla_{(i} \xi_{j)} = 0\)), then:

\[
p_\xi = g_{ij} \xi^i \frac{dx^j}{d\lambda}
\]

is conserved along geodesics.

\subsubsection{Topological Charges}

Certain topological invariants of \(\Kseven\) remain constant along flow:
\begin{itemize}
    \item Winding numbers in periodic directions
    \item Holonomy charges around non-contractible loops
    \item Cohomology class representatives
\end{itemize}

\subsection{Solution Methods}

\subsubsection{Perturbative Expansion}

For small torsion \(|\mathbf{T}| \ll 1\), expand geodesics perturbatively:

\[
x^k(\lambda) = x^k_0(\lambda) + \epsilon \, x^k_1(\lambda) + \epsilon^2 \, x^k_2(\lambda) + \cdots
\]

where \(\epsilon \sim |\mathbf{T}| \approx 0.0164\).

\textbf{Zeroth order}: Straight lines (no torsion)

\[
x^k_0(\lambda) = a^k + b^k \lambda
\]

\textbf{First order}: Linear correction from torsion

\[
\ddot{x}^k_1 = \frac{1}{2} g^{kl} T_{ijl} b^i b^j
\]

integrates to:

\[
x^k_1(\lambda) = \frac{1}{4} g^{kl} T_{ijl} b^i b^j \lambda^2
\]

\subsubsection{Numerical Integration}

For non-perturbative solutions, use standard ODE integrators:

\textbf{Initial conditions}:
\begin{itemize}
    \item \(x^k(0) = x^k_{\text{initial}}\) (starting coupling values)
    \item \(\dot{x}^k(0) = v^k_{\text{initial}}\) (initial β-functions)
\end{itemize}

\textbf{Algorithm}: Runge-Kutta 4th order or adaptive step methods

\textbf{Code}: Available at github.com/gift-framework/GIFT

\subsubsection{Fixed Point Analysis}

Fixed points satisfy \(\dot{x}^k = 0\) and \(\ddot{x}^k = 0\):

\[
g^{kl} T_{ijl} v^i v^j = 0 \quad \text{for all } k
\]

\textbf{Types}:
\begin{itemize}
    \item \textbf{Stable (attractor)}: Negative eigenvalues of linearized flow
    \item \textbf{Unstable (repeller)}: Positive eigenvalues
    \item \textbf{Saddle}: Mixed eigenvalues
\end{itemize}

\subsubsection{Geodesic Deviation}

Nearby geodesics separate according to:

\[
\frac{D^2 \xi^k}{d\lambda^2} = R^k{}_{ijl} \dot{x}^i \xi^j \dot{x}^l + \text{(torsion terms)}
\]

where \(\xi^k\) is the separation vector. This determines stability of flow.

\section{RG Flow Connection}

\subsection{Identification \texorpdfstring{\(\lambda = \ln(\mu)\)}{λ = ln(μ)}}

\subsubsection{Physical Motivation}

The renormalization group describes how physical quantities change with energy scale \(\mu\). The identification:

\[
\lambda = \ln\left(\frac{\mu}{\mu_0}\right)
\]

connects geodesic flow to RG evolution.

\textbf{Justifications}:
\begin{enumerate}
    \item Both are one-parameter flows on coupling space
    \item Both exhibit nonlinear dynamics
    \item Dimensional analysis: \(\ln(\mu)\) is dimensionless
    \item Fixed points correspond in both frameworks
\end{enumerate}

\subsubsection{Scale Dependence}

Under this identification:

\begin{table}[H]
\centering
\begin{tabular}{lll}
\toprule
\(\lambda\) \textbf{range} & \textbf{Energy scale} & \textbf{Physics} \\
\midrule
\(\lambda \to +\infty\) & \(\mu \to \infty\) (UV) & \(\E_8\times\E_8\) symmetry \\
\(\lambda = 0\) & \(\mu = \mu_0\) (reference) & Electroweak scale \\
\(\lambda \to -\infty\) & \(\mu \to 0\) (IR) & Confinement \\
\bottomrule
\end{tabular}
\caption{Scale identification}
\end{table}

\subsubsection{Reference Scale}

Natural choice: \(\mu_0 = M_Z = 91.188\) GeV (Z boson mass)

Alternative choices:
\begin{itemize}
    \item \(\mu_0 = v_{\text{EW}} = 246.22\) GeV (Higgs VEV)
    \item \(\mu_0 = M_{\text{Planck}} = 1.22 \times 10^{19}\) GeV (Planck scale)
\end{itemize}

\subsection{Coupling Evolution}

\subsubsection{β-Functions as Velocities}

The RG β-function for coupling \(g_i\):

\[
\beta_i(g) = \frac{dg_i}{d\ln\mu}
\]

becomes under \(\lambda = \ln(\mu)\):

\[
\beta_i = \frac{dx^i}{d\lambda}
\]

\textbf{Interpretation}: β-functions are geodesic velocities on \(\Kseven\).

\subsubsection{β-Function Evolution}

The geodesic equation gives:

\[
\frac{d\beta^k}{d\lambda} = \frac{d^2 x^k}{d\lambda^2} = \frac{1}{2} g^{kl} T_{ijl} \beta^i \beta^j
\]

\textbf{Physical meaning}: The evolution of β-functions (two-loop and higher) is determined by torsion.

\subsubsection{Standard QFT β-Functions}

In perturbative QFT:

\[
\beta(g) = \beta_0 g^3 + \beta_1 g^5 + \beta_2 g^7 + \cdots
\]

\textbf{GIFT interpretation}: The coefficients \(\beta_0, \beta_1, \beta_2\) arise from torsion tensor components:

\[
\beta_n \sim g^{nm} T_{ijm} \times \text{(combinatorial factors)}
\]

\subsubsection{Gauge Coupling Evolution}

For the strong coupling \(\alpha_s(\mu)\):

\[
\frac{d\alpha_s}{d\ln\mu} = -\frac{b_0}{2\pi} \alpha_s^2 - \frac{b_1}{(2\pi)^2} \alpha_s^3 + \cdots
\]

with \(b_0 = 11 - 2n_f/3\) for \(\SU(3)\) QCD.

\textbf{Geometric origin}: \(b_0\) relates to torsion components in the strong sector of \(\Kseven\).

\subsection{Fixed Points}

\subsubsection{UV Fixed Point}

At high energies (\(\lambda \to +\infty\)), the theory approaches the \(\E_8\times\E_8\) symmetric point:

\begin{itemize}
    \item All couplings unified
    \item Maximum symmetry
    \item ``Free'' theory in some sense
\end{itemize}

\textbf{Geometric picture}: The geodesic approaches the symmetric point on \(\Kseven\).

\subsubsection{IR Fixed Point}

At low energies (\(\lambda \to -\infty\)):

\begin{itemize}
    \item Symmetry broken to Standard Model
    \item Couplings reach observed values
    \item Confinement in QCD sector
\end{itemize}

\textbf{Geometric picture}: The geodesic reaches the physical vacuum.

\subsubsection{Intermediate Fixed Points}

Possible fixed points at intermediate scales:

\begin{itemize}
    \item \textbf{GUT scale} (\(\sim 10^{16}\) GeV): Gauge coupling unification
    \item \textbf{Electroweak scale} (\(\sim 10^2\) GeV): Symmetry breaking
    \item \textbf{QCD scale} (\(\sim 10^{-1}\) GeV): Confinement
\end{itemize}

\subsubsection{Fixed Point Stability}

Linearizing the geodesic equation around fixed point \(x^*\):

\[
\ddot{\xi}^k = M^k{}_j \xi^j
\]

where \(\xi^k = x^k - x^{*k}\) and \(M\) is the stability matrix.

\textbf{Classification}:
\begin{itemize}
    \item All eigenvalues negative: Stable (attractor)
    \item All eigenvalues positive: Unstable (UV repeller)
    \item Mixed signs: Saddle point
\end{itemize}

\subsection{Flow Velocity}

\subsubsection{Ultra-Slow Velocity Requirement}

Experimental bounds on constant variation:

\[
\left|\frac{\dot{\alpha}}{\alpha}\right| < 10^{-17} \text{ yr}^{-1}
\]

constrain the \(\Kseven\) flow velocity.

\subsubsection{Velocity Bound Derivation}

The variation rate:

\[
\frac{\dot{\alpha}}{\alpha} \sim H_0 \times |\Gamma| \times |v|^2
\]

where:
\begin{itemize}
    \item \(H_0 \approx 70\) km/s/Mpc \(\approx 2.3 \times 10^{-18}\) s\(^{-1}\)
    \item \(|\Gamma| \sim |\mathbf{T}|/\Det(g) \approx 0.0164/2 \approx 0.008\)
    \item \(|v|\) = flow velocity
\end{itemize}

\textbf{Constraint}:

\[
|v|^2 < \frac{10^{-17}}{H_0 \times |\Gamma|} \approx \frac{10^{-17}}{2.3 \times 10^{-18} \times 0.008} \approx 0.5
\]

\[
|v| < 0.7
\]

\subsubsection{Framework Value}

From numerical simulations and RG flow matching:

\[
|v| \approx 0.015
\]

This ultra-slow velocity ensures:

\[
\frac{\dot{\alpha}}{\alpha} \sim 2.3 \times 10^{-18} \times 0.008 \times (0.015)^2 \approx 4 \times 10^{-24} \text{ s}^{-1} \approx 10^{-16} \text{ yr}^{-1}
\]

Well within experimental bounds.

\subsubsection{Cosmological Consistency}

The slow velocity \(|v| \approx 0.015 \ll 1\) ensures:
\begin{enumerate}
    \item Constants appear approximately fixed at laboratory scales
    \item Evolution occurs over cosmological time
    \item No conflict with precision measurements
    \item Consistency with Big Bang nucleosynthesis bounds
\end{enumerate}

\textbf{Status}: PHENOMENOLOGICAL (constrained by experiment)

\section{Physical Applications}

\subsection{Mass Hierarchies}

\subsubsection{Tau-Electron Ratio}

The mass ratio \(m_\tau/m_e = 3477\) (proven in Supplement S4) has geometric origin in the geodesic length in the \((e,\varphi)\) plane.

\textbf{Geodesic equation in \((e,\varphi)\) sector}:

\[
\frac{d^2 e}{d\lambda^2} = g^{\pi\pi} T_{e\varphi,\pi} \frac{de}{d\lambda} \frac{d\varphi}{d\lambda}
\]

\textbf{Numerical values}:
\begin{itemize}
    \item \(g^{\pi\pi} \approx 2/3\)
    \item \(T_{e\varphi,\pi} \approx -4.89\)
\end{itemize}

The large torsion component \(T_{e\varphi,\pi}\) amplifies the path length, generating the hierarchy.

\subsubsection{Connection to Topology}

The topological formula:

\[
\frac{m_\tau}{m_e} = \dimE(\Kseven) + 10 \times \dimE(\E_8) + 10 \times H^* = 7 + 2480 + 990 = 3477
\]

encodes the accumulated ``information content'' along the geodesic path.

\subsection{CP Violation}

\subsubsection{Geometric Phase}

The CP violation phase \(\delta_{\mathrm{CP}} = 197^\circ\) (proven in Supplement S4) arises from torsional twist in the \((\pi,\varphi)\) sector.

\textbf{Twist equation}:

\[
\frac{d^2 \varphi}{d\lambda^2} \propto T_{\pi\varphi,e} \frac{d\pi}{d\lambda} \frac{de}{d\lambda}
\]

The accumulated twist over one ``cycle'' gives the CP phase.

\subsubsection{Topological Origin}

\[
\delta_{\mathrm{CP}} = 7 \times \dimE(\Gtwo) + H^* = 7 \times 14 + 99 = 197^\circ
\]

The torsion component \(T_{\pi\varphi,e} \approx -0.45\) drives this geometric phase accumulation.

\subsection{Hubble Constant}

\subsubsection{Curvature-Torsion Relation}

The Hubble constant emerges from:

\[
H_0^2 \propto R \cdot |\mathbf{T}|^2
\]

where:
\begin{itemize}
    \item \(R \approx 1/54\): Effective scalar curvature
    \item \(|\mathbf{T}| \approx 0.0164\): Torsion magnitude
\end{itemize}

\subsubsection{Intermediate Value}

The framework predicts:

\[
H_0 \approx 69.8 \text{ km/s/Mpc}
\]

This intermediate value between CMB (67.4) and local (73.0) measurements suggests potential geometric resolution of the Hubble tension.

\section{Summary}

This supplement established the torsional geodesic dynamics of the GIFT framework:

\subsection*{Key Results}

\begin{table}[H]
\centering
\begin{tabular}{lll}
\toprule
\textbf{Result} & \textbf{Value} & \textbf{Status} \\
\midrule
Torsion magnitude & \(|\mathbf{T}| \approx 0.0164\) & THEORETICAL \\
\(T_{e\varphi,\pi}\) & \(-4.89\) & THEORETICAL \\
\(T_{\pi\varphi,e}\) & \(-0.45\) & THEORETICAL \\
\(T_{e\pi,\varphi}\) & \(\sim 3\times 10^{-5}\) & THEORETICAL \\
Flow velocity & \(|v| \approx 0.015\) & PHENOMENOLOGICAL \\
\(\dot{\alpha}/\alpha\) bound & \(< 10^{-16}\) yr\(^{-1}\) & PHENOMENOLOGICAL \\
\bottomrule
\end{tabular}
\caption{Key dynamical results}
\end{table}

\subsection*{Main Equations}

\textbf{Torsional connection}:
\[
\Gamma^k_{ij} = -\frac{1}{2} g^{kl} T_{ijl}
\]

\textbf{Geodesic equation}:
\[
\frac{d^2 x^k}{d\lambda^2} = \frac{1}{2} g^{kl} T_{ijl} \frac{dx^i}{d\lambda} \frac{dx^j}{d\lambda}
\]

\textbf{RG identification}:
\[
\lambda = \ln(\mu/\mu_0), \quad \beta^i = \frac{dx^i}{d\lambda}
\]

\subsection*{Physical Interpretation}

The framework provides geometric foundations for:
\begin{itemize}
    \item Mass hierarchies from geodesic lengths
    \item CP violation from torsional twist
    \item RG flow from geodesic evolution
    \item Constant stability from ultra-slow velocity
\end{itemize}

\noindent\hrulefill

\vfill

\begin{thebibliography}{99}

\bibitem{cartan1923}
Cartan, E. (1923). Sur les variétés à connexion affine et la théorie de la relativité généralisée. \textit{Ann. Sci. ENS}, \textbf{40}, 325.

\bibitem{kibble1961}
Kibble, T.W.B. (1961). Lorentz invariance and the gravitational field. \textit{J. Math. Phys.}, \textbf{2}, 212.

\bibitem{hehl1976}
Hehl, F.W., et al. (1976). General relativity with spin and torsion. \textit{Rev. Mod. Phys.}, \textbf{48}, 393.

\bibitem{joyce2000}
Joyce, D.D. (2000). \textit{Compact Manifolds with Special Holonomy}. Oxford University Press.

\bibitem{karigiannis2009}
Karigiannis, S. (2009). Flows of G2-structures. \textit{Q. J. Math.}, \textbf{60}, 487.

\bibitem{grigorian2013}
Grigorian, S. (2013). Short-time behaviour of a modified Laplacian coflow of G2-structures. \textit{Adv. Math.}, \textbf{248}, 378.

\bibitem{gift_2025}
de la Fournière, B. (2025). \textit{Geometric Information Field Theory}. Zenodo. \url{https://doi.org/10.5281/zenodo.17434034}

\end{thebibliography}

\vfill

\noindent\hrulefill

\vspace{0.5em}

\noindent\textit{GIFT Framework v2.1 - Supplement S3}

\noindent\textit{Torsional Dynamics}


\end{document}
