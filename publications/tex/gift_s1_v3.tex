\documentclass[11pt,a4paper]{article}

% ============================================
% ENCODING & FONTS
% ============================================
\usepackage[utf8]{inputenc}
\usepackage[T1]{fontenc}
\usepackage{lmodern}

% ============================================
% PAGE LAYOUT
% ============================================
\usepackage[margin=1.618cm, top=2.618cm, bottom=2.618cm]{geometry}

% ============================================
% ESSENTIAL PACKAGES
% ============================================
\usepackage{float}
\usepackage{caption}
\usepackage{setspace}
\usepackage{fancyhdr}
\usepackage{xcolor}
\usepackage{hyperref}
\usepackage{amsmath}
\usepackage{amssymb}
\usepackage{booktabs}
\usepackage{longtable}
\usepackage{array}
\usepackage{listings}
\usepackage{graphicx}
\DeclareUnicodeCharacter{00B0}{\ensuremath{^\circ}}

% ============================================
% LISTINGS CONFIGURATION
% ============================================
\lstset{
    basicstyle=\small\ttfamily,
    breaklines=true,
    frame=single,
    keepspaces=true,
    showstringspaces=false,
    breakatwhitespace=true,
    aboveskip=0.8em,
    belowskip=0.8em
}

% ============================================
% TITLE FORMATTING
% ============================================
\usepackage{titling}
\pretitle{\LARGE\bfseries}
\posttitle{\vspace{-0.4em}}
\preauthor{}
\postauthor{}
\predate{}
\postdate{}
\setlength{\droptitle}{-2.0em}

% ============================================
% HEADER/FOOTER
% ============================================
\setlength{\headheight}{14pt}
\pagestyle{fancy}
\fancyhf{}
\fancyhead[L]{GIFT Framework v3.0 --- Supplement S1}
\fancyhead[R]{\thepage}
\renewcommand{\headrulewidth}{0.2pt}

% ============================================
% HYPERREF
% ============================================
\hypersetup{
    colorlinks=true,
    linkcolor=blue,
    citecolor=blue,
    urlcolor=blue,
    pdftitle={GIFT Supplement S1: Mathematical Foundations},
    pdfauthor={Brieuc de La Fourniere}
}

% ============================================
% SPACING
% ============================================
\setstretch{1.2}
\setlength{\parskip}{0.4em}
\setlength{\parindent}{0pt}

% ============================================
% CUSTOM COMMANDS
% ============================================
\newcommand{\E}{\mathrm{E}}
\newcommand{\Gtwo}{\mathrm{G}_2}
\newcommand{\Kseven}{K_7}
\newcommand{\dimE}{\mathrm{dim}}
\newcommand{\Weyl}{\mathrm{Weyl}}
\newcommand{\rk}{\mathrm{rank}}
\newcommand{\proven}{\textsc{Proven}}
\newcommand{\topomark}{\textsc{Topological}}
\newcommand{\derived}{\textsc{Derived}}
\newcommand{\AdS}{\mathrm{AdS}}
\newcommand{\CP}{\mathrm{CP}}
\newcommand{\GIFT}{\textrm{GIFT}}
\newcommand{\SU}{\mathrm{SU}}
\newcommand{\U}{\mathrm{U}}

\pdfstringdefDisableCommands{%
  \def\Gtwo{G2}%
  \def\Kseven{K7}%
  \def\E{E}%
  \def\dimE{dim}%
  \def\Weyl{Weyl}%
  \def\rk{rank}%
  \def\proven{Proven}%
  \def\topomark{Topological}%
  \def\derived{Derived}%
  \def\AdS{AdS}%
  \def\CP{CP}%
  \def\GIFT{GIFT}%
  \def\SU{SU}%
  \def\U{U}%
}

% ============================================
% TITLE
% ============================================
\title{%
\LARGE\textbf{Supplement S1: Mathematical Foundations}\\[0.5em]
\large $\E_8$ Exceptional Lie Algebra, $\Gtwo$ Holonomy Manifolds,\\
and $\Kseven$ Construction
}
\author{}
\date{}

\begin{document}
\maketitle
\noindent\rule{\textwidth}{0.2pt}

\noindent\textbf{Brieuc de La Fournière}\\
Independent researcher

\vfill

\begin{abstract}
We present the mathematical architecture underlying GIFT v3.0. Part I develops $\E_8$ exceptional Lie algebra with the Exceptional Chain theorem. Part II introduces $\Gtwo$ holonomy manifolds. Part III establishes $\Kseven$ manifold construction via twisted connected sum. Part IV presents the metric structure with formal verification. These structures provide rigorous basis for the $\E_8 \times \E_8 \to \Kseven \to$ Standard Model reduction.
\end{abstract}
\vfill
\noindent\rule{\textwidth}{0.2pt}
\newpage
\tableofcontents

\newpage

% ============================================
\section*{Part I: $\E_8$ Exceptional Lie Algebra}
\addcontentsline{toc}{section}{Part I: $\E_8$ Exceptional Lie Algebra}
% ============================================

\section{Root System and Dynkin Diagram}

\subsection{Basic Data}

\begin{table}[H]
\centering
\begin{tabular}{lcc}
\toprule
Property & Value & GIFT Role \\
\midrule
Dimension & $\dimE(\E_8) = 248$ & Gauge DOF \\
Rank & $\rk(\E_8) = 8$ & Cartan subalgebra \\
Number of roots & $|\Phi(\E_8)| = 240$ & $\E_8$ kissing number \\
Root length & $\sqrt{2}$ & $\alpha_s$ numerator \\
Coxeter number & $h = 30$ & Icosahedron edges \\
Dual Coxeter number & $h^\vee = 30$ & McKay correspondence \\
\bottomrule
\end{tabular}
\end{table}

\subsection{Root System Construction}

$\E_8$ root system in $\mathbb{R}^8$ has 240 roots:

\textbf{Type I (112 roots)}: Permutations and sign changes of $(\pm 1, \pm 1, 0, 0, 0, 0, 0, 0)$

\textbf{Type II (128 roots)}: Half-integer coordinates with even minus signs:
\[
\frac{1}{2}(\pm 1, \pm 1, \pm 1, \pm 1, \pm 1, \pm 1, \pm 1, \pm 1)
\]

\textbf{Verification}: $112 + 128 = 240$ roots, all length $\sqrt{2}$.

\subsection{Cartan Matrix}

\[
A_{\E_8} = \begin{pmatrix}
2 & 0 & -1 & 0 & 0 & 0 & 0 & 0 \\
0 & 2 & 0 & -1 & 0 & 0 & 0 & 0 \\
-1 & 0 & 2 & -1 & 0 & 0 & 0 & 0 \\
0 & -1 & -1 & 2 & -1 & 0 & 0 & 0 \\
0 & 0 & 0 & -1 & 2 & -1 & 0 & 0 \\
0 & 0 & 0 & 0 & -1 & 2 & -1 & 0 \\
0 & 0 & 0 & 0 & 0 & -1 & 2 & -1 \\
0 & 0 & 0 & 0 & 0 & 0 & -1 & 2
\end{pmatrix}
\]

\textbf{Properties}: $\det(A) = 1$ (unimodular), positive definite.


% ============================================
\section{Weyl Group}
% ============================================

\subsection{Order and Factorization}

\[
|W(\E_8)| = 696,729,600 = 2^{14} \times 3^5 \times 5^2 \times 7
\]

\subsection{Topological Factorization Theorem}

\textbf{Theorem}: The Weyl group order factorizes entirely into GIFT constants:

\[
|W(\E_8)| = p_2^{\dimE(\Gtwo)} \times N_{\mathrm{gen}}^{\Weyl} \times \Weyl^{p_2} \times \dimE(\Kseven)
\]

\begin{table}[H]
\centering
\begin{tabular}{lccc}
\toprule
Factor & Exponent & Value & GIFT Origin \\
\midrule
$2^{14}$ & $\dimE(\Gtwo) = 14$ & 16384 & $p_2^{\text{holonomy dim}}$ \\
$3^5$ & $\Weyl = 5$ & 243 & $N_{\mathrm{gen}}^{\Weyl \text{ factor}}$ \\
$5^2$ & $p_2 = 2$ & 25 & $\Weyl^{\text{binary}}$ \\
$7^1$ & 1 & 7 & $\dimE(\Kseven)$ \\
\bottomrule
\end{tabular}
\end{table}

\textbf{Status}: \textbf{\proven\ (Lean)}: \texttt{weyl\_E8\_topological\_factorization}

% ============================================
\section{Exceptional Chain}
% ============================================

\subsection{The Pattern}

A remarkable pattern connects exceptional algebra dimensions to primes:

\begin{table}[H]
\centering
\begin{tabular}{ccccc}
\toprule
Algebra & $n$ & $\dimE(\E_n)$ & Prime & Index \\
\midrule
$\E_6$ & 6 & 78 & 13 & prime(6) \\
$\E_7$ & 7 & 133 & 19 & prime(8) = prime($\rk(\E_8)$) \\
$\E_8$ & 8 & 248 & 31 & prime(11) = prime($D_{\text{bulk}}$) \\
\bottomrule
\end{tabular}
\end{table}

\subsection{Exceptional Chain Theorem}

\textbf{Theorem}: For $n \in \{6, 7, 8\}$:
\[
\dimE(\E_n) = n \times \mathrm{prime}(g(n))
\]

where $g(6) = 6$, $g(7) = \rk(\E_8) = 8$, $g(8) = D_{\text{bulk}} = 11$.

\textbf{Proof} (verified in Lean):
\begin{itemize}
\item $\E_6$: $6 \times 13 = 78$ \checkmark
\item $\E_7$: $7 \times 19 = 133$ \checkmark
\item $\E_8$: $8 \times 31 = 248$ \checkmark
\end{itemize}

\textbf{Status}: \textbf{\proven\ (Lean)}: \texttt{exceptional\_chain\_certified}


% ============================================
\section{$\E_8 \times \E_8$ Product Structure}
% ============================================

\subsection{Direct Sum}

\begin{table}[H]
\centering
\begin{tabular}{lc}
\toprule
Property & Value \\
\midrule
Dimension & $496 = 248 \times 2$ \\
Rank & $16 = 8 \times 2$ \\
Roots & $480 = 240 \times 2$ \\
\bottomrule
\end{tabular}
\end{table}

\subsection{$\tau$ Numerator Connection}

The hierarchy parameter numerator:
\[
\tau_{\mathrm{num}} = 3472 = 7 \times 496 = \dimE(\Kseven) \times \dimE(\E_8 \times \E_8)
\]

\textbf{Status}: \textbf{\proven\ (Lean)}: \texttt{tau\_num\_E8xE8}

\subsection{Binary Duality Parameter}

\textbf{Triple geometric origin of $p_2 = 2$}:

\begin{enumerate}
\item \textbf{Local}: $p_2 = \dimE(\Gtwo)/\dimE(\Kseven) = 14/7 = 2$
\item \textbf{Global}: $p_2 = \dimE(\E_8 \times \E_8)/\dimE(\E_8) = 496/248 = 2$
\item \textbf{Root}: $\sqrt{2}$ in $\E_8$ root normalization
\end{enumerate}

% ============================================
\section{Octonionic Structure}
% ============================================

\subsection{Exceptional Jordan Algebra $J_3(\mathbb{O})$}

\begin{table}[H]
\centering
\begin{tabular}{lc}
\toprule
Property & Value \\
\midrule
$\dimE(J_3(\mathbb{O}))$ & $27 = 3^3$ \\
$\dimE(J_3(\mathbb{O})_0)$ & 26 (traceless) \\
\bottomrule
\end{tabular}
\end{table}

\subsection{$F_4$ Connection}

$F_4$ is the automorphism group of $J_3(\mathbb{O})$:
\[
\dimE(F_4) = 52 = p_2^2 \times \alpha_{\mathrm{sum}}^B = 4 \times 13
\]

\subsection{Exceptional Differences}

\begin{table}[H]
\centering
\begin{tabular}{lcc}
\toprule
Difference & Value & GIFT \\
\midrule
$\dimE(\E_8) - \dimE(J_3(\mathbb{O}))$ & $221 = 13 \times 17$ & $\alpha_B \times \lambda_{H,\text{num}}$ \\
$\dimE(F_4) - \dimE(J_3(\mathbb{O}))$ & $25 = 5^2$ & $\Weyl^2$ \\
$\dimE(\E_6) - \dimE(F_4)$ & 26 & $\dimE(J_3(\mathbb{O})_0)$ \\
\bottomrule
\end{tabular}
\end{table}

\textbf{Status}: \textbf{\proven\ (Lean)}: \texttt{exceptional\_differences\_certified}

% ============================================
\section*{Part II: $\Gtwo$ Holonomy Manifolds}
\addcontentsline{toc}{section}{Part II: $\Gtwo$ Holonomy Manifolds}
% ============================================

\section{Definition and Properties}

\subsection{$\Gtwo$ as Exceptional Holonomy}

\begin{table}[H]
\centering
\begin{tabular}{lcc}
\toprule
Property & Value & GIFT Role \\
\midrule
$\dimE(\Gtwo)$ & 14 & $Q_{\mathrm{Koide}}$ numerator \\
$\rk(\Gtwo)$ & 2 & Lie rank \\
Definition & $\mathrm{Aut}(\mathbb{O})$ & Octonion automorphisms \\
\bottomrule
\end{tabular}
\end{table}

\subsection{Holonomy Classification (Berger)}

\begin{table}[H]
\centering
\begin{tabular}{lcc}
\toprule
Dimension & Holonomy & Geometry \\
\midrule
\textbf{7} & \textbf{$\Gtwo$} & \textbf{Exceptional} \\
8 & $\mathrm{Spin}(7)$ & Exceptional \\
\bottomrule
\end{tabular}
\end{table}

\subsection{Torsion Conditions}

\textbf{Torsion-free}: $\nabla\varphi = 0 \Leftrightarrow d\varphi = 0$, $d{*}\varphi = 0$

\textbf{Controlled non-closure} (GIFT):
\[
|d\varphi|^2 + |d{*}\varphi|^2 = \kappa_T^2 = \frac{1}{61^2}
\]

% ============================================
\section{Topological Invariants}
% ============================================

\subsection{Derived Constants}

\begin{table}[H]
\centering
\begin{tabular}{lcc}
\toprule
Constant & Formula & Value \\
\midrule
$\det(g)$ & $p_2 + 1/(b_2 + \dimE(\Gtwo) - N_{\mathrm{gen}})$ & $65/32$ \\
$\kappa_T$ & $1/(b_3 - \dimE(\Gtwo) - p_2)$ & $1/61$ \\
$\sin^2\theta_W$ & $b_2/(b_3 + \dimE(\Gtwo))$ & $3/13$ \\
\bottomrule
\end{tabular}
\end{table}

\subsection{The 61 Decomposition}

\[
\kappa_T^{-1} = 61 = \dimE(F_4) + N_{\mathrm{gen}}^2 = 52 + 9
\]

Alternative:
\[
61 = \Pi(\alpha_B^2) + 1 = 2 \times 5 \times 6 + 1
\]

\textbf{Status}: \textbf{\proven\ (Lean)}: \texttt{kappa\_T\_inv\_decomposition}

% ============================================
\section*{Part III: $\Kseven$ Manifold Construction}
\addcontentsline{toc}{section}{Part III: $\Kseven$ Manifold Construction}
% ============================================

\section{Twisted Connected Sum Framework}

\subsection{TCS Construction}

The twisted connected sum (TCS) construction provides the primary method for constructing compact $\Gtwo$ manifolds from asymptotically cylindrical building blocks.

\textbf{Key insight}: $\Gtwo$ manifolds can be built by gluing two asymptotically cylindrical (ACyl) $\Gtwo$ manifolds along their cylindrical ends, with the topology controlled by a twist diffeomorphism $\phi$.

\subsection{Asymptotically Cylindrical $\Gtwo$ Manifolds}

\textbf{Definition}: A complete Riemannian 7-manifold $(M, g)$ with $\Gtwo$ holonomy is asymptotically cylindrical (ACyl) if there exists a compact subset $K \subset M$ such that $M \setminus K$ is diffeomorphic to $(T_0, \infty) \times N$ for some compact 6-manifold $N$.

\subsection{Building Blocks}

For the GIFT framework, $\Kseven$ is constructed from two ACyl $\Gtwo$ manifolds:

\textbf{Region $M_1^T$} (asymptotic to $S^1 \times Y_3^{(1)}$):
\begin{itemize}
\item Betti numbers: $b_2(M_1) = 11$, $b_3(M_1) = 40$
\item Calabi-Yau: $Y_3^{(1)}$ with $h^{1,1}(Y_3^{(1)}) = 11$
\end{itemize}

\textbf{Region $M_2^T$} (asymptotic to $S^1 \times Y_3^{(2)}$):
\begin{itemize}
\item Betti numbers: $b_2(M_2) = 10$, $b_3(M_2) = 37$
\item Calabi-Yau: $Y_3^{(2)}$ with $h^{1,1}(Y_3^{(2)}) = 10$
\end{itemize}

\textbf{The compact manifold}:
\[
\Kseven = M_1^T \cup_\phi M_2^T
\]

\textbf{Global properties}:
\begin{itemize}
\item Compact 7-manifold (no boundary)
\item $\Gtwo$ holonomy preserved by construction
\item Ricci-flat: $\mathrm{Ric}(g) = 0$
\item Euler characteristic: $\chi(\Kseven) = 0$
\end{itemize}

\textbf{Status}: \topomark

% ============================================
\section{Cohomological Structure}
% ============================================

\subsection{Mayer-Vietoris Analysis}

The Mayer-Vietoris sequence provides the primary tool for computing cohomology:

\[
\cdots \to H^{k-1}(N) \xrightarrow{\delta} H^k(\Kseven) \xrightarrow{i^*} H^k(M_1) \oplus H^k(M_2) \xrightarrow{j^*} H^k(N) \to \cdots
\]

\subsection{Betti Number Derivation}

\textbf{Result for $b_2$}: The sequence analysis yields:
\[
b_2(\Kseven) = b_2(M_1) + b_2(M_2) = 11 + 10 = 21
\]

\textbf{Result for $b_3$}: Similarly:
\[
b_3(\Kseven) = b_3(M_1) + b_3(M_2) = 40 + 37 = 77
\]

\textbf{Status}: \topomark\ (exact)

\subsection{Complete Betti Spectrum}

\begin{table}[H]
\centering
\begin{tabular}{ccc}
\toprule
$k$ & $b_k(\Kseven)$ & Derivation \\
\midrule
0 & 1 & Connected \\
1 & 0 & Simply connected ($\Gtwo$ holonomy) \\
2 & 21 & Mayer-Vietoris \\
3 & 77 & Mayer-Vietoris \\
4 & 77 & Poincar\'e duality \\
5 & 21 & Poincar\'e duality \\
6 & 0 & Poincar\'e duality \\
7 & 1 & Poincar\'e duality \\
\bottomrule
\end{tabular}
\end{table}

\textbf{Euler characteristic verification}:
\[
\chi(\Kseven) = 1 - 0 + 21 - 77 + 77 - 21 + 0 - 1 = 0
\]

\textbf{Effective cohomological dimension}:
\[
H^* = b_2 + b_3 + 1 = 21 + 77 + 1 = 99
\]

\subsection{Third Betti Number Decomposition}

The $b_3 = 77$ harmonic 3-forms decompose as:

\[
H^3(\Kseven) = H^3_{\text{local}} \oplus H^3_{\text{global}}
\]

\begin{table}[H]
\centering
\begin{tabular}{lcc}
\toprule
Component & Dimension & Origin \\
\midrule
$H^3_{\text{local}}$ & $35 = C(7,3)$ & $\Lambda^3(\mathbb{R}^7)$ fiber forms \\
$H^3_{\text{global}}$ & $42 = 2 \times 21$ & TCS global modes \\
\bottomrule
\end{tabular}
\end{table}

\textbf{Verification}: $35 + 42 = 77$

\textbf{Status}: \topomark


% ============================================
\section*{Part IV: Metric Structure and Verification}
\addcontentsline{toc}{section}{Part IV: Metric Structure and Verification}
% ============================================

\section{Structural Metric Invariants}

\subsection{The Zero-Parameter Paradigm}

The GIFT framework proposes that all metric invariants derive from fixed mathematical structure. The constraints are \textbf{inputs}; the specific geometry is \textbf{emergent}.

\begin{table}[H]
\centering
\begin{tabular}{lccc}
\toprule
Invariant & Formula & Value & Status \\
\midrule
$\kappa_T$ & $1/(b_3 - \dimE(\Gtwo) - p_2)$ & $1/61$ & \topomark \\
$\det(g)$ & $(\Weyl \times (\rk(\E_8) + \Weyl))/2^5$ & $65/32$ & \topomark \\
\bottomrule
\end{tabular}
\end{table}

\subsection{Torsion Magnitude $\kappa_T = 1/61$}

\textbf{Derivation}:
\[
\kappa_T = \frac{1}{b_3 - \dimE(\Gtwo) - p_2} = \frac{1}{77 - 14 - 2} = \frac{1}{61}
\]

\textbf{Interpretation}:
\begin{itemize}
\item $61 =$ effective matter degrees of freedom
\item $b_3 = 77$ total fermion modes
\item $\dimE(\Gtwo) = 14$ gauge symmetry constraints
\item $p_2 = 2$ binary duality factor
\end{itemize}

\textbf{Status}: \topomark

\subsection{Metric Determinant $\det(g) = 65/32$}

\textbf{Topological formula} (exact target):
\[
\det(g) = \frac{\Weyl \times (\rk(\E_8) + \Weyl)}{2^{\Weyl}} = \frac{5 \times 13}{32} = \frac{65}{32}
\]

\textbf{Alternative derivations} (all equivalent):
\begin{itemize}
\item $\det(g) = p_2 + 1/(b_2 + \dimE(\Gtwo) - N_{\mathrm{gen}}) = 2 + 1/32 = 65/32$
\item $\det(g) = (H^* - b_2 - 13)/32 = (99 - 21 - 13)/32 = 65/32$
\end{itemize}

\textbf{Status}: \topomark\ (exact rational value)


% ============================================
\section{Formal Certification}
% ============================================

\subsection{Lean 4 Proof Structure}

A complete Lean 4 formalization of Joyce's Perturbation Theorem for $\Gtwo$ manifolds has been developed.

\begin{table}[H]
\centering
\begin{tabular}{lc}
\toprule
Metric & Value \\
\midrule
\textbf{Lean modules} & 5 core + infrastructure \\
\textbf{Total new lines} & $\sim$1,800 \\
\textbf{New theorems} & $\sim$50 \\
\bottomrule
\end{tabular}
\end{table}

\textbf{Main Result}:
\begin{lstlisting}
theorem k7_admits_torsion_free_g2 :
    exists phi : G2Space, IsTorsionFree phi
\end{lstlisting}

\subsection{Joyce Theorem Application}

\begin{table}[H]
\centering
\begin{tabular}{lccc}
\toprule
Requirement & Threshold & Achieved & Margin \\
\midrule
$\|T(\varphi_0)\| < \varepsilon_0$ & 0.0288 & 0.00140 & $20\times$ \\
$g(\varphi_0)$ positive & Required & $\lambda_{\min} = 1.078$ & Yes \\
$M$ compact & Required & $\Kseven$ compact & Yes \\
\bottomrule
\end{tabular}
\end{table}

\textbf{Conclusion}: By Joyce's theorem, since $\|T(\varphi_{\text{num}})\| < \varepsilon_0$ with $20\times$ margin, there exists an exact torsion-free $\Gtwo$ structure on $\Kseven$.

\textbf{Status}: \textbf{\proven\ (Lean-verified via Banach fixed point)}


% ============================================
\section{Physical Implications}
% ============================================

\subsection{Gauge Structure from $b_2 = 21$}

The 21 harmonic 2-forms correspond to:
\begin{itemize}
\item \textbf{8 gluons}: $\SU(3)$ color force
\item \textbf{3 weak bosons}: $\SU(2)_L$
\item \textbf{1 hypercharge}: $\U(1)_Y$
\item \textbf{9 hidden sector}: Beyond Standard Model
\end{itemize}

\subsection{Fermion Structure from $b_3 = 77$}

The 77 harmonic 3-forms decompose as:
\begin{itemize}
\item \textbf{35 local modes}: $\Lambda^3(\mathbb{R}^7)$ fiber at each point
\item \textbf{42 global modes}: TCS modes ($2 \times 21$)
\end{itemize}

The generation structure $N_{\mathrm{gen}} = 3$ emerges from the topology.
\vfill
\noindent\rule{\textwidth}{0.2pt}
\textit{GIFT Framework v3.0 - Supplement S1}\\
\textit{Mathematical Foundations: $\E_8$ + $\Gtwo$ + $\Kseven$}


\end{document}

