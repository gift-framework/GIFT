\documentclass[11pt,a4paper]{article}

% ============================================
% ENCODING & FONTS
% ============================================
\usepackage[utf8]{inputenc}
\usepackage[T1]{fontenc}
\usepackage{lmodern}

% ============================================
% PAGE LAYOUT
% ============================================
\usepackage[margin=1.618cm, top=2.618cm, bottom=2.618cm]{geometry}

% ============================================
% ESSENTIAL PACKAGES
% ============================================
\usepackage{float}
\usepackage{caption}
\usepackage{setspace}
\usepackage{fancyhdr}
\usepackage{xcolor}
\usepackage{hyperref}
\usepackage{amsmath}
\usepackage{amssymb}
\usepackage{booktabs}
\usepackage{longtable}
\usepackage{array}
\usepackage{listings}
\usepackage{graphicx}
\DeclareUnicodeCharacter{00B0}{\ensuremath{^\circ}}

% ============================================
% LISTINGS CONFIGURATION
% ============================================
\lstset{
    basicstyle=\small\ttfamily,
    breaklines=true,
    frame=single,
    keepspaces=true,
    showstringspaces=false,
    breakatwhitespace=true,
    aboveskip=0.8em,
    belowskip=0.8em
}

% ============================================
% TITLE FORMATTING
% ============================================
\usepackage{titling}
\pretitle{\LARGE\bfseries}
\posttitle{\vspace{-0.4em}}
\preauthor{}
\postauthor{}
\predate{}
\postdate{}
\setlength{\droptitle}{-2.0em}

% ============================================
% HEADER/FOOTER
% ============================================
\setlength{\headheight}{14pt}
\pagestyle{fancy}
\fancyhf{}
\fancyhead[L]{GIFT Framework v3.3 -- Supplement S2}
\fancyhead[R]{\thepage}
\renewcommand{\headrulewidth}{0.2pt}

% ============================================
% HYPERREF
% ============================================
\hypersetup{
    colorlinks=true,
    linkcolor=blue,
    citecolor=blue,
    urlcolor=blue,
    pdftitle={GIFT Supplement S2: Complete Derivations},
    pdfauthor={Brieuc de La Fourniere}
}

% ============================================
% SPACING
% ============================================
\setstretch{1.2}
\setlength{\parskip}{0.4em}
\setlength{\parindent}{0pt}

% ============================================
% CUSTOM COMMANDS
% ============================================
\newcommand{\E}{\mathrm{E}}
\newcommand{\Gtwo}{\mathrm{G}_2}
\newcommand{\Kseven}{K_7}
\newcommand{\dimE}{\mathrm{dim}}
\newcommand{\Weyl}{\mathrm{Weyl}}
\newcommand{\rk}{\mathrm{rank}}
\newcommand{\proven}{\textsc{Verified}}
\newcommand{\topomark}{\textsc{Topological}}
\newcommand{\GIFT}{\textrm{GIFT}}

\pdfstringdefDisableCommands{%
  \def\Gtwo{G2}%
  \def\Kseven{K7}%
  \def\E{E}%
  \def\dimE{dim}%
  \def\Weyl{Weyl}%
  \def\rk{rank}%
  \def\proven{Verified}%
  \def\topomark{Topological}%
  \def\GIFT{GIFT}%
}

\title{%
\LARGE\textbf{Supplement S2: Complete Derivations (Dimensionless)}\\[0.3em]
\Large Complete Mathematical Derivations for All 33 Dimensionless Predictions
}
\author{}
\date{}

\begin{document}

\maketitle
\noindent\rule{\textwidth}{0.2pt}

\noindent\textbf{Version}: 3.3

\noindent\textbf{Author}: Brieuc de La Fourni\`ere

\noindent Independent researcher

\vfill

\noindent{This supplement provides mathematical derivations for all dimensionless predictions in the GIFT framework. Each derivation proceeds from topological definitions to numerical predictions.}

\noindent\textbf{Status}: 18 core relations verified in Lean 4; 15 extended predictions with topological formulas

\noindent\textbf{Note on verification levels}: The main paper references 33 dimensionless predictions. Of these:
\begin{itemize}
\item \textbf{18 core relations} (Parts II--VII): VERIFIED status, algebraic identities machine-checked in Lean 4
\item \textbf{15 extended predictions} (Part IX): TOPOLOGICAL or HEURISTIC status, formulas use topological constants but lack full Lean verification
\end{itemize}

\noindent{The topological constants that determine these relations are described in S1.}

\vfill
\noindent\rule{\textwidth}{0.2pt}

\newpage
\tableofcontents

\newpage

% ============================================
\section*{Part 0: Derivation Philosophy}
\addcontentsline{toc}{section}{Part 0: Derivation Philosophy}
% ============================================

\section{What ``Derivation'' Means in GIFT}

Before presenting derivations, we clarify the logical structure:

\subsection{Inputs vs Outputs}

\textbf{Inputs} (taken as given):
\begin{itemize}
\item The octonion algebra $\mathbb{O}$ and its automorphism group $\Gtwo = \mathrm{Aut}(\mathbb{O})$
\item The $\E_8\times\E_8$ gauge structure
\item The $\Kseven$ manifold (TCS construction with $b_2 = 21$, $b_3 = 77$)
\end{itemize}

\textbf{Outputs} (derived from inputs):
\begin{itemize}
\item The 18 dimensionless predictions
\end{itemize}

\subsection{What We Do NOT Claim}

\begin{itemize}
\item That $\mathbb{O} \to \Gtwo \to \Kseven$ is the unique geometry for physics
\item That the formulas are uniquely determined by geometric principles
\item That the selection rule for specific combinations ($b_2/(b_3 + \dimE(\Gtwo))$ vs $b_2/b_3$) is understood
\end{itemize}

\subsection{What We Observe}

\begin{itemize}
\item Given the inputs, the outputs follow by algebra
\item The outputs match experiment to 0.26\% mean deviation (PDG 2024)
\item No continuous parameters are fitted
\end{itemize}

\subsection{Torsion Independence}

\textbf{Important}: All 18 predictions use only topological invariants. The torsion $T$ does not appear in any formula. Therefore:
\begin{itemize}
\item Predictions depend only on topology, not on the actual torsion value
\item The value $\kappa_T = 1/61$ is a topological bound, not a prediction ingredient
\end{itemize}

% ============================================
\section*{Part I: Foundations}
\addcontentsline{toc}{section}{Part I: Foundations}
% ============================================

\section{Status Classification}

\begin{table}[H]
\centering
\begin{tabular}{ll}
\toprule
Status & Criterion \\
\midrule
\textbf{VERIFIED} & Complete mathematical proof, exact result from topology \\
\textbf{VERIFIED (Lean 4)} & Verified by Lean 4 kernel with Mathlib (machine-checked) \\
\textbf{\topomark{}} & Direct consequence of manifold structure \\
\bottomrule
\end{tabular}
\end{table}

\section{Notation}

\begin{table}[H]
\centering
\small
\begin{tabular}{lll}
\toprule
Symbol & Value & Definition \\
\midrule
$\dimE(\E_8)$ & 248 & $\E_8$ Lie algebra dimension \\
$\rk(\E_8)$ & 8 & $\E_8$ Cartan subalgebra dimension \\
$\dimE(\Gtwo)$ & 14 & $\Gtwo$ holonomy group dimension \\
$\dimE(\Kseven)$ & 7 & Internal manifold dimension \\
$b_2(\Kseven)$ & 21 & Second Betti number \\
$b_3(\Kseven)$ & 77 & Third Betti number \\
$H^*$ & 99 & Cohomological sum $= b_2 + b_3 + 1$ \\
$\dimE(J_3(\mathbb{O}))$ & 27 & Exceptional Jordan algebra dimension \\
$N_{\mathrm{gen}}$ & 3 & Number of fermion generations \\
$p_2$ & 2 & Dimensional ratio: $\dimE(\Gtwo)/\dimE(\Kseven)$ \\
$w$ & 5 & Pentagonal index: $(\dimE(\Gtwo)+1)/N_{\mathrm{gen}} = b_2/N_{\mathrm{gen}} - p_2 = \dimE(\Gtwo) - \rk(\E_8) - 1$ \\
\bottomrule
\end{tabular}
\end{table}

% ============================================
\section*{Part II: Foundational Theorems}
\addcontentsline{toc}{section}{Part II: Foundational Theorems}
% ============================================

\section{Relation \#1: Generation Number $N_{\mathrm{gen}} = 3$}

\textbf{Statement}: The number of fermion generations is exactly 3.

\textbf{Classification}: \proven{} (three independent derivations)

\subsection{Derivation Method 1: Fundamental Topological Constraint}

\textit{Theorem}: For $\Gtwo$ holonomy manifold $\Kseven$ with $\E_8$ gauge structure:

$$(\rk(\E_8) + N_{\mathrm{gen}}) \cdot b_2(\Kseven) = N_{\mathrm{gen}} \cdot b_3(\Kseven)$$

\textit{Derivation}:
$$(8 + N_{\mathrm{gen}}) \times 21 = N_{\mathrm{gen}} \times 77$$
$$168 + 21 \cdot N_{\mathrm{gen}} = 77 \cdot N_{\mathrm{gen}}$$
$$168 = 56 \cdot N_{\mathrm{gen}}$$
$$N_{\mathrm{gen}} = \frac{168}{56} = 3$$

\textit{Verification}:
\begin{itemize}
\item LHS: $(8 + 3) \times 21 = 231$
\item RHS: $3 \times 77 = 231$ \checkmark
\end{itemize}

\subsection{Derivation Method 2: Atiyah--Singer Index Theorem}

$$\text{Index}(D_A) = \left( 77 - \frac{8}{3} \times 21 \right) \times \frac{1}{7} = 3$$

\textbf{Note}: Method 2 presents the index-theoretic formula schematically. The full intermediate computation (characteristic classes, Chern character of the gauge bundle) is deferred to future work.

\textbf{Status}: \proven{} $\Box$

% ---

\section{Relation \#2: Hierarchy Parameter $\tau = 3472/891$}

\textbf{Statement}: The hierarchy parameter is exactly rational.

\textbf{Classification}: \proven{}

\subsection{Derivation}

\textit{Step 1: Definition from topological integers}
$$\tau := \frac{\dimE(\E_8 \times \E_8) \cdot b_2(\Kseven)}{\dimE(J_3(\mathbb{O})) \cdot H^*}$$

\textit{Step 2: Substitute values}
$$\tau = \frac{496 \times 21}{27 \times 99} = \frac{10416}{2673}$$

\textit{Step 3: Reduce}
$$\gcd(10416, 2673) = 3$$
$$\tau = \frac{3472}{891}$$

\textit{Step 4: Prime factorization}
$$\tau = \frac{2^4 \times 7 \times 31}{3^4 \times 11}$$

\textit{Step 5: Numerical value}
$$\tau = 3.8967452300785634\ldots$$

\textbf{Status}: \proven{} $\Box$

% ---

\section{Relation \#3: Torsion Parameter $\kappa_T = 1/61$}

\textbf{Statement}: The topological torsion parameter equals exactly $1/61$.

\textbf{Classification}: \topomark{} (structural parameter, not physical prediction)

\subsection{Derivation}

\textit{Step 1: Define from cohomology}
$$61 = b_3(\Kseven) - \dimE(\Gtwo) - p_2 = 77 - 14 - 2 = 61$$

\textit{Step 2: Formula}
$$\kappa_T = \frac{1}{b_3 - \dimE(\Gtwo) - p_2} = \frac{1}{61}$$

\textit{Step 3: Geometric interpretation}
\begin{itemize}
\item 61 = effective degrees of freedom available for torsional deformation
\item $61 = \dimE(F_4) + N_{\mathrm{gen}}^2 = 52 + 9$
\end{itemize}

\subsection{Clarification}

\begin{table}[H]
\centering
\begin{tabular}{lll}
\toprule
Quantity & Definition & Value \\
\midrule
$\kappa_T$ & Topological capacity & $1/61$ (fixed) \\
$T_{\text{base}}$ & Torsion for torsion-free metric (Joyce) & \textbf{0} (by theorem) \\
$T_{\text{physical}}$ & Effective torsion for interactions & \textbf{Open question} \\
\bottomrule
\end{tabular}
\end{table}

\textbf{Role in predictions}: $\kappa_T$ appears in only one formula ($\alpha^{-1}$, as a small correction term $\det(g)\times\kappa_T \approx 0.033$). The other 17 predictions are independent of torsion parameter. It is primarily a structural parameter characterizing $\Kseven$, not a directly measured observable.

\textbf{Joyce's theorem}: Guarantees existence of a torsion-free metric on $\Kseven$ when perturbation bounds are satisfied.

\textbf{Status}: \topomark{} (structural, not predictive) $\Box$

% ---

\section{Relation \#4: Metric Determinant $\det(g) = 65/32$}

\textbf{Statement}: We impose $\det(g) = 65/32$ as a framework normalization fixing the overall volume scale of the $\Gtwo$ metric. This is not claimed to be a topological invariant.

\textbf{Classification}: MODEL NORMALIZATION

\subsection{Derivation}

\textit{Step 1: Define from topological structure}
$$\det(g) = p_2 + \frac{1}{b_2 + \dimE(\Gtwo) - N_{\mathrm{gen}}}$$

\textit{Step 2: Compute denominator}
$$b_2 + \dimE(\Gtwo) - N_{\mathrm{gen}} = 21 + 14 - 3 = 32$$

\textit{Step 3: Compute determinant}
$$\det(g) = 2 + \frac{1}{32} = \frac{65}{32}$$

\textit{Step 4: Alternative derivation}
$$\det(g) = \frac{w \times (\rk(\E_8) + w)}{2^5} = \frac{5 \times 13}{32} = \frac{65}{32}$$

\textbf{Verification}: The analytical metric $g = (65/32)^{1/7} \times I_7$ has $\det(g) = [(65/32)^{1/7}]^7 = 65/32$ exactly, consistent with the normalization.

\textbf{Status}: MODEL NORMALIZATION $\Box$

% ============================================
\section*{Part III: Gauge Sector}
\addcontentsline{toc}{section}{Part III: Gauge Sector}
% ============================================

\section{Relation \#5: Weinberg Angle $\sin^2\theta_W = 3/13$}

\textbf{Statement}: The weak mixing angle has exact rational form $3/13$.

\textbf{Classification}: \proven{}

\subsection{Derivation}

\textit{Step 1: Define ratio from Betti numbers}
$$\sin^2\theta_W = \frac{b_2(\Kseven)}{b_3(\Kseven) + \dimE(\Gtwo)} = \frac{21}{77 + 14} = \frac{21}{91}$$

\textit{Step 2: Simplify}
$$\gcd(21, 91) = 7$$
$$\sin^2\theta_W = \frac{3}{13} = 0.230769\ldots$$

\textit{Step 3: Experimental comparison}

\begin{table}[H]
\centering
\begin{tabular}{ll}
\toprule
Quantity & Value \\
\midrule
Experimental (PDG 2024) & $0.23122 \pm 0.00004$ \\
GIFT prediction & $0.230769$ \\
Deviation & 0.195\% \\
\bottomrule
\end{tabular}
\end{table}

\textbf{Status}: \proven{} $\Box$

% ---

\section{Relation \#6: Strong Coupling $\alpha_s = \sqrt{2}/12$}

\textbf{Statement}: The strong coupling at $M_Z$ scale.

\textbf{Classification}: \topomark{}

\subsection{Derivation}

\textit{Formula}:
$$\alpha_s(M_Z) = \frac{\sqrt{2}}{\dimE(\Gtwo) - p_2} = \frac{\sqrt{2}}{14 - 2} = \frac{\sqrt{2}}{12}$$

\textit{Components}:
\begin{itemize}
\item $\sqrt{2}$: $\E_8$ root length
\item $12 = \dimE(\Gtwo) - p_2$: Effective gauge degrees of freedom
\end{itemize}

\textit{Numerical value}: $\alpha_s = 0.117851$

\textit{Experimental comparison}:

\begin{table}[H]
\centering
\begin{tabular}{ll}
\toprule
Quantity & Value \\
\midrule
Experimental & $0.1179 \pm 0.0009$ \\
GIFT prediction & $0.11785$ \\
Deviation & 0.042\% \\
\bottomrule
\end{tabular}
\end{table}

\textbf{Status}: \topomark{} $\Box$

% ============================================
\section*{Part IV: Lepton Sector}
\addcontentsline{toc}{section}{Part IV: Lepton Sector}
% ============================================

\section{Relation \#7: Koide Parameter $Q = 2/3$}

\textbf{Statement}: The Koide parameter equals exactly 2/3.

\textbf{Classification}: \proven{}

\subsection{Derivation}

\textit{Formula}:
$$Q_{\mathrm{Koide}} = \frac{\dimE(\Gtwo)}{b_2(\Kseven)} = \frac{14}{21} = \frac{2}{3}$$

\textit{Physical definition}:
$$Q = \frac{m_e + m_\mu + m_\tau}{(\sqrt{m_e} + \sqrt{m_\mu} + \sqrt{m_\tau})^2}$$

\textit{Experimental comparison}:

\begin{table}[H]
\centering
\begin{tabular}{ll}
\toprule
Quantity & Value \\
\midrule
Experimental & $0.666661 \pm 0.000007$ \\
GIFT prediction & $0.666667$ \\
Deviation & 0.0009\% \\
\bottomrule
\end{tabular}
\end{table}

\textbf{Status}: \proven{} $\Box$

% ---

\section{Relation \#8: Tau-Electron Mass Ratio $m_\tau/m_e = 3477$}

\textbf{Statement}: The tau-electron mass ratio is exactly 3477.

\textbf{Classification}: \proven{}

\subsection{Derivation}

\textit{Formula}:
$$\frac{m_\tau}{m_e} = \dimE(\Kseven) + 10 \cdot \dimE(\E_8) + 10 \cdot H^*$$
$$= 7 + 10 \times 248 + 10 \times 99 = 7 + 2480 + 990 = 3477$$

\textit{Prime factorization}:
$$3477 = 3 \times 19 \times 61 = N_{\mathrm{gen}} \times \text{prime}(8) \times \kappa_T^{-1}$$

\textit{Experimental comparison}:

\begin{table}[H]
\centering
\begin{tabular}{ll}
\toprule
Quantity & Value \\
\midrule
Experimental & $3477.15 \pm 0.05$ \\
GIFT prediction & $3477$ (exact) \\
Deviation & 0.0043\% \\
\bottomrule
\end{tabular}
\end{table}

\textbf{Status}: \proven{} $\Box$

% ---

\section{Relation \#9: Muon-Electron Mass Ratio}

\textbf{Statement}: $m_\mu/m_e = 27^\phi$

\textbf{Classification}: \topomark{}

\subsection{Derivation}

\textit{Formula}:
$$\frac{m_\mu}{m_e} = [\dimE(J_3(\mathbb{O}))]^\phi = 27^\phi = 207.012$$

\textit{Components}:
\begin{itemize}
\item $27 = \dimE(J_3(\mathbb{O}))$: Exceptional Jordan algebra
\item $\phi = (1+\sqrt{5})/2$: Golden ratio from McKay correspondence
\end{itemize}

\textit{Experimental comparison}:

\begin{table}[H]
\centering
\begin{tabular}{ll}
\toprule
Quantity & Value \\
\midrule
Experimental & 206.768 \\
GIFT prediction & 207.01 \\
Deviation & 0.1179\% \\
\bottomrule
\end{tabular}
\end{table}

\textbf{Status}: \topomark{} $\Box$

% ============================================
\section*{Part V: Quark Sector}
\addcontentsline{toc}{section}{Part V: Quark Sector}
% ============================================

\section{Relation \#10: Strange-Down Ratio $m_s/m_d = 20$}

\textbf{Statement}: The strange-down quark mass ratio is exactly 20.

\textbf{Classification}: \proven{}

\subsection{Derivation}

\textit{Formula}:
$$\frac{m_s}{m_d} = p_2^2 \times w = 4 \times 5 = 20$$

\textit{Geometric interpretation}:
\begin{itemize}
\item $p_2^2 = 4$: Binary structure squared
\item $w = 5$: Pentagonal symmetry
\end{itemize}

\textit{Experimental comparison}:

\begin{table}[H]
\centering
\begin{tabular}{ll}
\toprule
Quantity & Value \\
\midrule
Experimental & $20.0 \pm 1.0$ \\
GIFT prediction & $20$ (exact) \\
Deviation & 0.00\% \\
\bottomrule
\end{tabular}
\end{table}

\textbf{Status}: \proven{} $\Box$

% ---

\section{Relation \#10b: Charm-Strange Ratio $m_c/m_s = 246/21$}

\textbf{Statement}: The charm-strange quark mass ratio.

\textbf{Classification}: \topomark{}

\subsection{Derivation}

\textit{Formula}:
$$\frac{m_c}{m_s} = \frac{\dimE(\E_8) - p_2}{b_2(\Kseven)} = \frac{248 - 2}{21} = \frac{246}{21} = 11.714\ldots$$

\textit{Components}:
\begin{itemize}
\item $246 = \dimE(\E_8) - p_2$: Effective $\E_8$ dimension
\item $21 = b_2(\Kseven)$: Second Betti number
\end{itemize}

\textit{Experimental comparison}:

\begin{table}[H]
\centering
\begin{tabular}{ll}
\toprule
Quantity & Value \\
\midrule
Experimental & $11.7 \pm 0.3$ \\
GIFT prediction & $11.714$ \\
Deviation & 0.12\% \\
\bottomrule
\end{tabular}
\end{table}

\textbf{Status}: \topomark{} $\Box$

% ---

\section{Relation \#10c: Bottom-Top Ratio $m_b/m_t = 1/42$}

\textbf{Statement}: The bottom-top quark mass ratio involves the constant $42 = p_2 \times N_{\mathrm{gen}} \times \dimE(\Kseven)$.

\textbf{Classification}: \topomark{}

\subsection{Derivation}

\textit{Step 1: Define the structural constant}
$$42 = p_2 \times N_{\mathrm{gen}} \times \dimE(\Kseven) = 2 \times 3 \times 7$$

This constant 42 also equals $2 \times b_2 = 2 \times 21$.

\textit{Step 2: Formula}
$$\frac{m_b}{m_t} = \frac{b_0}{42} = \frac{1}{42} = 0.02381\ldots$$

\textit{Components}:
\begin{itemize}
\item $b_0 = 1$: Zeroth Betti number
\item $42$: Structural constant from $\Kseven$ geometry
\end{itemize}

\textit{Experimental comparison}:

\begin{table}[H]
\centering
\begin{tabular}{ll}
\toprule
Quantity & Value \\
\midrule
Experimental & $0.024 \pm 0.001$ \\
GIFT prediction & $0.02381$ \\
Deviation & 0.79\% \\
\bottomrule
\end{tabular}
\end{table}

\textit{Geometric interpretation}: The same constant 42 appears in the cosmological ratio $\Omega_{\mathrm{DM}}/\Omega_b = (1 + 42)/8 = 43/8$ (Section~\ref{sec:dm_baryon}), connecting quark physics to cosmological structure through the $\Kseven$ geometry.

\textbf{Status}: \topomark{} $\Box$

% ---

\section{Relation \#10d: Up-Down Ratio $m_u/m_d = 79/168$}

\textbf{Statement}: The up-down quark mass ratio.

\textbf{Classification}: \topomark{}

\subsection{Derivation}

\textit{Formula}:
$$\frac{m_u}{m_d} = \frac{b_0 + \dimE(\E_6)}{|PSL_2(7)|} = \frac{1 + 78}{168} = \frac{79}{168} = 0.4702\ldots$$

\textit{Components}:
\begin{itemize}
\item $\dimE(\E_6) = 78$: Exceptional Lie algebra dimension
\item $|PSL_2(7)| = 168$: Order of the simple group $PSL_2(7) = \rk(\E_8) \times b_2$
\end{itemize}

\textit{Experimental comparison}:

\begin{table}[H]
\centering
\begin{tabular}{ll}
\toprule
Quantity & Value \\
\midrule
Experimental & $0.47 \pm 0.03$ \\
GIFT prediction & $0.4702$ \\
Deviation & 0.05\% \\
\bottomrule
\end{tabular}
\end{table}

\textbf{Status}: \topomark{} $\Box$

% ============================================
\section*{Part V-B: CKM Matrix}
\addcontentsline{toc}{section}{Part V-B: CKM Matrix}
% ============================================

\section{Relation \#10e: Cabibbo Angle $\sin^2\theta_{12}^{\mathrm{CKM}} = 7/31$}

\textbf{Statement}: The CKM Cabibbo mixing angle.

\textbf{Classification}: \topomark{}

\subsection{Derivation}

\textit{Formula}:
$$\sin^2\theta_{12}^{\mathrm{CKM}} = \frac{\dimE(\mathrm{fund}_{\E_7})}{\dimE(\E_8)} = \frac{56}{248} = \frac{7}{31} = 0.2258\ldots$$

\textit{Alternative expressions}:
\begin{itemize}
\item $(b_3 - b_2)/\dimE(\E_8) = (77 - 21)/248 = 56/248$
\item $(2b_2 + \dimE(\Gtwo))/\dimE(\E_8) = (42 + 14)/248 = 56/248$
\end{itemize}

\textit{Experimental comparison}:

\begin{table}[H]
\centering
\begin{tabular}{ll}
\toprule
Quantity & Value \\
\midrule
Experimental & $0.2250 \pm 0.0006$ \\
GIFT prediction & $0.2258$ \\
Deviation & 0.36\% \\
\bottomrule
\end{tabular}
\end{table}

\textbf{Status}: \topomark{} $\Box$

% ---

\section{Relation \#10f: Wolfenstein $A$ Parameter $= 83/99$}

\textbf{Statement}: The Wolfenstein $A$ parameter of the CKM matrix.

\textbf{Classification}: \topomark{}

\subsection{Derivation}

\textit{Formula}:
$$A_{\mathrm{Wolf}} = \frac{w + \dimE(\E_6)}{H^*} = \frac{5 + 78}{99} = \frac{83}{99} = 0.8384\ldots$$

\textit{Alternative expression}:
\begin{itemize}
\item $(b_3 + p_2 \times N_{\mathrm{gen}})/H^* = (77 + 6)/99 = 83/99$
\end{itemize}

\textit{Experimental comparison}:

\begin{table}[H]
\centering
\begin{tabular}{ll}
\toprule
Quantity & Value \\
\midrule
Experimental & $0.836 \pm 0.015$ \\
GIFT prediction & $0.8384$ \\
Deviation & 0.29\% \\
\bottomrule
\end{tabular}
\end{table}

\textbf{Status}: \topomark{} $\Box$

% ---

\section{Relation \#10g: CKM $\theta_{23}$ Mixing $\sin^2\theta_{23}^{\mathrm{CKM}} = 1/24$}

\textbf{Statement}: The CKM 23-mixing angle.

\textbf{Classification}: \topomark{}

\subsection{Derivation}

\textit{Formula}:
$$\sin^2\theta_{23}^{\mathrm{CKM}} = \frac{\dimE(\Kseven)}{|PSL_2(7)|} = \frac{7}{168} = \frac{1}{24} = 0.04167\ldots$$

\textit{Experimental comparison}:

\begin{table}[H]
\centering
\begin{tabular}{ll}
\toprule
Quantity & Value \\
\midrule
Experimental & $0.0412 \pm 0.0008$ \\
GIFT prediction & $0.04167$ \\
Deviation & 1.13\% \\
\bottomrule
\end{tabular}
\end{table}

\textbf{Status}: \topomark{} $\Box$

% ============================================
\section*{Part VI: Neutrino Sector}
\addcontentsline{toc}{section}{Part VI: Neutrino Sector}
% ============================================

\section{Relation \#11: CP Violation Phase $\delta_{\mathrm{CP}} = 197^\circ$}

\textbf{Statement}: The CP violation phase is exactly $197^\circ$.

\textbf{Classification}: \proven{}

\subsection{Derivation}

\textit{Formula}:
$$\delta_{\mathrm{CP}} = \dimE(\Kseven) \cdot \dimE(\Gtwo) + H^* = 7 \times 14 + 99 = 98 + 99 = 197^\circ$$

\textit{Experimental comparison}:

\begin{table}[H]
\centering
\begin{tabular}{ll}
\toprule
Quantity & Value \\
\midrule
Experimental (T2K + NOvA) & $197^\circ \pm 24^\circ$ \\
GIFT prediction & $197^\circ$ (exact) \\
Deviation & 0.00\% \\
\bottomrule
\end{tabular}
\end{table}

\textbf{Note}: The T2K+NOvA joint analysis (Nature, 2025) reports $\delta_{\mathrm{CP}}$ consistent with values in the range ${\sim}180$--$220$ degrees. DUNE (2028--2040) will test with resolution of a few degrees to ${\sim}15$ degrees. Hyper-Kamiokande provides independent verification starting ${\sim}2034$.

\textbf{Status}: \proven{} $\Box$

% ---

\section{Relation \#12: Reactor Mixing Angle $\theta_{13} = \pi/21$}

\textbf{Statement}: The reactor neutrino mixing angle.

\textbf{Classification}: \topomark{}

\subsection{Derivation}

\textit{Formula}:
$$\theta_{13} = \frac{\pi}{b_2(\Kseven)} = \frac{\pi}{21} = 8.571^\circ$$

\textit{Experimental comparison}:

\begin{table}[H]
\centering
\begin{tabular}{ll}
\toprule
Quantity & Value \\
\midrule
Experimental (NuFIT 6.0) & $8.54^\circ \pm 0.12^\circ$ \\
GIFT prediction & $8.571^\circ$ \\
Deviation & 0.368\% \\
\bottomrule
\end{tabular}
\end{table}

\textbf{Status}: \topomark{} $\Box$

% ---

\section{Relation \#13: Atmospheric Mixing Angle $\theta_{23}$}

\textbf{Statement}: The atmospheric neutrino mixing angle.

\textbf{Classification}: \topomark{}

\subsection{Derivation}

\textit{Formula}:
$$\theta_{23} = \arcsin\left(\frac{b_3 - p_2}{H^*}\right) = \arcsin\left(\frac{75}{99}\right) = \arcsin\left(\frac{25}{33}\right) = 49.251^\circ$$

\textit{Components}:
\begin{itemize}
\item $b_3 = 77$: Third Betti number (3-cycles of $\Kseven$)
\item $p_2 = 2$: dimensional ratio $\dimE(\Gtwo)/\dimE(\Kseven)$
\item $H^* = 99$: Effective cohomology ($b_2 + b_3 + 1$)
\end{itemize}

\textit{Physical interpretation}:
The atmospheric mixing angle $\theta_{23}$ governs $\tau$--$\mu$ flavor mixing. The formula $(b_3 - p_2)/H^*$ represents the relative weight of spin-corrected 3-cycles in the total cohomology. This captures how the $\tau$--$\mu$ sector couples through the 3-cycle topology of $\Kseven$, with the $p_2$ correction accounting for the dimensional ratio that distinguishes fermionic generations.

\textit{Experimental comparison}:

\begin{table}[H]
\centering
\begin{tabular}{ll}
\toprule
Quantity & Value \\
\midrule
Experimental (NuFIT 6.0) & $49.3^\circ \pm 1.0^\circ$ \\
GIFT prediction & $49.251^\circ$ \\
Deviation & 0.10\% \\
\bottomrule
\end{tabular}
\end{table}

\textbf{Status}: \topomark{} $\Box$

% ---

\section{Relation \#14: Solar Mixing Angle $\theta_{12}$}

\textbf{Statement}: The solar neutrino mixing angle.

\textbf{Classification}: \topomark{}

\subsection{Derivation}

\textit{Formula}:
$$\theta_{12} = \arctan\left(\sqrt{\frac{\delta}{\gamma_{\text{GIFT}}}}\right) = 33.40^\circ$$

\textit{Components}:
\begin{itemize}
\item $\delta = 2\pi/w^2 = 2\pi/25$
\item $\gamma_{\text{GIFT}} = 511/884$
\end{itemize}

\textit{Derivation of $\gamma_{\text{GIFT}}$}:
$$\gamma_{\text{GIFT}} = \frac{2 \cdot \rk(\E_8) + 5 \cdot H^*}{10 \cdot \dimE(\Gtwo) + 3 \cdot \dimE(\E_8)} = \frac{511}{884}$$

\textbf{Note}: The integer coefficients (2, 5, 10, 3) in $\gamma_{\text{GIFT}}$ are not yet derived from first principles. This prediction has the highest complexity cost (52) in the selection analysis.

\textit{Experimental comparison}:

\begin{table}[H]
\centering
\begin{tabular}{ll}
\toprule
Quantity & Value \\
\midrule
Experimental (NuFIT 6.0) & $33.41^\circ \pm 0.75^\circ$ \\
GIFT prediction & $33.40^\circ$ \\
Deviation & 0.030\% \\
\bottomrule
\end{tabular}
\end{table}

\textbf{Status}: \topomark{} $\Box$

% ---

\subsection{PMNS Matrix: $\sin^2$ Form}

The PMNS mixing angles can also be expressed directly as $\sin^2$ values, providing alternative topological formulas.

\subsubsection{Relation \#14b: $\sin^2\theta_{12}^{\mathrm{PMNS}} = 4/13$}

\textit{Formula}:
$$\sin^2\theta_{12}^{\mathrm{PMNS}} = \frac{b_0 + N_{\mathrm{gen}}}{\alpha_{\mathrm{sum}}} = \frac{1 + 3}{13} = \frac{4}{13} = 0.3077\ldots$$

\textit{Components}:
\begin{itemize}
\item $\alpha_{\mathrm{sum}} = 13$: Anomaly coefficient sum
\item $b_0 + N_{\mathrm{gen}} = 4$: Cohomological + generation count
\end{itemize}

\begin{table}[H]
\centering
\begin{tabular}{ll}
\toprule
Quantity & Value \\
\midrule
Experimental & $0.307 \pm 0.013$ \\
GIFT prediction & $0.3077$ \\
Deviation & 0.23\% \\
\bottomrule
\end{tabular}
\end{table}

\subsubsection{Relation \#14c: $\sin^2\theta_{23}^{\mathrm{PMNS}} = 6/11$}

\textit{Formula}:
$$\sin^2\theta_{23}^{\mathrm{PMNS}} = \frac{D_{\mathrm{bulk}} - w}{D_{\mathrm{bulk}}} = \frac{11 - 5}{11} = \frac{6}{11} = 0.5455\ldots$$

\textit{Alternative expression}:
\begin{itemize}
\item $42/b_3 = 42/77 = 6/11$ (after reduction)
\end{itemize}

\begin{table}[H]
\centering
\begin{tabular}{ll}
\toprule
Quantity & Value \\
\midrule
Experimental & $0.546 \pm 0.021$ \\
GIFT prediction & $0.5455$ \\
Deviation & 0.10\% \\
\bottomrule
\end{tabular}
\end{table}

\subsubsection{Relation \#14d: $\sin^2\theta_{13}^{\mathrm{PMNS}} = 11/496$}

\textit{Formula}:
$$\sin^2\theta_{13}^{\mathrm{PMNS}} = \frac{D_{\mathrm{bulk}}}{\dimE(\E_8 \times \E_8)} = \frac{11}{496} = 0.02218\ldots$$

\begin{table}[H]
\centering
\begin{tabular}{ll}
\toprule
Quantity & Value \\
\midrule
Experimental & $0.0220 \pm 0.0007$ \\
GIFT prediction & $0.02218$ \\
Deviation & 0.81\% \\
\bottomrule
\end{tabular}
\end{table}

\textbf{Status}: \topomark{} $\Box$

% ============================================
\section*{Part VII: Higgs \& Cosmology}
\addcontentsline{toc}{section}{Part VII: Higgs \& Cosmology}
% ============================================

\section{Relation \#15: Higgs Coupling $\lambda_H = \sqrt{17}/32$}

\textbf{Statement}: The Higgs quartic coupling has explicit geometric origin.

\textbf{Classification}: \proven{}

\subsection{Derivation}

\textit{Formula}:
$$\lambda_H = \frac{\sqrt{\dimE(\Gtwo) + N_{\mathrm{gen}}}}{2^{w}} = \frac{\sqrt{14 + 3}}{2^5} = \frac{\sqrt{17}}{32}$$

\textit{Properties of 17}:
\begin{itemize}
\item 17 is prime
\item $17 = \dimE(\Gtwo) + N_{\mathrm{gen}} = 14 + 3$
\end{itemize}

\textit{Numerical value}: $\lambda_H = 0.128847$

\textit{Experimental comparison}:

\begin{table}[H]
\centering
\begin{tabular}{ll}
\toprule
Quantity & Value \\
\midrule
Experimental & $0.129 \pm 0.003$ \\
GIFT prediction & $0.12885$ \\
Deviation & 0.119\% \\
\bottomrule
\end{tabular}
\end{table}

\textbf{Status}: \proven{} $\Box$

% ---

\section{Boson Mass Ratios}

\textbf{Statement}: The ratios of electroweak boson masses have topological origins.

\textbf{Classification}: \proven{} (v3.3)

\subsection{Relation: $m_W/m_Z = 37/42$ (v3.3 correction)}

\textit{Formula}:
$$\frac{m_W}{m_Z} = \frac{2b_2 - w}{2b_2} = \frac{42 - 5}{42} = \frac{37}{42}$$

\textit{Physical interpretation}:
\begin{itemize}
\item $2b_2 = 42$ is the structural constant ($= p_2 \times b_2$)
\item $w = 5$ is the triple identity factor
\item The ratio involves (structural constant $-$ $w$) / structural constant
\end{itemize}

\textbf{Note}: The true Euler characteristic $\chi(\Kseven) = 0$ for odd-dimensional manifolds. The constant $42 = 2b_2$ is a distinct topological invariant.

\textit{Numerical value}: $m_W/m_Z = 0.8810$

\textit{Experimental comparison}:

\begin{table}[H]
\centering
\begin{tabular}{ll}
\toprule
Quantity & Value \\
\midrule
Experimental & $0.8815 \pm 0.0002$ \\
GIFT prediction & $0.8810$ \\
Deviation & \textbf{0.06\%} \\
\bottomrule
\end{tabular}
\end{table}

\textbf{Consistency note}: The tree-level Standard Model relation $m_W/m_Z = \cos\theta_W$ gives $\sqrt{1 - 3/13} = \sqrt{10/13} \approx 0.8771$, while the direct GIFT prediction is $37/42 \approx 0.8810$ (0.45\% discrepancy). This reflects that the two predictions correspond to different renormalization schemes: $\sin^2\theta_W = 3/13$ matches the $\overline{\mathrm{MS}}$ value at $M_Z$, while $m_W/m_Z = 37/42$ matches the pole mass ratio. Radiative corrections bridge the two.

\subsection{Relation: $m_H/m_t = 56/77$}

\textit{Formula}:
$$\frac{m_H}{m_t} = \frac{\mathrm{fund}(\E_7)}{b_3} = \frac{56}{77} = \frac{8}{11}$$

\textit{Numerical value}: $m_H/m_t = 0.7273$

\begin{table}[H]
\centering
\begin{tabular}{ll}
\toprule
Quantity & Value \\
\midrule
Experimental & $0.725 \pm 0.003$ \\
GIFT prediction & $0.7273$ \\
Deviation & 0.31\% \\
\bottomrule
\end{tabular}
\end{table}

\subsection{Relation: $m_H/m_W = 81/52$}

\textit{Formula}:
$$\frac{m_H}{m_W} = \frac{N_{\mathrm{gen}} + \dimE(\E_6)}{\dimE(F_4)} = \frac{3 + 78}{52} = \frac{81}{52}$$

\textit{Numerical value}: $m_H/m_W = 1.5577$

\begin{table}[H]
\centering
\begin{tabular}{ll}
\toprule
Quantity & Value \\
\midrule
Experimental & $1.558 \pm 0.002$ \\
GIFT prediction & $1.5577$ \\
Deviation & \textbf{0.02\%} \\
\bottomrule
\end{tabular}
\end{table}

\textbf{Status}: \proven{} $\Box$

% ---

\section{Relation \#16: Dark Energy Density $\Omega_{\mathrm{DE}}$}

\textbf{Statement}: The dark energy density fraction.

\textbf{Classification}: \proven{}

\subsection{Derivation}

\textit{Formula}:
$$\Omega_{\mathrm{DE}} = \ln(p_2) \cdot \frac{b_2 + b_3}{H^*} = \ln(2) \cdot \frac{98}{99} = 0.686146$$

\textit{Binary information origin of $\ln(2)$}:
$$\ln(p_2) = \ln(2)$$
$$\ln\left(\frac{\dimE(\Gtwo)}{\dimE(\Kseven)}\right) = \ln(2)$$

\textit{Experimental comparison}:

\begin{table}[H]
\centering
\begin{tabular}{ll}
\toprule
Quantity & Value \\
\midrule
Experimental (Planck 2020) & $0.6847 \pm 0.0073$ \\
GIFT prediction & $0.6861$ \\
Deviation & 0.211\% \\
\bottomrule
\end{tabular}
\end{table}

\textbf{Status}: \proven{} $\Box$

% ---

\section{Relation \#17: Spectral Index $n_s$}

\textbf{Statement}: The primordial scalar spectral index.

\textbf{Classification}: \proven{}

\subsection{Derivation}

\textit{Formula}:
$$n_s = \frac{\zeta(D_{\mathrm{bulk}})}{\zeta(w)} = \frac{\zeta(11)}{\zeta(5)} = 0.9649$$

\textit{Components}:
\begin{itemize}
\item $\zeta(11)$: From 11D bulk spacetime
\item $\zeta(5)$: From pentagonal index
\end{itemize}

\textit{Experimental comparison}:

\begin{table}[H]
\centering
\begin{tabular}{ll}
\toprule
Quantity & Value \\
\midrule
Experimental (Planck 2020) & $0.9649 \pm 0.0042$ \\
GIFT prediction & $0.9649$ \\
Deviation & 0.004\% \\
\bottomrule
\end{tabular}
\end{table}

\textbf{Status}: \proven{} $\Box$

% ---

\section{Relation \#17c: Dark Matter to Baryon Ratio $\Omega_{\mathrm{DM}}/\Omega_b = 43/8$}
\label{sec:dm_baryon}

\textbf{Statement}: The dark matter to baryon density ratio.

\textbf{Classification}: \topomark{}

\subsection{Derivation}

\textit{Formula}:
$$\frac{\Omega_{\mathrm{DM}}}{\Omega_b} = \frac{b_0 + 42}{\rk(\E_8)} = \frac{1 + 42}{8} = \frac{43}{8} = 5.375$$

\textit{Components}:
\begin{itemize}
\item $42 = p_2 \times N_{\mathrm{gen}} \times \dimE(\Kseven)$: The same constant appearing in $m_b/m_t = 1/42$
\item $\rk(\E_8) = 8$: Cartan subalgebra dimension
\end{itemize}

\textit{Experimental comparison}:

\begin{table}[H]
\centering
\begin{tabular}{ll}
\toprule
Quantity & Value \\
\midrule
Experimental (Planck 2020) & $5.375 \pm 0.05$ \\
GIFT prediction & $5.375$ \\
Deviation & 0.00\% \\
\bottomrule
\end{tabular}
\end{table}

\textbf{Status}: \topomark{} $\Box$

% ---

\section{Relation \#17d: Reduced Hubble Parameter $h = 167/248$}

\textbf{Statement}: The reduced Hubble parameter $H_0 = 100h$ km/s/Mpc.

\textbf{Classification}: \topomark{}

\subsection{Derivation}

\textit{Formula}:
$$h = \frac{|PSL_2(7)| - b_0}{\dimE(\E_8)} = \frac{168 - 1}{248} = \frac{167}{248} = 0.6734\ldots$$

\textit{Experimental comparison}:

\begin{table}[H]
\centering
\begin{tabular}{ll}
\toprule
Quantity & Value \\
\midrule
Experimental (Planck 2020) & $0.674 \pm 0.005$ \\
GIFT prediction & $0.6734$ \\
Deviation & 0.09\% \\
\bottomrule
\end{tabular}
\end{table}

\textbf{Status}: \topomark{} $\Box$

% ---

\section{Relation \#17e: Baryon Fraction $\Omega_b/\Omega_m = 5/32$}

\textbf{Statement}: The baryon fraction of total matter.

\textbf{Classification}: \topomark{}

\subsection{Derivation}

\textit{Formula}:
$$\frac{\Omega_b}{\Omega_m} = \frac{w}{\det(g)_{\mathrm{den}}} = \frac{5}{32} = 0.15625$$

\textit{Experimental comparison}:

\begin{table}[H]
\centering
\begin{tabular}{ll}
\toprule
Quantity & Value \\
\midrule
Experimental (Planck 2020) & $0.156 \pm 0.003$ \\
GIFT prediction & $0.15625$ \\
Deviation & 0.16\% \\
\bottomrule
\end{tabular}
\end{table}

\textbf{Status}: \topomark{} $\Box$

% ---

\section{Relation \#17f: Amplitude of Fluctuations $\sigma_8 = 17/21$}

\textbf{Statement}: The amplitude of matter fluctuations at $8\,h^{-1}$ Mpc.

\textbf{Classification}: \topomark{}

\subsection{Derivation}

\textit{Formula}:
$$\sigma_8 = \frac{p_2 + \det(g)_{\mathrm{den}}}{42} = \frac{2 + 32}{42} = \frac{34}{42} = \frac{17}{21} = 0.8095\ldots$$

\textit{Experimental comparison}:

\begin{table}[H]
\centering
\begin{tabular}{ll}
\toprule
Quantity & Value \\
\midrule
Experimental (Planck 2020) & $0.811 \pm 0.006$ \\
GIFT prediction & $0.8095$ \\
Deviation & 0.18\% \\
\bottomrule
\end{tabular}
\end{table}

\textbf{Status}: \topomark{} $\Box$

% ---

\section{Relation \#17g: Primordial Helium Fraction $Y_p = 15/61$}

\textbf{Statement}: The primordial helium mass fraction from Big Bang nucleosynthesis.

\textbf{Classification}: \topomark{}

\subsection{Derivation}

\textit{Formula}:
$$Y_p = \frac{b_0 + \dimE(\Gtwo)}{\kappa_T^{-1}} = \frac{1 + 14}{61} = \frac{15}{61} = 0.2459\ldots$$

\textit{Experimental comparison}:

\begin{table}[H]
\centering
\begin{tabular}{ll}
\toprule
Quantity & Value \\
\midrule
Experimental & $0.245 \pm 0.003$ \\
GIFT prediction & $0.2459$ \\
Deviation & 0.37\% \\
\bottomrule
\end{tabular}
\end{table}

\textbf{Status}: \topomark{} $\Box$

% ---

\section{Relation \#17b: Matter Density $\Omega_m$}

\textbf{Statement}: The matter density fraction derives from dark energy via $\sqrt{w}$.

\textbf{Classification}: DERIVED (from pentagonal triple identity + $\Omega_{\mathrm{DE}}$)

\subsection{Derivation}

\textit{Step 1: Establish $\sqrt{w}$ as structural}

From the pentagonal triple identity (S1, Section 2.3):
$$w = \frac{\dimE(\Gtwo) + 1}{N_{\mathrm{gen}}} = \frac{b_2}{N_{\mathrm{gen}}} - p_2 = \dimE(\Gtwo) - \rk(\E_8) - 1 = 5$$

Therefore $\sqrt{w} = \sqrt{5}$ is a derived quantity.

\textit{Step 2: Matter--dark energy ratio}

The cosmological density ratio:
$$\frac{\Omega_{\mathrm{DE}}}{\Omega_m} = \sqrt{w} = \sqrt{5}$$

\textit{Step 3: Compute $\Omega_m$}

Using $\Omega_{\mathrm{DE}} = \ln(2) \times (b_2 + b_3)/H^* = 0.6861$ (Relation \#16):
$$\Omega_m = \frac{\Omega_{\mathrm{DE}}}{\sqrt{w}} = \frac{\ln(2) \times 98/99}{\sqrt{5}} = \frac{0.6861}{2.236} = 0.3068$$

\textit{Step 4: Verify closure}

$$\Omega_{\mathrm{total}} = \Omega_{\mathrm{DE}} + \Omega_m = 0.6861 + 0.3068 = 0.9929$$

$\Omega_{\mathrm{total}} = 0.993$, a 0.7\% deficit from exact closure. This tension (comparable to the 2.7\% deviation in $\Omega_m$ itself) represents the framework's least precise cosmological prediction. DESI and Euclid will test $\Omega_m$ to sub-percent precision by 2028.

\textit{Experimental comparison}:

\begin{table}[H]
\centering
\begin{tabular}{ll}
\toprule
Quantity & Value \\
\midrule
Experimental (Planck 2020) & $0.3153 \pm 0.007$ \\
GIFT prediction & $0.3068$ \\
Deviation & 2.7\% \\
\bottomrule
\end{tabular}
\end{table}

\subsection{Interpretation}

The $\sqrt{5}$ ratio between dark energy and matter densities emerges from the same structural constant ($w = 5$) that determines:
\begin{itemize}
\item $\det(g) = 65/32$ (metric determinant)
\item $|W(\E_8)|$ factorization (group theory)
\item $N_{\mathrm{gen}}^3$ coefficient in $|W(\E_8)|$ (topology)
\end{itemize}

\textbf{Status}: DERIVED (structural, 2.7\% deviation) $\Box$

% ---

\section{Relation \#18: Fine Structure Constant $\alpha^{-1}$}

\textbf{Statement}: The inverse fine structure constant.

\textbf{Classification}: \topomark{}

\subsection{Derivation}

\textit{Formula}:
$$\alpha^{-1}(M_Z) = \frac{\dimE(\E_8) + \rk(\E_8)}{2} + \frac{H^*}{D_{\mathrm{bulk}}} + \det(g) \cdot \kappa_T$$
$$= 128 + 9 + \frac{65}{32} \times \frac{1}{61} = 137.033$$

\textit{Components}:
\begin{itemize}
\item $128 = (248 + 8)/2$: Algebraic
\item $9 = 99/11$: Bulk impedance
\item $65/1952$: Torsional correction
\end{itemize}

\textit{Experimental comparison}:

\begin{table}[H]
\centering
\begin{tabular}{ll}
\toprule
Quantity & Value \\
\midrule
Experimental & 137.035999 \\
GIFT prediction & 137.033 \\
Deviation & 0.002\% \\
\bottomrule
\end{tabular}
\end{table}

\textbf{Status}: \topomark{} $\Box$

% ============================================
\section*{Part VIII: Summary Table}
\addcontentsline{toc}{section}{Part VIII: Summary Table}
% ============================================

\section{The 18 Core Dimensionless Relations}

\textbf{Note}: All predictions use only topological invariants ($b_2$, $b_3$, $\dimE(\Gtwo)$, etc.). None depend on the realized torsion value $T$. The 33 predictions decompose as 18 core (this table) + 15 extended (Part IX). Relation \#19 ($\Omega_m$) is listed here as DERIVED from $\Omega_{\mathrm{DE}}$ via the pentagonal triple identity.

\begin{table}[H]
\centering
\tiny
\begin{tabular}{clccccc}
\toprule
\# & Relation & Formula & Value & Exp. & Dev. & Status \\
\midrule
1 & $N_{\mathrm{gen}}$ & Atiyah--Singer & 3 & 3 & exact & \proven{} \\
2 & $\tau$ & $496\times21/(27\times99)$ & 3472/891 & --- & --- & \proven{} \\
3 & $\kappa_T$ & $1/(77-14-2)$ & 1/61 & --- & --- & \topomark{} \\
4 & $\det(g)$ & $5\times13/32$ & 65/32 & --- & --- & MODEL NORM. \\
5 & $\sin^2\theta_W$ & 21/91 & 3/13 & 0.23122 & 0.195\% & \proven{} \\
6 & $\alpha_s$ & $\sqrt{2}/12$ & 0.11785 & 0.1179 & 0.042\% & \topomark{} \\
7 & $Q_{\mathrm{Koide}}$ & 14/21 & 2/3 & 0.666661 & 0.0009\% & \proven{} \\
8 & $m_\tau/m_e$ & 7+2480+990 & 3477 & 3477.15 & 0.0043\% & \proven{} \\
9 & $m_\mu/m_e$ & $27^\phi$ & 207.01 & 206.768 & 0.118\% & \topomark{} \\
10 & $m_s/m_d$ & $4\times5$ & 20 & 20.0 & 0.00\% & \proven{} \\
11 & $\delta_{\mathrm{CP}}$ & $7\times14+99$ & $197^\circ$ & $197^\circ$ & 0.00\% & \proven{} \\
12 & $\theta_{13}$ & $\pi/21$ & $8.57^\circ$ & $8.54^\circ$ & 0.368\% & \topomark{} \\
13 & $\theta_{23}$ & $\arcsin((b_3{-}p_2)/H^*)$ & $49.25^\circ$ & $49.3^\circ$ & 0.10\% & \topomark{} \\
14 & $\theta_{12}$ & $\arctan(\ldots)$ & $33.40^\circ$ & $33.41^\circ$ & 0.030\% & \topomark{} \\
15 & $\lambda_H$ & $\sqrt{17}/32$ & 0.1288 & 0.129 & 0.119\% & \proven{} \\
16 & $\Omega_{\mathrm{DE}}$ & $\ln(2)\times(b_2{+}b_3)/H^*$ & 0.6861 & 0.6847 & 0.211\% & \proven{} \\
17 & $n_s$ & $\zeta(11)/\zeta(5)$ & 0.9649 & 0.9649 & 0.004\% & \proven{} \\
18 & $\alpha^{-1}$ & 128+9+corr & 137.033 & 137.036 & 0.002\% & \topomark{} \\
19 & $\Omega_m$ & $\Omega_{\mathrm{DE}}/\sqrt{w}$ & 0.3068 & 0.3153 & 2.7\% & DERIVED \\
\bottomrule
\end{tabular}
\end{table}

*$\kappa_T$ is a topological parameter, not a directly measured observable. It appears as a small correction in $\alpha^{-1}$ (relation \#18) via $\det(g) \times \kappa_T \approx 0.033$.

% ---

\section{Deviation Statistics}

\begin{table}[H]
\centering
\begin{tabular}{lcc}
\toprule
Range & Count & Percentage \\
\midrule
0.00\% (exact) & 4 & 22\% \\
$<0.01\%$ & 3 & 17\% \\
0.01--0.1\% & 4 & 22\% \\
0.1--0.5\% & 7 & 39\% \\
\bottomrule
\end{tabular}
\end{table}

\textbf{Mean deviation}: 0.26\% (PDG 2024, 33 observables)

% ---

\section{Statistical Uniqueness of $(b_2=21, b_3=77)$}

The comprehensive Monte Carlo validation (192,349 configurations tested, zero outperforming GIFT) is presented in the main paper, Section 7. Key results: $\E_8\times\E_8$ achieves approximately $10\times$ better agreement than alternatives; $\Gtwo$ holonomy approximately $5\times$ better than Calabi--Yau; significance $> 4.5\sigma$.

Formula-level selection analysis (main paper Section 7.5) shows GIFT formulas rank first or near-first among all bounded-grammar alternatives, with joint null-model $p < 1.5 \times 10^{-5}$.

% ============================================
\section*{Part IX: Observable Catalog}
\addcontentsline{toc}{section}{Part IX: Observable Catalog}
% ============================================

\section{Structural Redundancy and Expression Counts}

Each prediction admits multiple algebraically distinct representations that reduce to the same fraction. This multiplicity provides a measure of structural robustness: quantities arising from many paths through the topological invariants are less likely to represent numerical coincidence.

\subsection{Classification Scheme}

\begin{table}[H]
\centering
\begin{tabular}{lll}
\toprule
Classification & Expressions & Interpretation \\
\midrule
\textbf{CANONICAL} & $\geq 20$ & Maximally over-determined; emerges from algebraic web \\
\textbf{ROBUST} & 10--19 & Highly constrained; multiple independent derivations \\
\textbf{SUPPORTED} & 5--9 & Structural redundancy \\
\textbf{DERIVED} & 2--4 & Dual derivation minimum \\
\textbf{SINGULAR} & 1 & Unique path (possible coincidence) \\
\bottomrule
\end{tabular}
\end{table}

\subsection{Core 18 Predictions with Expression Counts}

\begin{table}[H]
\centering
\tiny
\begin{tabular}{clcccccc}
\toprule
\# & Observable & Formula & Value & Exp. & Dev. & Expr. & Class \\
\midrule
1 & $N_{\mathrm{gen}}$ & Atiyah--Singer & 3 & 3 & 0.00\% & 24+ & CANONICAL \\
2 & $\sin^2\theta_W$ & $b_2/(b_3{+}\dimE_{\Gtwo})$ & 3/13 & 0.2312 & 0.20\% & 14 & ROBUST \\
3 & $\alpha_s(M_Z)$ & $\sqrt{2}/12$ & 0.1179 & 0.1179 & 0.04\% & 9 & SUPPORTED \\
4 & $\lambda_H$ & $\sqrt{17}/32$ & 0.1288 & 0.129 & 0.12\% & 4 & DERIVED \\
5 & $\alpha^{-1}$ & 128+9+corr & 137.033 & 137.036 & 0.002\% & 3 & DERIVED \\
6 & $Q_{\mathrm{Koide}}$ & $\dimE_{\Gtwo}/b_2$ & 2/3 & 0.6667 & 0.001\% & 20 & CANONICAL \\
7 & $m_\tau/m_e$ & $7{+}10{\times}248{+}10{\times}99$ & 3477 & 3477.2 & 0.004\% & 3 & DERIVED \\
8 & $m_\mu/m_e$ & $27^\phi$ & 207.01 & 206.77 & 0.12\% & 2 & DERIVED \\
9 & $m_s/m_d$ & $p_2^2{\times}w$ & 20 & 20.0 & 0.00\% & 14 & ROBUST \\
10 & $m_b/m_t$ & $1/(2b_2)$ & 1/42 & 0.024 & 0.79\% & 21 & CANONICAL \\
11 & $m_u/m_d$ & $(1{+}\dimE_{\E_6})/PSL_{2,7}$ & 79/168 & 0.47 & 0.05\% & 1 & SINGULAR \\
12 & $\delta_{\mathrm{CP}}$ & $\dimE_{\Kseven}{\times}\dimE_{\Gtwo}{+}H^*$ & $197^\circ$ & $197^\circ$ & 0.00\% & 3 & DERIVED \\
13 & $\theta_{13}$ & $\pi/b_2$ & $8.57^\circ$ & $8.54^\circ$ & 0.37\% & 3 & DERIVED \\
14 & $\theta_{23}$ & $\arcsin((b_3{-}p_2)/H^*)$ & $49.25^\circ$ & $49.3^\circ$ & 0.10\% & 2 & DERIVED \\
15 & $\theta_{12}$ & $\arctan(\sqrt{\delta/\gamma})$ & $33.40^\circ$ & $33.41^\circ$ & 0.03\% & 2 & DERIVED \\
16 & $\Omega_{\mathrm{DE}}$ & $\ln(2){\times}(b_2{+}b_3)/H^*$ & 0.6861 & 0.6847 & 0.21\% & 2 & DERIVED \\
17 & $n_s$ & $\zeta(11)/\zeta(5)$ & 0.9649 & 0.9649 & 0.004\% & 2 & DERIVED \\
18 & $\det(g)$ & 65/32 & 2.0313 & --- & --- & 8 & MODEL NORM. \\
\bottomrule
\end{tabular}
\end{table}

\textbf{Distribution}: 4 CANONICAL (22\%), 4 ROBUST (22\%), 2 SUPPORTED (11\%), 7 DERIVED (39\%), 1 SINGULAR (6\%).

\subsection{Extended Predictions (15)}

\begin{table}[H]
\centering
\tiny
\begin{tabular}{clcccccc}
\toprule
\# & Observable & Formula & Value & Exp. & Dev. & Expr. & Class \\
\midrule
19 & $\sin^2\theta_{12}^{\mathrm{PMNS}}$ & $(1{+}N_{\mathrm{gen}})/\alpha_{\mathrm{sum}}$ & 4/13 & 0.307 & 0.23\% & 28 & CANONICAL \\
20 & $\sin^2\theta_{23}^{\mathrm{PMNS}}$ & $(D_{\mathrm{bulk}}{-}w)/D_{\mathrm{bulk}}$ & 6/11 & 0.546 & 0.10\% & 15 & ROBUST \\
21 & $\sin^2\theta_{13}^{\mathrm{PMNS}}$ & $D_{\mathrm{bulk}}/\dimE_{\E_8^2}$ & 11/496 & 0.022 & 0.81\% & 5 & SUPPORTED \\
22 & $\sin^2\theta_{12}^{\mathrm{CKM}}$ & 7/31 & 0.2258 & 0.225 & 0.36\% & 16 & ROBUST \\
23 & $A_{\mathrm{Wolf}}$ & $(w{+}\dimE_{\E_6})/H^*$ & 83/99 & 0.836 & 0.29\% & 4 & DERIVED \\
24 & $\sin^2\theta_{23}^{\mathrm{CKM}}$ & $\dimE_{\Kseven}/PSL_{2,7}$ & 1/24 & 0.041 & 1.13\% & 3 & DERIVED \\
25 & $m_H/m_t$ & 8/11 & 0.7273 & 0.725 & 0.31\% & 19 & ROBUST \\
26 & $m_H/m_W$ & 81/52 & 1.5577 & 1.558 & 0.02\% & 1 & SINGULAR \\
27 & $m_W/m_Z$ & $(2b_2{-}w)/(2b_2) = 37/42$ & 0.8810 & 0.8815 & \textbf{0.06\%} & 8 & SUPPORTED \\
28 & $m_\mu/m_\tau$ & 5/84 & 0.0595 & 0.0595 & 0.04\% & 9 & SUPPORTED \\
29 & $\Omega_{\mathrm{DM}}/\Omega_b$ & $(1{+}42)/\rk_{\E_8}$ & 43/8 & 5.375 & 0.00\% & 6 & SUPPORTED \\
30 & $\Omega_b/\Omega_m$ & $w/\det(g)_{\mathrm{den}}$ & 5/32 & 0.156 & 0.16\% & 7 & SUPPORTED \\
31 & $\Omega_\Lambda/\Omega_m$ & $(\det_{g,\mathrm{den}}{-}\dimE_{\Kseven})/D_{\mathrm{bulk}}$ & 25/11 & 2.27 & 0.12\% & 6 & SUPPORTED \\
32 & $h$ & $(PSL_{2,7}{-}1)/\dimE_{\E_8}$ & 167/248 & 0.674 & 0.09\% & 3 & DERIVED \\
33 & $\sigma_8$ & $(p_2{+}\det_{g,\mathrm{den}})/(2b_2)$ & 34/42 & 0.811 & 0.18\% & 4 & DERIVED \\
\bottomrule
\end{tabular}
\end{table}

\subsection{Illustrative Examples of Multiple Expressions}

\textbf{$\sin^2\theta_W = 3/13$} (14 algebraically distinct representations):

\begin{table}[H]
\centering
\small
\begin{tabular}{cll}
\toprule
\# & Expression & Evaluation \\
\midrule
1 & $N_{\mathrm{gen}} / \alpha_{\mathrm{sum}}$ & $3/13$ \\
2 & $N_{\mathrm{gen}} / (p_2 + D_{\mathrm{bulk}})$ & $3/(2+11) = 3/13$ \\
3 & $b_2 / (b_3 + \dimE_{\Gtwo})$ & $21/91 = 3/13$ \\
4 & $\dimE(J_3\mathbb{O}) / (\dimE_{F_4} + \det(g)_{\mathrm{num}})$ & $27/117 = 3/13$ \\
5 & $(b_0 + \dimE_{\Gtwo}) / \det(g)_{\mathrm{num}}$ & $15/65 = 3/13$ \\
6 & $(p_2 + b_0) / \alpha_{\mathrm{sum}}$ & $3/13$ \\
7 & $\dimE_{\Kseven} / (b_2 + \dimE_{\Kseven} + \dimE_{\Gtwo})$ & $7/42 \neq 3/13$ \texttimes \\
\bottomrule
\end{tabular}
\end{table}

(Expression 7 illustrates that not all combinations work; only those reducing to $3/13$ are valid.)

\textbf{$Q_{\mathrm{Koide}} = 2/3$} (20 algebraically distinct representations):

\begin{table}[H]
\centering
\small
\begin{tabular}{cll}
\toprule
\# & Expression & Evaluation \\
\midrule
1 & $p_2 / N_{\mathrm{gen}}$ & $2/3$ \\
2 & $\dimE_{\Gtwo} / b_2$ & $14/21 = 2/3$ \\
3 & $\dimE_{F_4} / \dimE_{\E_6}$ & $52/78 = 2/3$ \\
4 & $\rk_{\E_8} / (w + \dimE_{\Kseven})$ & $8/12 = 2/3$ \\
5 & $(\dimE_{\Gtwo} - \rk_{\E_8}) / (\rk_{\E_8} + 1)$ & $6/9 = 2/3$ \\
\bottomrule
\end{tabular}
\end{table}

\textbf{$m_b/m_t = 1/42$} (21 algebraically distinct representations):

\begin{table}[H]
\centering
\small
\begin{tabular}{cll}
\toprule
\# & Expression & Evaluation \\
\midrule
1 & $b_0 / (2b_2)$ & $1/42$ \\
2 & $(b_0 + N_{\mathrm{gen}}) / PSL(2{,}7)$ & $4/168 = 1/42$ \\
3 & $p_2 / (\dimE_{\Kseven} + b_3)$ & $2/84 = 1/42$ \\
4 & $N_{\mathrm{gen}} / (\dimE(J_3\mathbb{O}) + H^*)$ & $3/126 = 1/42$ \\
5 & $\dimE_{\Kseven} / (\dimE_{\E_8} + \dimE(J_3\mathbb{O}) + \dimE_{\Kseven})$ & $7/294 = 1/42$ \\
\bottomrule
\end{tabular}
\end{table}

The ratio $m_b/m_t = 1/42 = 1/(2b_2)$ illustrates structural redundancy: the bottom-to-top mass hierarchy equals the inverse of the structural constant $2b_2 = p_2 \times b_2$.

\textbf{Note}: The true Euler characteristic $\chi(\Kseven) = 0$ for $\Gtwo$ manifolds (odd-dimensional). The constant 42 is the structural invariant $2b_2$.

\subsection{The Algebraic Web}

The topological constants satisfy interconnected identities:

\begin{table}[H]
\centering
\small
\begin{tabular}{lll}
\toprule
Identity & Left side & Right side \\
\midrule
Fiber-holonomy & $\dimE(\Gtwo) = 14$ & $p_2 \times \dimE(\Kseven) = 2 \times 7$ \\
Gauge moduli & $b_2 = 21$ & $N_{\mathrm{gen}} \times \dimE(\Kseven) = 3 \times 7$ \\
Matter-holonomy & $b_3 + \dimE(\Gtwo) = 91$ & $\dimE(\Kseven) \times \alpha_{\mathrm{sum}} = 7 \times 13$ \\
Fano order & $PSL(2{,}7) = 168$ & $\rk(\E_8) \times b_2 = 8 \times 21$ \\
Fano order & $PSL(2{,}7) = 168$ & $N_{\mathrm{gen}} \times \mathrm{fund}(\E_7) = 3 \times 56$ \\
Anomaly sum & $\alpha_{\mathrm{sum}} = 13$ & $\rk(\E_8) + w = 8 + 5$ \\
\bottomrule
\end{tabular}
\end{table}

These relations form a closed algebraic system. The mod-7 structure ($\dimE(\Kseven) = 7$ divides $\dimE(\Gtwo)$, $b_2$, $b_3$, $PSL(2{,}7)$) reflects the Fano plane underlying octonion multiplication.

\subsection{Fibonacci--Lucas Embedding}

The GIFT constants embed naturally into the Fibonacci ($F_n$) and Lucas ($L_n$) sequences:

\begin{table}[H]
\centering
\begin{tabular}{ccll}
\toprule
$n$ & $F_n$ & GIFT Constant & Role \\
\midrule
3 & 2 & $p_2$ & Dimensional ratio \\
4 & 3 & $N_{\mathrm{gen}}$ & Fermion generations \\
5 & 5 & $w$ & Pentagonal symmetry \\
6 & 8 & $\rk(\E_8)$ & Cartan subalgebra \\
7 & 13 & $\alpha^2_B$ sum & Structure coefficient \\
8 & 21 & $b_2$ & Second Betti number \\
\bottomrule
\end{tabular}
\end{table}

This sequence propagates via the recurrence:

$$F_3 + F_4 = F_5 \quad \Rightarrow \quad p_2 + N_{\mathrm{gen}} = w$$

Lucas numbers also appear naturally:

\begin{table}[H]
\centering
\begin{tabular}{ccl}
\toprule
$L_n$ & Value & GIFT Role \\
\midrule
$L_4$ & 7 & $\dimE(\Kseven)$ \\
$L_5$ & 11 & $D_{\mathrm{bulk}}$ \\
$L_8$ & 47 & Scale bridge exponent \\
\bottomrule
\end{tabular}
\end{table}

The Lucas identity $L_8 = F_7 + F_9 = 13 + 34$ decomposes as:

$$L_8 = \alpha_{\mathrm{sum}}^B + d_{\mathrm{hidden}} = 13 + 34 = 47$$

This structure reflects the icosahedral geometry underlying the McKay correspondence $\E_8 \leftrightarrow 2I$, where icosahedral coordinates involve the golden ratio $\phi = \lim(F_{n+1}/F_n)$.

\textbf{Status}: EXPLORATORY (mathematical fact; physical significance unclear)

% ============================================
% REFERENCES
% ============================================

\begin{thebibliography}{99}

\bibitem{joyce2000} Joyce, D. D. (2000). \textit{Compact Manifolds with Special Holonomy}. Oxford.

\bibitem{atiyah1968} Atiyah, M. F., Singer, I. M. (1968). \textit{The index of elliptic operators}.

\bibitem{pdg2024} Particle Data Group (2024). \textit{Review of Particle Physics}. Phys.\ Rev.\ D 110, 030001.

\bibitem{nufit2024} NuFIT 6.0 (2024). Global neutrino oscillation analysis. \url{www.nu-fit.org}.

\bibitem{planck2020} Planck Collaboration (2020). Cosmological parameters. A\&A 641, A6.

\bibitem{t2k2025} T2K, NOvA Collaborations (2025). Nature 646(8086), 818--824. DOI: 10.1038/s41586-025-09599-3

\end{thebibliography}

\subsection*{Related Works}
\begin{itemize}
\item GIFT Framework, \textit{Geometric Information Field Theory} (main paper)
\item GIFT Framework, \textit{Supplement S1: Mathematical Foundations}
\item GIFT Framework, \textit{Numerical $\Gtwo$ Metric Construction via Physics-Informed Neural Networks} (companion numerical paper)
\end{itemize}

\vfill
\noindent\rule{\textwidth}{0.2pt}
\textit{GIFT Framework - Supplement S2}\\
\textit{Complete Derivations: 33 Dimensionless Relations}

\end{document}
