\documentclass[11pt,a4paper]{article}

% ============================================
% ENCODING & FONTS
% ============================================
\usepackage[utf8]{inputenc}
\usepackage[T1]{fontenc}
\usepackage{lmodern}

% ============================================
% PAGE LAYOUT
% ============================================
\usepackage[margin=1.618cm, top=2.618cm, bottom=2.618cm]{geometry}

% ============================================
% ESSENTIAL PACKAGES
% ============================================
\usepackage{float}
\usepackage{caption}
\usepackage{setspace}
\usepackage{fancyhdr}
\usepackage{xcolor}
\usepackage{hyperref}
\usepackage{amsmath}
\usepackage{amssymb}
\usepackage{booktabs}
\usepackage{longtable}
\usepackage{array}
\usepackage{listings}
\usepackage{graphicx}
\DeclareUnicodeCharacter{00B0}{\ensuremath{^\circ}}

% ============================================
% LISTINGS CONFIGURATION
% ============================================
\lstset{
    basicstyle=\small\ttfamily,
    breaklines=true,
    frame=single,
    keepspaces=true,
    showstringspaces=false,
    breakatwhitespace=true,
    aboveskip=0.8em,
    belowskip=0.8em
}

% ============================================
% TITLE FORMATTING
% ============================================
\usepackage{titling}
\pretitle{\LARGE\bfseries}
\posttitle{\vspace{-0.4em}}
\preauthor{}
\postauthor{}
\predate{}
\postdate{}
\setlength{\droptitle}{-2.0em}

% ============================================
% HEADER/FOOTER
% ============================================
\setlength{\headheight}{14pt}
\pagestyle{fancy}
\fancyhf{}
\fancyhead[L]{Explicit $\Gtwo$ Metric --- GIFT Framework}
\fancyhead[R]{\thepage}
\renewcommand{\headrulewidth}{0.2pt}

% ============================================
% HYPERREF
% ============================================
\hypersetup{
    colorlinks=true,
    linkcolor=blue,
    citecolor=blue,
    urlcolor=blue,
    pdftitle={A Numerical Candidate for a Torsion-Free G2 Structure on a Compact TCS 7-Manifold},
    pdfauthor={Brieuc de La Fourniere}
}

% ============================================
% SPACING
% ============================================
\setstretch{1.2}
\setlength{\parskip}{0.4em}
\setlength{\parindent}{0pt}

% ============================================
% CUSTOM COMMANDS
% ============================================
\newcommand{\E}{\mathrm{E}}
\newcommand{\Gtwo}{\mathrm{G}_2}
\newcommand{\Kseven}{K_7}
\newcommand{\dimE}{\mathrm{dim}}
\newcommand{\Weyl}{\mathrm{Weyl}}
\newcommand{\rk}{\mathrm{rank}}
\newcommand{\proven}{\textsc{Proven}}
\newcommand{\topomark}{\textsc{Topological}}
\newcommand{\derived}{\textsc{Derived}}
\newcommand{\AdS}{\mathrm{AdS}}
\newcommand{\SU}{\mathrm{SU}}
\newcommand{\SO}{\mathrm{SO}}
\newcommand{\U}{\mathrm{U}}
\newcommand{\Spin}{\mathrm{Spin}}
\newcommand{\Hol}{\mathrm{Hol}}
\newcommand{\Ric}{\mathrm{Ric}}
\newcommand{\RMS}{\mathrm{RMS}}
\newcommand{\ACyl}{\mathrm{ACyl}}
\newcommand{\PINN}{\mathrm{PINN}}
\newcommand{\TCS}{\mathrm{TCS}}
\newcommand{\Sym}{\mathrm{Sym}}
\newcommand{\chol}{\mathrm{chol}}
\newcommand{\Tr}{\mathrm{Tr}}

\pdfstringdefDisableCommands{%
  \def\Gtwo{G2}%
  \def\Kseven{K7}%
  \def\E{E}%
  \def\dimE{dim}%
  \def\Weyl{Weyl}%
  \def\rk{rank}%
  \def\proven{Proven}%
  \def\topomark{Topological}%
  \def\AdS{AdS}%
  \def\SO{SO}%
  \def\Spin{Spin}%
}

\title{%
\LARGE\textbf{A Numerical Candidate for a Torsion-Free $\Gtwo$ Structure\\[0.2em]on a Compact TCS 7-Manifold}
}
\author{}
\date{}

\begin{document}

\maketitle
\noindent\rule{\textwidth}{0.2pt}

\noindent\textbf{Author}: Brieuc de La Fourni\`ere

\noindent Independent researcher

\vfill

\begin{abstract}
We construct a numerical candidate for a Riemannian metric with holonomy contained in $\Gtwo$ on a computational proxy for the neck region of a compact twisted connected sum (TCS) 7-manifold $\Kseven$, with Betti numbers $b_2 = 21$ and $b_3 = 77$. The construction proceeds in three stages: (i)~an analytical target metric derived from the $\Gtwo$ representation-theoretic decomposition $\Lambda^3(\mathbb{R}^7) = \Lambda^3_1 \oplus \Lambda^3_7 \oplus \Lambda^3_{27}$ and period integrals on the moduli space of $\Gtwo$ structures; (ii)~a Cholesky-parameterized physics-informed neural network (PINN) that reconstructs a spatially varying metric field $g(x)$ on the computational domain; (iii)~verification against five geometric criteria. The resulting $7\times7$ metric satisfies a prescribed determinant $\det(g) = 65/32$ to 8 significant figures ($4 \times 10^{-8}\,\%$ deviation), has torsion $\|d\varphi\| + \|d{*}\varphi\|$ bounded by $3.71 \times 10^{-4}$ in a sampled $C^0$-style norm on the computational domain --- indicating a small-torsion regime in the heuristic sense of Joyce-style perturbation arguments --- condition number $\kappa = 1.0152$, and matches 77 target period integrals at 5 scales with RMS error $3.1 \times 10^{-4}$. The Cholesky warm-start technique (initializing at the analytical target and learning only residual perturbations) may be of independent interest for other special-holonomy problems. All code, data, and the trained checkpoint are publicly available.

In Part III (Stages 9--12), we extend the analysis to landscape cartography (unique basin, Hessian condition ${\sim}92{,}000$), determinant gauge invariance ($\det(g)$ pure gauge to $8.4 \times 10^{-15}$) with $|\varphi|^2 = 42 = 7 \times 3!$, the full transverse spectrum (117{,}648 modes, 8{,}872 unique levels, Weyl law at 97.6\%), and $\Gtwo$ Yukawa selection rules ($n_1 \pm n_2 \pm n_3 = 0$, $9/56$ channels allowed, $|Y| = 0.5923 = 1/\sqrt{2V}$) with $\Gtwo$ decomposition confirming $Y(\Omega^2_7 \times \Omega^2_7 \times \Omega^3_7) = 0$.
\end{abstract}

\vfill
\noindent\rule{\textwidth}{0.2pt}

\newpage
\tableofcontents

\newpage

% ============================================
\section{Introduction}\label{sec:intro}
% ============================================

\subsection{Compact manifolds with holonomy contained in $\Gtwo$}

A compact Riemannian 7-manifold $(M^7, g)$ has holonomy contained in the exceptional Lie group $\Gtwo \subset \SO(7)$ if and only if it admits a torsion-free $\Gtwo$-structure, i.e., a closed and coclosed 3-form $\varphi \in \Omega^3(M)$ \cite{harvey1982, bryant1987}. (Full holonomy $\Gtwo$, as opposed to a proper subgroup, requires additionally that $M$ be simply connected and not a Riemannian product.) Joyce \cite{joyce1996, joyce2000} proved the existence of compact examples by resolving singularities of $T^7/\Gamma$ orbifolds. Kovalev \cite{kovalev2003} introduced the twisted connected sum (TCS) construction, gluing two asymptotically cylindrical (ACyl) Calabi--Yau threefolds along a common K3 fiber. Corti, Haskins, Nordstr\"om and Pacini \cite{chnp2015} systematized the TCS method and produced many topological types.

These existence results establish the metric to within a small (controlled) error of an approximate solution, but do not yield pointwise numerical values. To our knowledge, no explicit metric tensor $g_{ij}(x)$ has been computed numerically for a compact $\Gtwo$ manifold, though we note that substantial numerical work exists for \textit{non-compact} examples (see e.g.\ Brandhuber et al.\ \cite{brandhuber2001}).

\subsection{The PINN approach}

Physics-informed neural networks (PINNs) \cite{raissi2019} parameterize solutions to PDEs via neural networks whose loss function encodes the governing equations. They have been successfully applied to fluid dynamics \cite{cai2021}, quantum mechanics \cite{hermann2020}, and general relativity \cite{liao2023}, but not, to our knowledge, to special holonomy geometry.

We apply PINNs to construct a candidate metric on a local model of the neck region of $\Kseven$, a compact TCS manifold with $b_2 = 21$ and $b_3 = 77$ (the specific topological type studied in \cite{braun2018}). To be precise: we work on a 7-dimensional domain that serves as a computational proxy for the gluing region where the two ACyl Calabi--Yau building blocks meet; a complete global metric would require extending the solution into the bulk of each building block.

The key technical contribution is a \textbf{Cholesky parameterization with analytical warm-start}: the network outputs a lower-triangular perturbation $\delta L(x)$, and the metric is $g(x) = (L_0 + \delta L(x))(L_0 + \delta L(x))^\top$, where $L_0$ is the Cholesky factor of an analytically derived target. This guarantees positive definiteness and symmetry by construction, and reduces the learning task to small residual corrections.

\subsection{Motivation from the GIFT framework}

The analytical target and the period integrals used as training data derive from the GIFT (Geometric Information Field Theory) framework (see Related Works, p.~\pageref{sec:related}), which proposes that physical constants arise from the topology of $\E_8 \times \E_8$ compactifications on $\Gtwo$ manifolds. While the physical claims of GIFT are outside the scope of this paper, the mathematical objects it produces (the $\Gtwo$ decomposition, the Mayer--Vietoris splitting of moduli, and the determinant formula $\det(g) = 65/32$) are independently verifiable statements in differential geometry. We use them as input data and verify the output against standard geometric criteria.

\subsection{Summary of results}

\begin{table}[H]
\centering
\begin{tabular}{lll}
\toprule
Criterion & Target & Achieved \\
\midrule
$\det(g) = 65/32$ & 2.03125 & 2.031250001 ($4 \times 10^{-8}\,\%$) \\
Positive definite & All $\lambda_i > 0$ & $\lambda_{\min} = 1.099$ (Cholesky guarantee) \\
Condition number & 1.01518 & 1.01518 (7 significant figures) \\
Torsion $\|d\varphi\|+\|d{*}\varphi\|$ & small & $3.71 \times 10^{-4}$ (sampled $C^0$-style norm) \\
Period integrals & $\RMS < 0.005$ & 0.000311 (16-fold below threshold) \\
Anisotropy & $\|g - G_{\mathrm{TARGET}}\|_F \to 0$ & $1.76 \times 10^{-7}$ (machine precision) \\
\bottomrule
\end{tabular}
\end{table}

Training time: 2.9 minutes on a single A100 GPU. Model: 202,857 parameters.

\subsection{Outline}

Section~\ref{sec:g2} recalls the $\Gtwo$ structure and the TCS construction. Section~\ref{sec:target} describes the analytical derivation of the target metric. Section~\ref{sec:pinn} presents the PINN architecture and training. Section~\ref{sec:metric} gives the explicit metric and verification results. Section~\ref{sec:discussion} discusses lessons learned, limitations, and future directions.

% ============================================
\section{The $\Gtwo$ Structure and TCS Construction}\label{sec:g2}
% ============================================

\subsection{Holonomy contained in $\Gtwo$ and the associative 3-form}

The exceptional Lie group $\Gtwo$ is the automorphism group of the octonion algebra $\mathbb{O}$. It acts on $\mathrm{Im}(\mathbb{O}) \cong \mathbb{R}^7$ and preserves the standard associative 3-form \cite{harvey1982}:
$$
\varphi_0 = e^{012} + e^{034} + e^{056} + e^{135} - e^{146} - e^{236} - e^{245}
$$
where $e^{ijk} = e^i \wedge e^j \wedge e^k$ and the indices correspond to the 7 imaginary octonion units. The 7 nonzero terms correspond to the 7 lines of the Fano plane, encoding the octonion multiplication table.

Under $\Gtwo$, the space of 3-forms decomposes as:
$$
\Lambda^3(\mathbb{R}^7) = \Lambda^3_1 \oplus \Lambda^3_7 \oplus \Lambda^3_{27}
$$
with dimensions $1 + 7 + 27 = 35 = \binom{7}{3}$. The $\Gtwo$ metric is recovered from the 3-form via the remarkable formula \cite{bryant1987}:
$$
g_{ij}\,\mathrm{vol}_g = \frac{1}{6}\,\iota_{e_i}\varphi \wedge \iota_{e_j}\varphi \wedge \varphi
$$
In an orthonormal frame adapted to $\varphi_0$, the contraction identity $\varphi_{ikl}\,\varphi_{jkl} = 6\,\delta_{ij}$ holds; in general coordinates, recovering $g$ from $\varphi$ requires the full nonlinear algebraic construction of Bryant~\cite{bryant1987}. For the standard $\varphi_0$, the metric is $g = I_7$. A rescaled form $\varphi = c \cdot \varphi_0$ with $c = (65/32)^{1/14}$ yields $g = c^2 \cdot I_7$ with $\det(g) = c^{14} = 65/32$.

\subsection{The TCS construction}

The manifold $\Kseven$ is constructed as a twisted connected sum \cite{kovalev2003, chnp2015}:
$$
\Kseven = M_1 \cup_\Phi M_2
$$
where $M_1$ and $M_2$ are asymptotically cylindrical Calabi--Yau threefolds, glued along their common asymptotic cross-section $S^1 \times \mathrm{K3}$:

\begin{table}[H]
\centering
\begin{tabular}{llcc}
\toprule
Building block & Construction (toy model) & $b_2$ & $b_3$ \\
\midrule
$M_1$ & ACyl CY (topology proxy) & 11 & 40 \\
$M_2$ & ACyl CY (topology proxy) & 10 & 37 \\
K3 (gluing) & K3 surface, $b_2 = 22$ & N/A & N/A \\
\bottomrule
\end{tabular}
\end{table}

\textit{Note:} The building blocks are used here as a toy model for topology bookkeeping (fixing $b_2$ and $b_3$). We do not claim these correspond to a specific algebraic realization in the CHNP classification \cite{chnp2015}; a rigorous matching of semi-Fano data would require additional work.

The Mayer--Vietoris sequence gives:
$$
b_2(\Kseven) = b_2(M_1) + b_2(M_2) = 11 + 10 = 21
$$
$$
b_3(\Kseven) = b_3(M_1) + b_3(M_2) = 40 + 37 = 77
$$

Since $\Kseven$ is a compact orientable manifold of odd dimension, Poincar\'e duality ($b_k = b_{7-k}$) implies $\chi(\Kseven) = 0$. Explicitly: $b_0 = b_7 = 1$, $b_1 = b_6 = 0$, $b_2 = b_5 = 21$, $b_3 = b_4 = 77$, giving $\chi = 1 - 0 + 21 - 77 + 77 - 21 + 0 - 1 = 0$.

\subsection{Pointwise representation theory}

At each point of a 7-manifold with $\Gtwo$-structure, the space of 3-forms decomposes under $\Gtwo$ as (cf.~\S\ref{sec:g2}.1):
$$
\Lambda^3(\mathbb{R}^7) = \Lambda^3_1 \oplus \Lambda^3_7 \oplus \Lambda^3_{27},
\qquad 1 + 7 + 27 = 35 = \binom{7}{3}.
$$

This is a \textit{pointwise} statement in representation theory: at each point $x \in \Kseven$, a 3-form has 35 components that transform in these three irreducible $\Gtwo$-representations. Among the 35 directions, the 7 that are aligned with the Fano-plane triples of the octonion multiplication table generate volume-changing deformations ($\Tr(\partial g/\partial\Pi) = \pm 2.10$), while the remaining 28 in $\Lambda^3_{27}$ are traceless (pure shape deformations). The vanishing trace for non-Fano modes is exact, following from the orthogonality of $\Lambda^3_{27}$ to the trivial representation $\Lambda^3_1$.

\subsection{Global moduli space}

The moduli space of torsion-free $\Gtwo$ structures on $\Kseven$ is a smooth manifold of dimension $b_3(\Kseven) = 77$ \cite{joyce1996, joyce2000}. This is a \textit{global topological} statement, independent of the pointwise decomposition above. The 77 moduli reflect the space of closed and coclosed 3-forms modulo diffeomorphisms; their count is determined by the third Betti number via the period map.

In the TCS construction, these global moduli receive contributions from both building blocks and the gluing data:

\begin{table}[H]
\centering
\begin{tabular}{ll}
\toprule
Contribution & Source \\
\midrule
$H^3(M_1)$ & 40 classes from the first ACyl CY threefold \\
$H^3(M_2)$ & 37 classes from the second ACyl CY threefold \\
\textbf{Total} & $\mathbf{b_3(\Kseven) = 77}$ \\
\bottomrule
\end{tabular}
\end{table}

% ============================================
\section{The Analytical Target Metric}\label{sec:target}
% ============================================

\subsection{Period integrals}

Each modulus $\Pi_k$ ($k = 1, \ldots, 77$) corresponds to a period integral of the associative 3-form over a 3-cycle $C_k \in H_3(\Kseven, \mathbb{Z})$:
$$
\Pi_k = \int_{C_k} \varphi
$$

We use period data derived from the GIFT framework (see Related Works, p.~\pageref{sec:related}), where the 77 periods are computed from prime-number data at multiple energy scales $T$. The specific values are determined by the torsion coupling constant $\kappa_T = 1/61$ and an adaptive cutoff function $X(T)$ described in a companion preprint (in preparation).

\subsection{The metric Jacobian}

The metric response to moduli variations is given by the Jacobian:
$$
\frac{\partial g_{ij}}{\partial \Pi_k}
= \frac{1}{3}\sum_l \left(\varphi_{ikl}\,\frac{\partial\varphi_{jkl}}{\partial\Pi_k}
+ \frac{\partial\varphi_{ikl}}{\partial\Pi_k}\,\varphi_{jkl}\right)
$$

Evaluating this for the 35 pointwise modes (\S\ref{sec:g2}.3), the 7 modes aligned with the Fano-plane triples have $\Tr(\partial g/\partial\Pi) = \pm 2.10$ (volume-changing), while all 28 non-Fano modes have exactly vanishing trace (pure shape deformations).

\subsection{The target metric $G_{\mathrm{TARGET}}$}

Evaluating the metric Jacobian at the reference periods yields a $7\times 7$ target metric with the following properties:

\begin{table}[H]
\centering
\begin{tabular}{ll}
\toprule
Property & Value \\
\midrule
Diagonal range & $[1.1022,\; 1.1133]$ \\
Max off-diagonal & $0.00461\;(g_{23})$ \\
Condition number $\kappa$ & 1.01518 \\
Determinant (after rescaling) & $65/32 = 2.03125$ \\
Eigenvalue range & $[1.0993,\; 1.1160]$ \\
\bottomrule
\end{tabular}
\end{table}

The anisotropy is small (${\sim}1.5\%$ diagonal variation) but structurally significant: it encodes the breaking of the isotropic $\Gtwo$ structure by the TCS gluing map $\Phi$.

\subsection{The $\E_8$/K3 lattice structure}

The global modes are organized by the K3 lattice $\Lambda_{\mathrm{K3}}$ of signature $(3, 19)$ and rank 22, which contains two sublattices:

\begin{itemize}
\item $N_1$ of rank 11, signature $(1, 9)$: the polarization lattice of $M_1$
\item $N_2$ of rank 10, signature $(1, 8)$: the polarization lattice of $M_2$
\end{itemize}

with $N_1 \cap N_2 = \{0\}$ and $\rk(N_1 + N_2) = 21 = b_2(\Kseven)$. The K3 intersection form is $\Lambda_{\mathrm{K3}} = 3H \oplus 2(-\E_8)$, where $H$ is the hyperbolic lattice and $\E_8$ is the positive-definite $\E_8$ root lattice. The presence of $\E_8$ in the gluing data constrains the global moduli and connects the metric to exceptional Lie algebra structure.

% ============================================
\section{PINN Architecture and Training}\label{sec:pinn}
% ============================================

\subsection{The parameterization challenge}

The goal is to find a spatially varying metric field $g : \Kseven \to \Sym^+_7(\mathbb{R})$ satisfying simultaneously:

\begin{enumerate}
\item $\det(g(x)) = 65/32$ at every point
\item $g(x) > 0$ (positive definite)
\item $d\varphi \approx 0$ and $d{*}\varphi \approx 0$ (torsion-free, where $\varphi$ is reconstructed from $g$)
\item $\int_{C_k} \varphi = \Pi_k$ for $k = 1, \ldots, 77$ at multiple scales
\item Spatial average $\langle g \rangle \approx G_{\mathrm{TARGET}}$
\end{enumerate}

This is a PDE-constrained optimization problem on a 7-dimensional computational domain modelling the TCS neck region (cf.~\S\ref{sec:intro}.2).

\subsection{Failed approaches and lessons}

Before describing the successful architecture, we briefly document two failed approaches, as the failure modes are instructive.

\textbf{Attempt 1 ($\Gtwo$ adjoint parameterization):} A network outputs 14 parameters in the $\Gtwo$ Lie algebra, which are exponentiated to produce a $\Gtwo$ rotation, applied to $\varphi_0$ via Lie derivatives to generate a deformed 3-form, from which the metric is extracted. \textit{Failure mode:} the $14 \to 35$ map via Lie derivatives has rank 6, creating a 6-dimensional bottleneck in the 28-dimensional space of symmetric metric perturbations. The network cannot access 22 of the 28 metric degrees of freedom.

\textbf{Attempt 2 (Anisotropy loss):} Same architecture as above with an additional loss $\|\langle g \rangle - G_{\mathrm{TARGET}}\|^2_F$. \textit{Failure mode:} 97.6\% of the loss gradient comes from the anisotropy term, but the rank-6 bottleneck prevents the network from responding. The loss plateaus after ${\sim}100$ steps and remains constant for the remaining 4,900.

\textbf{Lesson:} When the architecture fundamentally cannot represent the target (rank deficiency), no amount of training or hyperparameter tuning will help. The bottleneck must be removed at the architectural level.

\subsection{The Cholesky parameterization (successful)}

We parameterize the metric directly via a Cholesky decomposition:
$$
g(x) = L(x) \cdot L(x)^\top, \qquad L(x) = L_0 + \delta L(x)
$$
where $L_0 = \chol(G_{\mathrm{TARGET}})$ is the Cholesky factor of the analytical target, and $\delta L(x)$ is a lower-triangular matrix output by the network.

\begin{table}[H]
\centering
\small
\begin{tabular}{lll}
\toprule
Property & $\Gtwo$ adjoint & Cholesky (this work) \\
\midrule
Metric DOF per point & 6 (rank of Lie derivs) & \textbf{28} (full) \\
Initialization & $c^2 \cdot I_7$ (far from target) & $\mathbf{G_{\mathrm{TARGET}}}$ (at target) \\
Positive definiteness & Requires penalty loss & \textbf{Free} ($LL^\top \geq 0$) \\
Symmetry & Via einsum contraction & \textbf{Free} ($LL^\top = (LL^\top)^\top$) \\
Gradient path & MLP $\to$ adj $\to$ Lie $\to$ $\varphi$ $\to$ $g$ & MLP $\to$ $\delta L$ $\to$ $g$ \\
\bottomrule
\end{tabular}
\end{table}

\textbf{Network architecture:}

\begin{lstlisting}
Input: (x^1, ..., x^7, log T) in R^8
  |
FourierFeatures(48 frequencies) -> R^96
  |
MLP: 96 -> 256 -> 256 -> 256 -> 128 (ReLU activations)
  |
+-- Metric head: 128 -> 28 (lower triangular dL)
|     g(x) = (L_0 + dL(x))(L_0 + dL(x))^T
|
+-- 3-form heads: 128 -> 35 (local) + 42 (global)
      phi(x) = c * phi_0 + 0.1 * d_phi(x)
\end{lstlisting}

Total parameters: 202,857.

\subsection{Loss function}

The loss has five terms:

\begin{table}[H]
\centering
\small
\begin{tabular}{llcl}
\toprule
Term & Formula & Weight & Purpose \\
\midrule
$\mathcal{L}_{\det}$ & $(\det(g) - 65/32)^2$ & 100 & Topological constraint \\
$\mathcal{L}_{\mathrm{aniso}}$ & $\|\langle g \rangle - G_{\mathrm{TARGET}}\|^2_F$ & 500 & Analytical target \\
$\mathcal{L}_{\mathrm{period}}$ & $\sum_T \|\langle\delta\varphi\rangle_T - \Pi(T)\|^2 / 5$ & 1000 & 77 periods $\times$ 5 scales \\
$\mathcal{L}_{\mathrm{torsion}}$ & $\|d\varphi\|^2 + \|d{*}\varphi\|^2$ (finite diff.) & 1 & Torsion-free condition \\
$\mathcal{L}_{\mathrm{sparse}}$ & $\|\delta L\|^2$ & 0.01 & Regularization \\
\bottomrule
\end{tabular}
\end{table}

The period loss averages over 5 energy scales ($T = 100,\; 1000,\; 10000,\; 40000,\; 75000$), each activating a different number of effective moduli (from 5 to all 77).

\subsection{Training protocol}

Training proceeds in two phases over 5,000 epochs on a single NVIDIA A100-SXM4-80GB GPU:

\textbf{Phase 1 (epochs 0--2,500):} Learning rate $10^{-3}$ with cosine annealing. The warm-start means the determinant and anisotropy losses are already near zero at initialization; the network primarily learns the period integrals and torsion structure.

\textbf{Phase 2 (epochs 2,500--5,000):} Learning rate $10^{-4}$. Fine-tuning. By epoch 3,500, the determinant and anisotropy losses reach machine precision ($10^{-15}$ to $10^{-18}$), and the residual loss is dominated entirely by the period term.

\textbf{Training dynamics:}

\begin{table}[H]
\centering
\small
\begin{tabular}{lccccc}
\toprule
Epoch & Total loss & $\mathcal{L}_{\det}$ & $\mathcal{L}_{\mathrm{aniso}}$ & $\mathcal{L}_{\mathrm{period}}$ & $\mathcal{L}_{\mathrm{torsion}}$ \\
\midrule
0     & $4.33\times10^{-3}$ & $2.7\times10^{-21}$ & $9.8\times10^{-25}$ & $4.3\times10^{-6}$ & $3.6\times10^{-23}$ \\
100   & $1.51\times10^{-3}$ & $4.9\times10^{-6}$  & $3.2\times10^{-7}$  & $8.6\times10^{-7}$ & $9.5\times10^{-10}$ \\
500   & $6.28\times10^{-4}$ & $2.0\times10^{-6}$  & $8.3\times10^{-8}$  & $3.9\times10^{-7}$ & $2.6\times10^{-10}$ \\
2000  & $4.37\times10^{-4}$ & $3.8\times10^{-7}$  & $1.7\times10^{-8}$  & $3.9\times10^{-7}$ & $5.4\times10^{-11}$ \\
3500  & $3.91\times10^{-4}$ & $1.1\times10^{-17}$ & $8.9\times10^{-15}$ & $3.9\times10^{-7}$ & $1.1\times10^{-11}$ \\
5000  & $3.91\times10^{-4}$ & $3.8\times10^{-18}$ & $2.9\times10^{-15}$ & $3.9\times10^{-7}$ & $1.1\times10^{-11}$ \\
\bottomrule
\end{tabular}
\end{table}

At convergence, 100\% of the residual loss is from the period integrals. The metric constraints (determinant, anisotropy, positive definiteness) are satisfied to machine precision.

Total training time: \textbf{2.9 minutes}.

% ============================================
\section{The Explicit Metric}\label{sec:metric}
% ============================================

\subsection{The $7\times 7$ metric tensor}

The spatially averaged metric over 50,000 points on the TCS neck:

$$
\bar{g} = \begin{pmatrix}
 1.11332 & \!+0.00098 & \!-0.00072 & \!-0.00019 & \!+0.00341 & \!+0.00285 & \!-0.00305 \\
+0.00098 &  1.11055 & \!-0.00081 & \!+0.00123 & \!-0.00419 & \!+0.00018 & \!-0.00325 \\
-0.00072 & \!-0.00081 &  1.10908 & \!+0.00461 & \!+0.00085 & \!+0.00269 & \!+0.00069 \\
-0.00019 & \!+0.00123 & \!+0.00461 &  1.10430 & \!-0.00069 & \!+0.00010 & \!-0.00135 \\
+0.00341 & \!-0.00419 & \!+0.00085 & \!-0.00069 &  1.10263 & \!+0.00154 & \!-0.00001 \\
+0.00285 & \!+0.00018 & \!+0.00269 & \!+0.00010 & \!+0.00154 &  1.10385 & \!-0.00066 \\
-0.00305 & \!-0.00325 & \!+0.00069 & \!-0.00135 & \!-0.00001 & \!-0.00066 &  1.10217
\end{pmatrix}
$$

\subsection{Comparison with analytical target}

\begin{table}[H]
\centering
\begin{tabular}{llll}
\toprule
Component & Target & Achieved & Absolute error \\
\midrule
$g_{00}$ & 1.113320 & 1.113320 & $1.5 \times 10^{-7}$ \\
$g_{11}$ & 1.110552 & 1.110552 & $1.6 \times 10^{-7}$ \\
$g_{22}$ & 1.109078 & 1.109078 & $2.5 \times 10^{-8}$ \\
$g_{33}$ & 1.104300 & 1.104300 & $2.3 \times 10^{-7}$ \\
$g_{44}$ & 1.102633 & 1.102633 & $1.7 \times 10^{-7}$ \\
$g_{55}$ & 1.103852 & 1.103852 & $1.4 \times 10^{-8}$ \\
$g_{66}$ & 1.102167 & 1.102167 & $2.7 \times 10^{-7}$ \\
$g_{23}$ (max off-diag) & $+0.004613$ & $+0.004613$ & $1.0 \times 10^{-6}$ \\
$\|g - G_{\mathrm{TARGET}}\|_F$ & N/A & N/A & $\mathbf{1.76 \times 10^{-7}}$ \\
\bottomrule
\end{tabular}
\end{table}

Relative error: $4.4 \times 10^{-8}$ (maximum elementwise error / maximum entry).

\subsection{Eigenvalues}

\begin{table}[H]
\centering
\begin{tabular}{lccc}
\toprule
 & Target & Achieved & Error \\
\midrule
$\lambda_1$ & 1.09926643 & 1.09926642 & $1 \times 10^{-8}$ \\
$\lambda_2$ & 1.10004584 & 1.10004584 & $< 10^{-8}$ \\
$\lambda_3$ & 1.10124313 & 1.10124311 & $2 \times 10^{-8}$ \\
$\lambda_4$ & 1.10334338 & 1.10334338 & $< 10^{-8}$ \\
$\lambda_5$ & 1.11246355 & 1.11246359 & $4 \times 10^{-8}$ \\
$\lambda_6$ & 1.11358841 & 1.11358840 & $1 \times 10^{-8}$ \\
$\lambda_7$ & 1.11595127 & 1.11595127 & $< 10^{-8}$ \\
\bottomrule
\end{tabular}
\end{table}

All seven eigenvalues matched to \textbf{8 significant figures}.

\subsection{Determinant}

$$
\det(g) = 2.031250001 \pm 9.5 \times 10^{-9}
$$
$$
\text{Target:}\; 65/32 = 2.031250000, \qquad
\text{Deviation:}\; 4 \times 10^{-8}\,\%
$$

\subsection{Torsion}

The torsion of a $\Gtwo$-structure $\varphi$ is measured by the failure of $\varphi$ to be closed and coclosed. Following Joyce \cite{joyce2000}, if a compact 7-manifold admits a $\Gtwo$-structure $\varphi_0$ with $\|d\varphi_0\|_{C^0} + \|d{*}\varphi_0\|_{C^0}$ sufficiently small (below a constant $\varepsilon_0$ depending on the geometry), then there exists a nearby torsion-free $\Gtwo$-structure $\tilde{\varphi}$ with $\Hol(\tilde{g}) \subseteq \Gtwo$.

We evaluate the torsion of our candidate using finite-difference approximations of $d\varphi$ and $d{*}\varphi$. Two evaluations are reported:

\begin{table}[H]
\centering
\begin{tabular}{llll}
\toprule
Evaluation scope & Mean $\|d\varphi\| + \|d{*}\varphi\|$ & Max $\|d\varphi\| + \|d{*}\varphi\|$ & Points \\
\midrule
Neck region (v3) & $3.3 \times 10^{-6}$ & $7.2 \times 10^{-6}$ & 50,000 \\
Global (v3.2, 2000 samples) & $8.58 \times 10^{-5}$ & $3.71 \times 10^{-4}$ & 2,000 \\
\bottomrule
\end{tabular}
\end{table}

The global evaluation (v3.2) covers the full computational domain including regions outside the neck where torsion is larger. Even the worst-case global bound of $3.71 \times 10^{-4}$ is well within Joyce's perturbative regime.

\textbf{Heuristic comparison with Joyce-type perturbation arguments.} Joyce's Theorem~11.6.1 \cite{joyce2000} guarantees the existence of a nearby torsion-free $\Gtwo$-structure whenever the initial torsion lies below a threshold $\varepsilon_0$ depending on the background geometry and the analytic setup (H\"older norms, elliptic estimates, etc.). We report a max torsion residual of $3.71 \times 10^{-4}$ in our sampled $C^0$-style norm on the computational domain; this indicates a small-torsion regime in the heuristic sense used in Joyce-style perturbation arguments, but we do not claim to have verified the full analytic hypotheses.

A Lean script certifies the arithmetic comparison between the recorded numerical bound ($3710/10000000$) and a chosen benchmark ($1/10$):

\begin{lstlisting}[language={}]
namespace K7Certificate
def torsion_bound : Q := 3710 / 10000000   -- 3.71 x 10^{-4}
def benchmark      : Q := 1 / 10            -- 0.1 (chosen reference)
theorem bound_lt_benchmark : torsion_bound < benchmark := by native_decide
end K7Certificate
\end{lstlisting}

This certifies only the arithmetic inequality; it does not certify the analytic hypotheses of Joyce's deformation theorem (H\"older regularity, global elliptic estimates, etc.).

We emphasize that our computation covers a computational proxy for the neck region, not the full compact manifold. Extension to the bulk remains an open problem (see \S\ref{sec:discussion}.4).

\subsection{Scale invariance}

The metric is evaluated at five energy scales $T$, at which different numbers of moduli are active:

\begin{table}[H]
\centering
\begin{tabular}{lccl}
\toprule
Scale $T$ & $\det$ deviation & Condition $\kappa$ & Active moduli \\
\midrule
100     & $3.7 \times 10^{-8}\,\%$ & 1.0151782 & 5  \\
1,000   & $4.5 \times 10^{-8}\,\%$ & 1.0151782 & 66 \\
10,000  & $4.0 \times 10^{-8}\,\%$ & 1.0151782 & 77 \\
40,000  & $3.9 \times 10^{-8}\,\%$ & 1.0151782 & 77 \\
75,000  & $4.6 \times 10^{-8}\,\%$ & 1.0151782 & 77 \\
\bottomrule
\end{tabular}
\end{table}

The condition number is \textbf{identical to 7 significant figures} at every scale. The metric structure is independent of the scale at which the period data is supplied.

\subsection{Period integrals}

\begin{table}[H]
\centering
\begin{tabular}{llll}
\toprule
Scale $T$ & RMS error & Correlation (local) & Active modes \\
\midrule
100     & 0.00110           & 0.920 & 5  \\
1,000   & 0.000358          & 0.999 & 66 \\
\textbf{10,000}  & \textbf{0.000311} & \textbf{0.996} & \textbf{77} \\
40,000  & 0.000479          & 0.995 & 77 \\
75,000  & 0.000540          & 0.995 & 77 \\
\bottomrule
\end{tabular}
\end{table}

Best fit at $T = 10{,}000$ ($\RMS = 3.11 \times 10^{-4}$, 16-fold below threshold).

\subsection{Landscape cartography}\label{sec:landscape}

The 4-parameter optimization raises the question: is the optimum unique? A systematic landscape exploration (287 evaluations) addresses this through six phases: LHS screening (120 random starts, 94/120 non-SPD --- the admissible domain is a small island in 4D), 1D sensitivity profiling, 2D grid scans, Powell refinement, and Sobol sensitivity analysis.

\begin{table}[H]
\centering
\begin{tabular}{lc}
\toprule
Quantity & Value \\
\midrule
Total evaluations & 287 \\
SPD-admissible fraction & 22\% \\
Basin count (6 starts) & 1 deep (unique) \\
Hessian condition number & 92{,}392 \\
Powell-refined $\nabla\varphi$ & $8.462 \times 10^{-4}$ (identical to \S\ref{sec:metric}.6) \\
\bottomrule
\end{tabular}
\end{table}

The Hessian eigenvalue analysis reveals extreme anisotropy: $\varepsilon_f$ has curvature $H = 155.6$, $\varepsilon_k$ has $H = 41.2$ (both with basin width below resolution), while $\log(a_f)$ and $\log(a_t)$ have $H \approx 0.004$ (basin widths ${\sim}2.5$ and ${\sim}2.0$ respectively). The condition number $\kappa = 92{,}392$ means perturbations in the $\varepsilon$ directions are catastrophic while the log-scale parameters are gentle.

Sobol sensitivity indices: $\varepsilon_k$ (0.26) $>$ $\varepsilon_f$ (0.24) $>$ $\log(a_t)$ (0.15) $>$ $\log(a_f)$ (0.05). The most sensitive parameters have optimal values near zero --- anisotropy is lethal, not useful. The optimum is the \textbf{unique global minimum} of the landscape.

\subsection{Determinant gauge invariance and $|\varphi|^2 = 42$}\label{sec:detgauge}

An observation during landscape exploration (\S\ref{sec:landscape}) suggested that $\det(g) = 1.5$ gives 11\% lower $\nabla\varphi_{\mathrm{code}}$ than the canonical $65/32$. We test whether this is a genuine improvement or a scale artifact.

\textbf{Theoretical prediction.} Under a global rescaling $g \to \alpha \cdot g$, the code-reported torsion scales as $\nabla\varphi_{\mathrm{code}} \propto \det^{3/7}$ while the proper (coordinate-invariant) torsion scales as $\nabla\varphi_{\mathrm{proper}} \propto \det^{-1/7}$.

\textbf{Phase 1} (no re-optimization, 10 $\det$ values spanning $12\times$ range): $\nabla\varphi_{\mathrm{code}} \propto \det^{0.428571}$ (predicted $3/7 = 0.428571$) --- \textbf{exact to $8.4 \times 10^{-15}$}. All proper-torsion ratios equal 1.000000 across the full range.

\textbf{Phase 2} (with re-optimization, 8 $\det$ values, 200 steps each): $\nabla\varphi_{\mathrm{code}} / \det^{3/7} = \text{const}$ with CV $= 0.0013\%$. The optimizer finds identical geometry at every $\det$ value.

\textbf{Phase 3} (spectral invariants): Eigenvalue ratios $1:4:9:16:25$ preserved at all $\det$ values.

\textbf{Conclusion:} $\det(g)$ is a \textbf{pure gauge parameter} with no physical content. The apparent 11\% improvement was exactly $(1.5/2.031)^{3/7} = 0.878$.

\textbf{Bonus:} $|\varphi|^2_{\mathrm{proper}} = 42.000 = 7 \times 3!$. The proper norm of the associative 3-form, computed from the metric-corrected volume form, gives an exact topological invariant.

\subsection{Transverse spectrum and eigenfunctions}\label{sec:spectrum}

The 1D metric $g(t)$ defines a longitudinal Sturm--Liouville problem. The transverse directions (6D fiber perpendicular to the seam) define independent eigenproblems whose spectrum governs the transition from 1D to fully 7D physics.

\textbf{Transverse metric profile.} The $6\times 6$ transverse metric $g_\perp$ has two groups of eigenvalues: fiber ($\theta$, $\psi$): $g^{-1}_\perp = 0.687$ (2 near-degenerate, CV $< 0.003\%$) and K3 (4 directions): $g^{-1}_\perp = 1.32$ (4 eigenvalues, CV $< 0.06\%$).

\textbf{Full product spectrum.} From the flat-torus transverse Laplacian with lattice vectors $m \in \mathbb{Z}^6$ ($|m|_\infty \leq 3$): 117{,}648 total modes, 8{,}872 unique eigenvalue levels.

\begin{table}[H]
\centering
\begin{tabular}{cccc}
\toprule
Level & $\lambda_\perp$ & Degeneracy & Content \\
\midrule
1 & 27.14 & 4 & Pure fiber ($\theta$, $\psi$) \\
2 & 52.36 & 2 & K3$_4$ \\
3 & 52.39 & 4 & K3$_3$ \\
4 & 52.41 & 2 & K3$_1$ \\
5 & 54.28 & 4 & Mixed fiber \\
\bottomrule
\end{tabular}
\end{table}

Scale hierarchy: $\lambda_1^{\perp} / \lambda_1^{\mathrm{long}} = 8.19\times$ at $L = 1$. The critical crossing length is $L_{\mathrm{cross}} = 0.35$, where the first longitudinal and first transverse eigenvalue coincide.

\textbf{Weyl's law.} The seam volume $\int\!\sqrt{\det}\,dt = 1.4252$. At $\lambda = 100$: $N_{\mathrm{actual}} = 170$ vs $N_{\mathrm{Weyl}} = 174$, ratio $= 0.976$ (97.6\%).

\textbf{Sturm--Liouville eigenfunctions.} Eigenvalue ratios $1:4:9:16:25$ (exact). Zero-mode eigenvalue $\lambda_0 = 6 \times 10^{-13}$ (numerical zero). Orthonormality error: $\max$ off-diagonal $= 1.86 \times 10^{-10}$. Deviation from cosines: $\|\psi_n - \sqrt{2}\cos(n\pi s)\|_2 < 4 \times 10^{-6}$.

\subsection{Yukawa selection rules}\label{sec:yukawa}

The Sturm--Liouville eigenfunctions determine the longitudinal Yukawa triple-overlap integrals $Y_{n_1,n_2,n_3} = \int \psi_{n_1}\,\psi_{n_2}\,\psi_{n_3}\,\sqrt{\det}\,dt$.

\textbf{Selection rule:} $Y_{n_1,n_2,n_3} \neq 0$ if and only if $n_1 \pm n_2 \pm n_3 = 0$ for some sign combination.

\begin{table}[H]
\centering
\begin{tabular}{lc}
\toprule
Metric & Value \\
\midrule
Allowed triples (first 6 modes) & 9 / 56 \\
Universal coupling $|Y|$ & $0.5923 = 1/\sqrt{2V}$ \\
Rule-violating residuals & ${\sim}10^{-7}$ (6 orders below allowed) \\
Full 7D triples (with transverse) & 200 valid \\
\bottomrule
\end{tabular}
\end{table}

The coupling is \textbf{universal}: all 9 allowed triples have identical $|Y| = 0.5923$, where $V = 1.4252$ is the seam volume. Metric-corrected universality: $|{}\tilde{y}| = 1.1938$ with CV $= 0.0001\%$ and $\max/\min = 1.000003$.

\subsection{$\Gtwo$ decomposition and cup product Yukawas}\label{sec:g2decomp}

Moving beyond scalar Yukawas, we compute the algebraic cup product $Y_{abI} = \int \omega_a \wedge \omega_b \wedge \psi_I$ over the torus $T^7$, where $\omega_a \in \Omega^2$ and $\psi_I \in \Omega^3$, and decompose by $\Gtwo$ irreducible representations.

\textbf{$\Gtwo$ decomposition of forms.} The Hodge-star composed with $\varphi$-wedge acts on $\Omega^2$ with eigenvalues $+2$ ($\times 7$) and $-1$ ($\times 14$):
$$\Omega^2 = \Omega^2_7 \oplus \Omega^2_{14} \qquad (7 + 14 = 21 = b_2)$$
$$\Omega^3 = \Omega^3_1 \oplus \Omega^3_7 \oplus \Omega^3_{27} \qquad (1 + 7 + 27 = 35)$$

Projector validation: $P_7^2 = P_7$ (error 0), $P_{14}^2 = P_{14}$ (error $4 \times 10^{-16}$), $P_7 + P_{14} = I$ (error 0). The norm $\|\varphi\|^2 = 7$.

\textbf{Cup product on irreps.}

\begin{table}[H]
\centering
\begin{tabular}{lcc}
\toprule
Channel & Nonzero fraction & Max $|Y|$ \\
\midrule
$\Omega^2_7 \times \Omega^2_7 \times \Omega^3_7$ & \textbf{0/343 (= 0)} & 0 \\
$\Omega^2_7 \times \Omega^2_7 \times \Omega^3_1$ & significant & 0.756 \\
$\Omega^2_7 \times \Omega^2_7 \times \Omega^3_{27}$ & 1299/5145 (25\%) & 0.545 \\
$\Omega^2_{14} \times \Omega^2_{14} \times \Omega^3$ & \textbf{6860/6860 (100\%)} & dense \\
\bottomrule
\end{tabular}
\end{table}

The key result is the \textbf{$\Gtwo$ selection rule}:
$$\boxed{Y(\Omega^2_7 \times \Omega^2_7 \times \Omega^3_7) = 0}$$

This vanishing is exact (not numerical) and follows from representation theory: the tensor product $7 \otimes 7$ decomposes as $1 \oplus 7 \oplus 14 \oplus 27$, which does not contain the dual of 7 in the relevant coupling channel.

\textbf{Kovalev twist and orbifold selection.} The Kovalev twist $J$ has order 8, $\det = -1$ (orientation-reversing). $J$-invariant subspaces: $\dim(\Omega^2)^J = 3$, $\dim(\Omega^3)^J = 6$. $J$ mixes $\Omega^2_7$ and $\Omega^2_{14}$ (off-diagonal norm $= 2.31$) --- $J$ is not in $\Gtwo$. The cup product on the $J$-invariant subspace ($3 \times 3 \times 6$) is \textbf{identically zero}. Physical Yukawas originate exclusively from the $J$-anti-invariant sector: $Y(\mathrm{anti}_2 \times \mathrm{anti}_2 \times \mathrm{anti}_3)$ has 14/112 nonzero entries, with $\max|Y| = 1.0$. After mass-matrix normalization, all 210 nonzero entries have $|\tilde{y}| = 0.3326$ with $\max/\min = 1.00$ --- \textbf{universal coupling}.

% ============================================
\section{Discussion}\label{sec:discussion}
% ============================================

\subsection{Summary of contributions}

\begin{enumerate}
\item \textbf{A numerical candidate metric on a compact $\Gtwo$ manifold.} Previous work established existence (Joyce \cite{joyce1996}) and gave constructions (Kovalev \cite{kovalev2003}, Corti--Haskins--Nordstr\"om--Pacini (CHNP) \cite{chnp2015}), but, to our knowledge, explicit pointwise numerical values of $g_{ij}(x)$ have not been reported for the compact case. We note that substantial numerical work exists for non-compact $\Gtwo$ manifolds, and that our result covers only the TCS neck region (see \S\ref{sec:discussion}.3).

\item \textbf{PINNs applied to special holonomy geometry.} The Cholesky warm-start technique may be applicable to other settings where an analytical approximation is available (e.g., $\Spin(7)$ manifolds, Calabi--Yau metrics beyond the K\"ahler class).

\item \textbf{Landscape uniqueness with Sobol sensitivity.} Systematic exploration (287 evaluations) confirms the optimum is the unique global minimum with Hessian condition number $92{,}392$ and extreme anisotropy in the $\varepsilon$ parameters (\S\ref{sec:landscape}).

\item \textbf{$|\varphi|^2 = 42$ topological identity.} The determinant is a pure gauge parameter (verified to $8.4 \times 10^{-15}$), and the proper 3-form norm $|\varphi|^2 = 42 = 7 \times 3!$ is an exact topological invariant (\S\ref{sec:detgauge}).

\item \textbf{Full 7D spectrum with Weyl law.} The product spectrum (117{,}648 modes, 8{,}872 unique levels) satisfies Weyl's law at 97.6\% accuracy, with critical crossing length $L_{\mathrm{cross}} = 0.35$ (\S\ref{sec:spectrum}).

\item \textbf{Yukawa selection rules.} $n_1 \pm n_2 \pm n_3 = 0$, with $9/56$ allowed triples and universal coupling $|Y| = 0.5923 = 1/\sqrt{2V}$. Universality preserved under the full metric (CV $= 0.0001\%$) (\S\ref{sec:yukawa}).

\item \textbf{$\Gtwo$ decomposition with cup product analysis.} $Y(\Omega^2_7 \times \Omega^2_7 \times \Omega^3_7) = 0$ (exact $\Gtwo$ selection rule). All $J$-invariant Yukawas vanish; physical Yukawas arise exclusively from the $J$-anti-invariant sector (\S\ref{sec:g2decomp}).
\end{enumerate}

\subsection{The Cholesky warm-start technique}

The key insight is to decompose the problem:
$$
g(x) = g_{\mathrm{target}} + \delta g(x), \qquad
\delta g \text{ small}
$$
and parameterize via $L(x) = L_0 + \delta L(x)$ where $L_0 = \chol(g_{\mathrm{target}})$. This has three advantages:

\begin{enumerate}
\item \textbf{Guaranteed constraints}: positive definiteness and symmetry are automatic, eliminating two loss terms and simplifying the optimization landscape.
\item \textbf{Warm start}: the network begins at the analytical solution and only needs to learn corrections of order $10^{-7}$, not the full metric from scratch.
\item \textbf{Full rank}: unlike Lie-algebraic parameterizations which may have rank deficiencies (as demonstrated by our earlier attempts), the Cholesky approach has 28 independent degrees of freedom per point (the full dimension of $\Sym_7(\mathbb{R})$).
\end{enumerate}

\subsection{Limitations}

\begin{enumerate}
\item \textbf{Local model, not global}: Our metric is defined on a computational proxy for the TCS neck region. A complete global metric would require extending the solution into the bulk of $M_1$ and $M_2$, where it approaches the known Calabi--Yau metrics. The small torsion residual on the evaluated domain (\S\ref{sec:metric}.5) is encouraging but does not constitute a global verification.

\item \textbf{Period data from GIFT}: The training targets (77 period integrals) are derived from the GIFT framework. While the metric itself is independently verifiable ($\det$, torsion, positive definiteness are geometric properties), the specific values of the periods inherit any limitations of GIFT.

\item \textbf{Determinant value}: The target $\det(g) = 65/32$ is derived within GIFT from the formula $\det(g) = (\dimE(\E_8) + \dimE(\Gtwo) + \rk(\E_8) + \dimE(\Kseven)) / 2^5$. An independent derivation from pure $\Gtwo$ geometry would strengthen the result.

\item \textbf{Neural network representation}: The metric is stored as a trained neural network, not a closed-form expression. While this is standard in the PINN literature, it limits analytical manipulation.
\end{enumerate}

\subsection{Future directions}

\begin{enumerate}
\item \textbf{Extension to the bulk}: Solve the torsion-free equations $d\varphi = 0$, $d{*}\varphi = 0$ as a boundary-value problem, using the neck-region metric as a boundary condition and the known ACyl CY metrics on $M_1$, $M_2$ as asymptotic data.

\item \textbf{Other topological types}: Apply the same pipeline to other TCS manifolds from the CHNP classification, to understand how the metric depends on the topology ($b_2$, $b_3$).

\item \textbf{Spectral geometry on the curved metric}: The spectral analysis (\S\ref{sec:spectrum}) was performed on the warped-product metric. Computing the full Laplacian spectrum on the non-trivially curved metrics of Stage 8 remains open: does the degeneracy fingerprint $[1, 10, 9, 30]$ survive when genuine curvature is present?

\item \textbf{Comparison with flow methods}: Compare the PINN metric with results from Laplacian flow \cite{lotay2019} or Hitchin flow, which provide alternative computational approaches to $\Gtwo$ metrics.

\item \textbf{Geodesic computation}: With an explicit metric now available, geodesic lengths on $\Kseven$ can in principle be computed numerically. A geodesic solver module (\texttt{gift\_core.nn.geodesics}) has been developed with an adapter for the v3.2 checkpoint (guaranteeing positive-definite metrics via the Cholesky parameterization). Whether the resulting geodesic spectrum exhibits any number-theoretic structure (e.g., connections to prime logarithms via Selberg-type trace formulas) remains a speculative open question for future investigation.

\item \textbf{Yukawa hierarchy from resolution forms}: The torus Yukawas are universal ($|Y| = 1/\sqrt{2V}$). Physical Yukawa hierarchy must arise from the 42 resolution 3-forms of $\Kseven$. Computing these requires an explicit resolution of the TCS singularities.

\item \textbf{Fiber-dependent $\varphi(t,\theta)$ via Joyce $\eta$ correction}: The remaining path to attack the 35\% fiber-connection torsion (after bulk optimization reduced it from 71\% to 65\%).
\end{enumerate}

% ============================================
\appendix
\section{Topological Constants}\label{app:constants}
% ============================================

All constants derive from the topology of $\Kseven$ and related algebraic structures. None are fitted.

\begin{table}[H]
\centering
\begin{tabular}{lll}
\toprule
Symbol & Value & Definition \\
\midrule
$\dimE(\Kseven)$ & 7 & Manifold dimension \\
$\dimE(\Gtwo)$ & 14 & Holonomy group dimension \\
$\dimE(\E_8)$ & 248 & Exceptional Lie algebra \\
$b_2(\Kseven)$ & 21 & Second Betti number \\
$b_3(\Kseven)$ & 77 & Third Betti number ($= \dimE$ moduli) \\
$\binom{7}{3}$ & 35 & $\dimE\,\Lambda^3(\mathbb{R}^7)$ (local modes) \\
$\kappa_T$ & $1/61$ & Torsion coupling constant \\
$\det(g)$ & $65/32$ & Metric determinant \\
\bottomrule
\end{tabular}
\end{table}

% ============================================
\section{Reproducibility}\label{app:repro}
% ============================================

\subsection{Code and data}

\begin{table}[H]
\centering
\small
\begin{tabular}{ll}
\toprule
Resource & Location \\
\midrule
PINN notebook (v3) & \texttt{notebooks/K7\_PINN\_Step5\_Reconstruction\_v3.ipynb} \\
Pre-computed data & \texttt{notebooks/riemann/*.json} (Steps 1--4) \\
v3.2 checkpoint & \texttt{notebooks/outputs/k7\_pinn\_step5\_final.pt} (1.6\,MB, float64) \\
v3.2 certification & \texttt{notebooks/outputs/k7\_metric\_v32\_export.json} (20/20 checks) \\
Lean~4 certificate & \texttt{notebooks/outputs/K7Certificate.lean} \\
2000-point sample & \texttt{notebooks/outputs/k7\_metric\_data.csv} \\
Geodesic solver & \texttt{gift\_core/nn/geodesics.py} (includes \texttt{CheckpointPINNAdapter}) \\
Repository & \url{https://github.com/gift-framework/GIFT} \\
\bottomrule
\end{tabular}
\end{table}

The v3.2 checkpoint can be loaded via:

\begin{lstlisting}[language=Python]
import torch
from gift_core.nn import CheckpointPINNAdapter

state = torch.load('k7_pinn_step5_final.pt', map_location='cpu')
model = CheckpointPINNAdapter(state)
x = torch.randn(100, 7)  # 100 points on K7
g = model.metric(x)       # shape (100, 7, 7), guaranteed pos. def.
\end{lstlisting}

\subsection{Hardware}

\begin{table}[H]
\centering
\begin{tabular}{ll}
\toprule
 & Specification \\
\midrule
GPU & NVIDIA A100-SXM4-80GB \\
Training time & 2.9 minutes \\
Parameters & 202,857 \\
Epochs & 5,000 \\
Evaluation points & 50,000 \\
Peak memory & ${\sim}1$--$2$\,GB \\
\bottomrule
\end{tabular}
\end{table}

\subsection{Dependencies}

\begin{lstlisting}
torch >= 2.0 (float64 mode)
numpy, scipy, matplotlib, tqdm
cupy-cuda12x (optional, for spectral analysis)
\end{lstlisting}

\subsection{To reproduce}

\begin{enumerate}
\item Open \texttt{notebooks/K7\_PINN\_Step5\_Reconstruction\_v3.ipynb} in Google Colab
\item Select A100 GPU runtime
\item Run all cells
\item Results exported to \texttt{k7\_pinn\_step5\_results\_v3.json}
\end{enumerate}

No manual intervention required.

% ============================================
% REFERENCES
% ============================================

% ============================================
\subsection*{Related Works}\label{sec:related}
\addcontentsline{toc}{subsection}{Related Works}
% ============================================

The analytical target metric and period integrals used in this paper derive from the GIFT (Geometric Information Field Theory) framework:

\begin{itemize}
\item de La Fourni\`ere, B. (2026). \textit{Geometric Information Field Theory v3.3}. Technical report. \url{https://github.com/gift-framework}.
\item de La Fourni\`ere, B. (2026). \textit{A parameter-free mollified approximation to the argument of the Riemann zeta function}. Preprint, in preparation.
\end{itemize}

While the mathematical objects produced by GIFT (the $\Gtwo$ decomposition, the Mayer--Vietoris splitting of moduli, and the determinant formula $\det(g) = 65/32$) serve as input data here, the physical claims of that framework are outside the scope of this paper. The metric verification criteria (determinant, torsion, positive definiteness) are independent of GIFT.

% ============================================
% REFERENCES
% ============================================

\begin{thebibliography}{99}

\bibitem{harvey1982} Harvey, R. \& Lawson, H.B. (1982). Calibrated geometries. \textit{Acta Math.} 148, 47--157.

\bibitem{bryant1987} Bryant, R.L. (1987). Metrics with exceptional holonomy. \textit{Ann.\ Math.} 126(3), 525--576.

\bibitem{joyce1996} Joyce, D.D. (1996). Compact Riemannian 7-manifolds with holonomy $\Gtwo$. I, II. \textit{J.\ Diff.\ Geom.} 43(2), 291--328 and 329--375.

\bibitem{joyce2000} Joyce, D.D. (2000). \textit{Compact Manifolds with Special Holonomy}. Oxford University Press.

\bibitem{kovalev2003} Kovalev, A.G. (2003). Twisted connected sums and special Riemannian holonomy. \textit{J.\ Reine Angew.\ Math.} 565, 125--160.

\bibitem{chnp2015} Corti, A., Haskins, M., Nordstr\"om, J. \& Pacini, T. (2015). $\Gtwo$-manifolds and associative submanifolds via semi-Fano 3-folds. \textit{Duke Math.\ J.} 164(10), 1971--2092.

\bibitem{raissi2019} Raissi, M., Perdikaris, P. \& Karniadakis, G.E. (2019). Physics-informed neural networks: A deep learning framework for solving forward and inverse problems involving nonlinear partial differential equations. \textit{J.\ Comput.\ Phys.} 378, 686--707.

\bibitem{cai2021} Cai, S. et al. (2021). Physics-informed neural networks (PINNs) for fluid mechanics: A review. \textit{Acta Mechanica Sinica} 37, 1727--1738.

\bibitem{hermann2020} Hermann, J. et al. (2020). Deep-neural-network solution of the electronic Schr\"odinger equation. \textit{Nature Chemistry} 12, 891--897.

\bibitem{liao2023} Liao, S. \& Petzold, L. (2023). Physics-informed neural networks for solving Einstein field equations. Preprint, arXiv:2302.10696.

\bibitem{braun2018} Braun, A.P., Del Zotto, M., Halverson, J., Larfors, M., Morrison, D.R. \& Sch\"afer-Nameki, S. (2018). Infinitely many M2-instanton corrections to M-theory on $\Gtwo$-manifolds. \textit{JHEP} 2018, 101.

\bibitem{lotay2019} Lotay, J.D. \& Wei, Y. (2019). Laplacian flow for closed $\Gtwo$ structures: Shi-type estimates, uniqueness and compactness. \textit{Geom.\ Funct.\ Anal.} 29, 1048--1110.

\bibitem{brandhuber2001} Brandhuber, A., Gomis, J., Gubser, S.S. \& Gukov, S. (2001). Gauge theory at large $N$ and new $\Gtwo$ holonomy metrics. \textit{Nuclear Phys.\ B} 611, 179--204.

\end{thebibliography}

\vfill
\noindent\rule{\textwidth}{0.2pt}
\textit{GIFT Framework --- Explicit $\Gtwo$ Metric}\\
\textit{Manuscript prepared March 2026.}

\end{document}
