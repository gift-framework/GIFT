\documentclass{article}
\usepackage{amsmath,amssymb,amsthm,booktabs}
\newtheorem{theorem}{Theorem}
\newtheorem{lemma}{Lemma}
\newtheorem{proposition}{Proposition}

\title{Computer-Assisted Existence Proof for GIFT v2.2 G$_2$ Geometry}
\author{Generated by RigorousVerifier}
\date{\today}

\begin{document}
\maketitle

\begin{abstract}
We present a computer-assisted proof of the existence of a G$_2$-structure
on the compact 7-manifold $K_7$ satisfying the GIFT v2.2 constraints.
Using interval arithmetic and Joyce's deformation theorem, we establish
rigorous bounds that guarantee existence.
\end{abstract}

\section{Mathematical Framework}

\begin{theorem}[Joyce, 2000]
Let $(M^7, \phi_0, g_0)$ be a compact 7-manifold with G$_2$-structure.
If the torsion satisfies $\|T(\phi_0)\|_{C^0} < \epsilon_0$ for sufficiently
small $\epsilon_0$, then there exists a torsion-free G$_2$-structure $\phi$
with $\|\phi - \phi_0\|_{C^{2,\alpha}} \leq K\epsilon_0$.
\end{theorem}

\section{Numerical Solution}

The Physics-Informed Neural Network produces $\phi_0$ satisfying:
\begin{itemize}
\item $\det(g(\phi_0)) \in [2.030902, 2.031534]$
\item Target: $\det(g) = 65/32 = 2.03125$
\item Relative error: $<0.0171\%$
\end{itemize}

\section{Rigorous Bounds}

Using interval arithmetic (Tucker, 2011), we compute:

\subsection{Metric Positivity}
\begin{equation}
\lambda_{\min}(g(\phi_0)) \in [1.078417, 1.078417]
\end{equation}
Since the lower bound is positive, $g(\phi_0)$ is rigorously positive definite.

\subsection{Torsion Bounds}
\begin{equation}
\|T(\phi_0)\|^2 \in [0.00e+00, 1.08e+04]
\end{equation}

\section{Existence Proof}

The Joyce theorem threshold is not met with current bounds.
Existence requires either tighter numerical bounds or an alternative theorem.

\section{Conclusion}

PARTIALLY VERIFIED: All constraints satisfied but Joyce theorem threshold not met. The solution is numerically valid but existence requires tighter bounds.

\begin{thebibliography}{9}
\bibitem{joyce} Joyce, D. (2000). \textit{Compact Manifolds with Special Holonomy}. Oxford.
\bibitem{tucker} Tucker, W. (2011). \textit{Validated Numerics}. Princeton.
\bibitem{hales} Hales, T. (2005). A proof of the Kepler conjecture. \textit{Annals of Math.}
\end{thebibliography}

\end{document}
